% &latex
\documentclass[preprint,12pt]{elsarticle}
%% Use the option review to obtain double line spacing
%% \documentclass[preprint,review,12pt]{elsarticle}

%% Use the options 1p,twocolumn; 3p; 3p,twocolumn; 5p; or 5p,twocolumn
%% for a journal layout:
%% \documentclass[final,1p,times]{elsarticle}
%% \documentclass[final,1p,times,twocolumn]{elsarticle}
%% \documentclass[final,3p,times]{elsarticle}
%% \documentclass[final,3p,times,twocolumn]{elsarticle}
%% \documentclass[final,5p,times]{elsarticle}
%% \documentclass[final,5p,times,twocolumn]{elsarticle}

%% if you use PostScript figures in your article
%% use the graphics package for simple commands
%% \usepackage{graphics}
%% or use the graphicx package for more complicated commands
%% \usepackage{graphicx}
%% or use the epsfig package if you prefer to use the old commands
%% \usepackage{epsfig}

%% The lineno packages adds line numbers. Start line numbering with
%% \begin{linenumbers}, end it with \end{linenumbers}. Or switch it on
%% for the whole article with \linenumbers after \end{frontmatter}.
\usepackage{lineno}

%% natbib.sty is loaded by default. However, natbib options can be
%% provided with \biboptions{...} command. Following options are
%% valid:

%%   round  -  round parentheses are used (default)
%%   square -  square brackets are used   [option]
%%   curly  -  curly braces are used      {option}
%%   angle  -  angle brackets are used    <option>
%%   semicolon  -  multiple citations separated by semi-colon
%%   colon  - same as semicolon, an earlier confusion
%%   comma  -  separated by comma
%%   numbers-  selects numerical citations
%%   super  -  numerical citations as superscripts
%%   sort   -  sorts multiple citations according to order in ref. list
%%   sort&compress   -  like sort, but also compresses numerical citations
%%   compress - compresses without sorting
%%
%% \biboptions{comma,round}

% \biboptions{}

\usepackage{pstricks}
\usepackage{multido,pst-plot,pst-node}

\usepackage{epic,eepic}
\usepackage[centertags]{amsmath}
\usepackage{amsfonts}
\usepackage{amssymb}
\usepackage{newlfont}
\usepackage[inference]{semantic}\setpremisesend{0pt}
\usepackage{alltt}
\usepackage{enumerate}
\usepackage[hypertex]{hyperref} %[hypertex] is the driver for dviout
\usepackage{varioref}
\usepackage{stmaryrd}
\usepackage{xypic}
\usepackage{subfig}
%\usepackage{endfloat}

%\hfuzz 0.1pt
%\overfullrule=15pt
%\brokenpenalty=10000

\newtheorem{definition}{Definition}[section]
\newtheorem{theorem}{Theorem}[section]
\newtheorem{proposition}{Proposition}[section]
\newtheorem{lemma}{Lemma}[section]
\newtheorem{rem}{Remark}[section]
\newtheorem{proof}{Proof}[section]

\newcommand{\prop}{\mathrm{Prop}}
\newcommand{\bocks}[1]{[#1]}
\newcommand{\lift}[1]{\widehat{#1}}
\newcommand{\nlift}[1]{\tilde{#1}}
\newcommand{\sound}{\mathrm{sound}}
\newcommand{\liftok}{\mathrm{lift_\beta}}
\newcommand{\qind}{\mathrm{qind}}
\newcommand{\exact}{\mathrm{exact}}
\newcommand{\subst}{\mathrm{subst}}
\newcommand{\Set}{\mathrm{Set}}
\newcommand{\emb}{\mathrm{emb}}
\newcommand{\compl}{\mathrm{complete}}
\newcommand{\stable}{\mathrm{stable}}
\newcommand{\List}{\mathrm{List}}
\newcommand{\Fin}{\mathrm{Fin}}
\newcommand{\Z}{\mathbb{Z}}
\newcommand{\pullbackcorner}[1][dr]{\save*!/#1-1.2pc/#1:(-1,1)@^{|-}\restore}
\newcommand{\pushoutcorner}[1][dr]{\save*!/#1+1.2pc/#1:(1,-1)@^{|-}\restore}
\newcommand{\todo}[1]{\textcolor{red}{#1}}
\newcommand{\now}{\mathrm{now}}
\newcommand{\later}{\mathrm{later}}
\newcommand{\noweq}{\mathrm{now}_\sqsubseteq}
\newcommand{\latereq}{\mathrm{later}_\sqsubseteq}
\newcommand{\laterleft}{\mathrm{later}_{\mathrm{left}}}
\newcommand{\inl}{\mathop{\mathrm{inl}}}
\newcommand{\inr}{\mathop{\mathrm{inr}}}
\journal{???}

\begin{document}

\renewcommand{\above}{\sqsupseteq} \renewcommand{\epsilon}{\varepsilon}
\renewcommand{\vec}{\overrightarrow}
\renewcommand{\emptyset}{\varnothing}

\providecommand{\bigcupbar}{\bigcup}
\providecommand{\cupbar}{\cup}
\providecommand{\downarrowvar}{\mathrel{\downarrow}}
\providecommand{\llbrace}{\mathopen{{\{\hspace{-3.5pt}|}}}
\providecommand{\real}{\mathbb{R}}
\providecommand{\rrbrace}{\mathopen{{|\hspace{-3.5pt}\}}}}
\providecommand{\uparrowvar}{\mathrel{\uparrow}}
\providecommand{\waydownarrow}{\mathrel{\downarrow\hspace{-9.3pt}\raisebox{2pt}{\(\downarrow\)}}}
\providecommand{\wayuparrow}{\mathrel{\uparrow\hspace{-9.3pt}\raisebox{2pt}{\(\uparrow\)}}}

\newcommand{\colonp}{\mathrel{\colon\hspace{-1.5ex}\subseteq}}
\newcommand{\coloneq}{\mathrel{\colon\hspace{-1.5ex}=}}
\newcommand{\tp}{\colon}
\newcommand{\Bool}{\mathtt{bool}}
\newcommand{\Bound}{\mathop{\mathtt{bound}}\nolimits}
\newcommand{\B}{{\mathbb B_\bot}}
\newcommand{\Cons}{\mathop{\mathtt{cons}}\nolimits}
\newcommand{\Definition}[1]{\emph{#1}}
\newcommand{\D}{D}
\newcommand{\False}{\mathop{\mathtt{false}}}
\newcommand{\ISZEROfalse}{\Rulec[\redRNiszerofalse]{}{\Iszero(\K{n+1})\redto \False}}
\newcommand{\ISZEROtrue}{\Rulec[\redRNiszerotrue]{}{\Iszero{(\K{0})}\redto \True}}
\newcommand{\ISZEROweak}{\Rulec[\redRNiszeroweak]{ L\redto  L'}{(=0)(L) \redto(=0)(L')}}
\newcommand{\Iftheni}[3]{\operatorname{\mathtt{if}} {#1} \mathrel{\operatorname{\mathtt{then}}}{#2}\mathrel{\operatorname{\mathtt{else}}}{#3}}
\newcommand{\Piftheni}[3]{\operatorname{\mathtt{pif}} {#1} \mathrel{\operatorname{\mathtt{then}}}{#2}\mathrel{\operatorname{\mathtt{else}}}{#3}}
\newcommand{\Ifthenl}[3]{\operatorname{\mathtt{if}}#1\begin{array}[t]{ll}\operatorname{\mathtt{then}}&#2\\\operatorname{\mathtt{else}}&#3\end{array}}
\newcommand{\Iszero}{({0}=)}
\newcommand{\K}[1]{\underline{#1}}
\newcommand{\NQ}{\rc{\mathcal{R}}}
\newcommand{\Nat}{\mathtt{nat}}
\newcommand{\Nif}{\operatorname{\mathtt{if}}}
\newcommand{\N}{{\mathbb N_\bot}}
\newcommand{\OR}[3][]{{#2}\mathrel{\mathtt{OR_{#1}}}{#3}}
\newcommand{\PSB}{\mathsf{B}}\newcommand{\PSN}{\mathsf{N}}\newcommand{\PSQ}{\mathsf{Q}}\newcommand{\PSR}{\mathsf{R}}\newcommand{\PSZ}{\mathsf{Z}}\newcommand{\pws}{\mathcal{P}}
\newcommand{\Pred}{\mathtt{pred}}
\newcommand{\Q}{\mathbb Q}
\newcommand{\REL}[3]{#1\subseteq#2\times#3}
\newcommand{\RQ}{\R_\mathbb{Q}}
\newcommand{\RT}{\widetilde{\mathcal R}}
\newcommand{\Real}{\mathtt{real}}
\newcommand{\Rtest}{\mathop{\mathtt{rtest}}\nolimits}
\newcommand{\Rulec}[4][]{\inference[$#1$]{#2}{#3}#4} \newcommand{\ruleetc}{\vdots}
\newcommand{\Rule}[3][]{\inference[]{#2}{#3}} \newcommand{\R}{\mathcal{R}}
\newcommand{\SM}[1]{#1^{\mathcal S}}
\newcommand{\Succ}{\mathtt{succ}}
\newcommand{\True}{\mathop{\mathtt{true}}}
\newcommand{\TtoR}{\rho_{\ast}}
\newcommand{\WTReal}{\widetilde\Real}
\newcommand{\WTR}{\widetilde\PSR}
\newcommand{\Wpor}{\operatorname{\texttt{wpor}}}
\newcommand{\Y}{\mathtt{Y}}
\newcommand{\absa}{\operatorname{abs}}
\newcommand{\abs}[1]{{|#1|}}
\newcommand{\adda}{\operatorname{add}}
\newcommand{\addinta}{\operatorname{add_{\mathbb{Z}}}}
\newcommand{\approxa}{\operatorname{bound}\nolimits''}
\newcommand{\approxb}{\operatorname{\phi}}
\newcommand{\approxc}{\operatorname{\psi}}
\newcommand{\below}{\sqsubseteq}
\newcommand{\bnd}{\overrightarrow{\sqcup}}
\newcommand{\bool}{\mathbb{B}}
\newcommand{\bound}{\operatorname{bound}}
\newcommand{\brBIG}[1]{\Bigg(#1\Bigg)}
\newcommand{\brBIg}[1]{\bigg(#1\bigg)}
\newcommand{\brBig}[1]{\Big(#1\Big)}
\newcommand{\brbig}[1]{\big(#1\big)}
\newcommand{\br}[1]{\left(#1\right)}
\newcommand{\case}[4]{\Case{#1}{#2}{#3}{#4}}
\newcommand{\Case}[4]{\operatorname{\operatorname{cases}}\begin{array}[t]{l@{\quad\rightarrow\quad}l}#1&#2\\#3&#4\end{array}}
\newcommand{\ccase}[4]{\operatorname{cases}\begin{array}[t]{l@{\quad\rightarrow\quad}l}#1&#2\\#3&#4\end{array}}
\newcommand{\cas}[4]{\setlength{\arraycolsep}{0pt}\begin{array}[t]{l}\operatorname{cases}\\\setlength{\arraycolsep}{1ex}\begin{array}[t]{l@{\quad\rightarrow\quad}l}#1&#2\\#3&#4\end{array}\end{array}}
\newcommand{\casA}[4]{\setlength{\arraycolsep}{0pt}\begin{array}[t]{l}\operatorname{cases}\\\setlength{\arraycolsep}{0ex}\begin{array}[t]{l@{\ \ \rightarrow\ \  }l}#1&#2\\#3&#4\end{array}\end{array}}
\newcommand{\casB}[4]{\setlength{\arraycolsep}{0pt}\begin{array}[t]{l}\setlength{\arraycolsep}{0ex}\begin{array}[t]{l@{\  \rightarrow\  }l}\mid#1&#2\\\mid#3&#4\end{array}\end{array}}
\newcommand{\cf}{\textit{cf}.}
\newcommand{\circc}{\mathrel{\circ}}
\newcommand{\cons}{\operatorname{cons}\nolimits}
\newcommand{\contains}{\supseteq}
\newcommand{\curry}{\operatorname{curry}}
\newcommand{\dYY}{\mathop{\mu}\nolimits}
\newcommand{\denapp}{\(\den{\Gamma\vdash{M(N)\tp\tau}}(\vec{d})=\den{\Gamma\vdash{M\tp\sigma}\to\tau}(\vec{d})\Big(\den{\Gamma\vdash{N\tp\sigma}}(\vec{d})\Big)\)}
\newcommand{\denbound}{\den{\Gamma\vdash\Bound_a(M)\tp\Real}(\vec{d})=\left(\smyth(\bound_a)\right)\left(\den{\Gamma\vdash{M\tp\Real}}\right)}
\newcommand{\denfalse}{\den{\Gamma\vdash\False\tp\Bool}(\vec{d})=\set{{\ffalse}}}
\newcommand{\denifr}{=\begin{cases}\den{\Gamma\vdash{M\tp\gamma}}(\vec{d}) & \text{if $\den{\Gamma\vdash{B\tp\Bool}}(\vec{d})=\set{\ttrue}$}\\\den{\Gamma\vdash{N\tp\gamma}}(\vec{d}) & \text{if $\den{\Gamma\vdash{B\tp\Bool}}(\vec{d})=\set{\ffalse}$}\\ {\den{\Gamma\vdash{M\tp\gamma}}(\vec{d})}
\\\quad{}\cup\quad{\den{\Gamma\vdash{N\tp\gamma}}(\vec{d})} & \text{if $\den{\Gamma\vdash{B\tp\Bool}}(\vec{d})=\set{\ttrue,\ffalse}$}\\\bot_{D_\gamma} & \text{otherwise}\end{cases}}
\newcommand{\denif}{\den{\Gamma\vdash\Nif_\gamma(B, M, N)\tp\gamma}(\vec{d})}
\newcommand{\deniszero}{\den{\Gamma\vdash{\Iszero(M)\tp\Bool}}(\vec{d})=\left(\smyth(\iszero)\right)\left(\den{\Gamma\vdash{M\tp\Nat}}\right)}
\newcommand{\denkn}{\den{\Gamma\vdash\K{n}\tp\Nat}(\vec{d})=\set{{n}}}
\newcommand{\denlambda}{\(\den{\Gamma\vdash(\lambda{x}\tp\sigma.M)\tp\sigma\to\tau}(\vec{d})(e)=\den{\Gamma,x\tp\sigma\vdash{M\tp\tau}}(\vec{d},e)\)}
\newcommand{\denom}{\operatorname{denominator}}
\newcommand{\denomd}{\operatorname{denomin.}}
\newcommand{\denpcffalse}{\den{\Gamma\vdash\False\tp\Bool}(\vec{d})={{\ffalse}}}
\newcommand{\denpcfifr}{=\begin{cases}\den{\Gamma\vdash{M\tp\gamma}}(\vec{d}) & \text{if $\den{\Gamma\vdash{B\tp\Bool}}(\vec{d})={\ttrue}$}\\\den{\Gamma\vdash{N\tp\gamma}}(\vec{d}) & \text{if $\den{\Gamma\vdash{B\tp\Bool}}(\vec{d})={\ffalse}$}\\\bot_{D_\gamma} & \text{otherwise}\end{cases}}
\newcommand{\denpcfiszero}{$\den{\Gamma\vdash{\Iszero(M)\tp\Bool}}(\vec{d})=(\iszero)\br{\den{\Gamma\vdash{M\tp\Nat}}}$}
\newcommand{\denpcfkn}{\den{\Gamma\vdash\K{n}\tp\Nat}(\vec{d})={{n}}}
\newcommand{\denpcfpred}{\den{\Gamma\vdash{\Pred(M)\tp\Nat}}(\vec{d})=\ppred\left(\den{\Gamma\vdash{M\tp\Nat}}\right)}
\newcommand{\denpcfsucc}{\den{\Gamma\vdash{\Succ(M)\tp\Nat}}(\vec{d})=\ssucc\left(\den{\Gamma\vdash{M\tp\Nat}}\right)}
\newcommand{\denpcftrue}{\den{\Gamma\vdash\True\tp\Bool}(\vec{d})={{{\ttrue}}}}
\newcommand{\denplus}{\den{\Gamma\vdash{p+(M)\tp\Real}}(\vec{d})=\left(\smyth(p+)\right)\left(\den{\Gamma\vdash{M\tp\Real}}\right)}
\newcommand{\denpred}{\den{\Gamma\vdash{\Pred(M)\tp\Nat}}(\vec{d})=\left(\smyth(\ppred)\right)\left(\den{\Gamma\vdash{M\tp\Nat}}\right)}
\newcommand{\densucc}{\den{\Gamma\vdash{\Succ(M)\tp\Nat}}(\vec{d})=\left(\smyth(\ssucc)\right)\left(\den{\Gamma\vdash{M\tp\Nat}}\right)}
\newcommand{\dentimes}{\den{\Gamma\vdash{p\times(M)\tp\Real}}(\vec{d})=\left(\smyth(p\times)\right)\left(\den{\Gamma\vdash{M\tp\Real}}\right)}
\newcommand{\dentrue}{\den{\Gamma\vdash\True\tp\Bool}(\vec{d})=\set{{{\ttrue}}}}
\newcommand{\denttest}{\den{\Gamma\vdash\Rtest(L)\tp\Bool}(\vec{d})=\ttest^{\ast}\Big(\den{\Gamma\vdash{L\tp\Real}}(\vec{d})\Big)}
\newcommand{\denvar}{\den{x_1\tp\sigma_1,\ldots,x_k\tp\sigma_k\vdash x_i\tp\sigma_i}(\vec{d}) = d_i}
\newcommand{\deny}{\den{\Gamma\vdash\Y_\sigma(M)\tp\sigma}(\vec{d})=\dYY_{D_\sigma}\left(\den{\Gamma\vdash{M\tp\sigma\to\sigma}}(\vec{d})\right)}
\newcommand{\den}[1]{\left\llbracket #1 \right\rrbracket}
\newcommand{\dirsup}{\bigsqcup}
\newcommand{\divinta}{\operatorname{div_\mathbb{Z}}}
\newcommand{\divintb}{\operatorname{\phi_1}}
\newcommand{\divintc}{\operatorname{div_\mathbb{N}}}
\newcommand{\divintd}{\operatorname{div_\mathbb{Q}}}
\newcommand{\dom}{\operatorname{dom}}
\newcommand{\downset}{\mathop{\downarrowvar}}
\newcommand{\dtest}{\operatorname{dtest}}
\newcommand{\eelse}{\operatorname{else}}
\newcommand{\escardo}{Escard\'{o}}
\newcommand{\ffalse}{\operatorname{false}}
\newcommand{\hoare}{\mathop{{\mathcal P}^{\mathcal H}}\nolimits}
\newcommand{\ida}{\operatorname{id}'}
\newcommand{\id}{\operatorname{id}}
\newcommand{\ie}{\textit{i.e.~}}
\newcommand{\iftheni}[3]{\operatorname{if}{#1}\operatorname{then}{#2}\operatorname{else}{#3}}
\newcommand{\ifthenll}[3]{\setlength{\arraycolsep}{5pt}\begin{array}[t]{ll}{\mathop{\operatorname{if}}}&#1\\{\mathop{\operatorname{then}}}&#2\\{\mathop{\operatorname{else}}}&#3\end{array}}
\newcommand{\ifthenl}[3]{\operatorname{if}{\,\,{#1}}{\begin{array}[t]{rl}\operatorname{then}&#2\\ \operatorname{else}&#3\end{array}}}
\newcommand{\infa}{\operatorname{inf}\nolimits}
\newcommand{\intervDU}{{\interv{-2}{2}}}
\newcommand{\intervHU}{\interv[2]{-1}{1}}
\newcommand{\intervU}{{\interv{-1}{1}}}
\newcommand{\interv}[2]{\left[#1,#2\right]}
\newcommand{\iszero}{\operatorname{0=}}
\newcommand{\jdgapp}{\Rule{\Gamma,x\tp\sigma\vdash M\tp\tau}{\Gamma\vdash(\lambda x\tp\sigma. M)\tp\sigma\to\tau}}
\newcommand{\jdgbound}{\Rule{\Gamma\vdash M\tp\Real}{\Gamma\vdash \Bound_a( M) \tp\Real}}
\newcommand{\jdgfalse}{\Rule{}{\Gamma\vdash\False\tp\Bool}}
\newcommand{\jdgif}{\Rule{\Gamma\vdash{L\tp\Bool},&\Gamma\vdash{M\tp\gamma},&\Gamma\vdash{N\tp\gamma}}{\Gamma\vdash\Nif_\gamma(L,M,N)\tp\gamma}}
\newcommand{\jdgiszero}{\Rule{\Gamma\vdash L\tp\Nat}{\Gamma\vdash\Iszero\br{L}\tp\Bool}}
\newcommand{\jdglambda}{\Rule{\Gamma\vdash M\tp\sigma\to\tau & \Gamma\vdash N\tp\sigma}{\Gamma\vdash M( N)\tp\tau}}
\newcommand{\jdgplus}{\Rule{\Gamma\vdash M\tp\Real}{\Gamma\vdash p+(M)\tp\Real}}
\newcommand{\jdgpred}{\Rule{\Gamma\vdash M\tp\Nat}{\Gamma\vdash \Pred{(M)}\tp\Nat}}
\newcommand{\jdgrtest}{\Rule{\Gamma\vdash  L\tp\Real}{\Gamma\vdash\Rtest( L)\tp\Bool}}
\newcommand{\jdgsucc}{\Rule{\Gamma\vdash M\tp\Nat}{\Gamma\vdash \Succ{(M)}\tp\Nat}}
\newcommand{\jdgtimes}{\Rule{\Gamma\vdash M\tp\Real}{\Gamma\vdash p\times( M)\tp\Real}}
\newcommand{\jdgtrue}{\Rule{}{\Gamma\vdash\True\tp\Bool}}
\newcommand{\jdgvar}{\Rule{}{\Gamma, x\tp\sigma\vdash x\tp\sigma}}
\newcommand{\jdgy}{\Rule{\Gamma\vdash M\tp\sigma\to\sigma}{\Gamma\vdash\Y_\sigma( M)\tp\sigma}}
\newcommand{\jdgzero}{\Rule{}{\Gamma\vdash \K{n}\tp\Nat}}
\newcommand{\lb}[1]{\underline{#1}}
\newcommand{\letin}[2]{\begin{array}[t]{ll}{\mathop{let}#1}\\{\mathop{in}#2}\end{array}}
\newcommand{\lima}{\operatorname{lim}}\newcommand{\limb}{\operatorname{\phi}}
\newcommand{\limc}{{\lim\nolimits'}_\infty}
\newcommand{\multinta}{\operatorname{mult_{\mathbb{Z}}}}
\newcommand{\natbnd}{\operatorname{natBound}}
\newcommand{\nat}{\mathbb {N}}
\newcommand{\nif}{\operatorname{if}}
\newcommand{\ntort}{\operatorname{\widetilde{\iota_{\mathbb{N},\mathbb{R}}}}}
\newcommand{\ntor}{\operatorname{\iota_{\mathbb{N},\mathbb{R}}}}
\newcommand{\num}{\operatorname{numerator}}
\newcommand{\opsem}[1]{\left[ #1\right] }
\newcommand{\op}[1]{\operatorname{#1}\nolimits}
\newcommand{\order}{\operatorname{order}}
\newcommand{\outputs}{\operatorname{Outputs}}
\newcommand{\out}[1]{|#1|}
\newcommand{\piftheni}[4][black]{\mathop{\textcolor{#1}{\pif}} #2 \mathrel{\tthen} #3 \mathrel{\eelse} #4}
\newcommand{\pif}{\operatorname{pif}}
\newcommand{\positive}{\operatorname{\phi}}
\newcommand{\ppred}{\operatorname{pred}}
\newcommand{\prg}{\mathtt{Prg}}
\newcommand{\qtort}{\operatorname{\widetilde{\iota_{\mathbb{Q},\mathbb{R}}}}}
\newcommand{\qtor}{\operatorname{\iota_{\mathbb{Q},\mathbb{R}}}}
\newcommand{\rat}{\mathbb{Q}}
\newcommand{\rc}[1]{\widehat{#1}}
\newcommand{\redAPP}{\Rulec[\redRNapp]{ M\redto M'}{ M( N)\redto M'( N)}}
\newcommand{\redBOUNDweak}{\Rulec[]{ M\redto M'}{\Bound_a( M)\redto\Bound_a( M')}{\out{ M}=\bot}}
\newcommand{\redBOUND}{\Rulec[]{}{\Bound_a\big(\Bound_b( M)\big) \redto \Bound_{\bound_a(b)}( M)}}
\newcommand{\redLAMBDA}{\Rulec[\redRNlambda]{}{(\lambda x\tp\sigma. M)( N)\redto M[ N/x]}}
\newcommand{\redNIFfalse}{\Rulec[\redRNiffalse]{}{\Nif(\False, M, N)\redto N}}
\newcommand{\redNIFtrue}{\Rulec[\redRNiftrue]{}{\Nif(\True, M, N) \redto M}}
\newcommand{\redNIFweak}{\Rulec[\redRNif]{B\redto B'}{\Nif(B,M,N) \redto \Nif(B',M,N)}}
\newcommand{\redPLUSbound}{\Rulec[]{}{p+\big(\Bound_a( M)\big)\redto\Bound_{p+a}\big(p+( M)\big)}}
\newcommand{\redPLUSweak}{\Rulec[]{ M\redto M'}{p+( M)\redto p+( M')}{\out{ M}=\bot}}
\newcommand{\redPREDweak}{\Rulec[\redRNpredweak]{ M\redto M'}{\Pred{(M)}\redto \Pred( M')}}
\newcommand{\redPREDzero}{\Rulec[\redRNpredzero]{}{\Pred{(\K{0})}\redto \K{0}}}
\newcommand{\redPRED}{\Rulec[\redRNpred]{}{\Pred{(\K{n+1})}\redto \K{n}}}
\newcommand{\redRNapp}{\redRNformat{3}}
\newcommand{\redRNboundbound}{\redRNformat{15}}
\newcommand{\redRNbound}{\redRNformat{16}}
\newcommand{\redRNformat}[1]{(#1)}
\newcommand{\redRNiffalse}{\redRNformat{5}}
\newcommand{\redRNiftrue}{\redRNformat{4}}
\newcommand{\redRNif}{\redRNformat{6}}
\newcommand{\redRNiszerofalse}{\redRNformat{8}}
\newcommand{\redRNiszerotrue}{\redRNformat{7}}
\newcommand{\redRNiszeroweak}{\redRNformat{9}}
\newcommand{\redRNlambda}{\redRNformat{2}}
\newcommand{\redRNplusbound}{\redRNformat{17}}
\newcommand{\redRNplus}{\redRNformat{18}}
\newcommand{\redRNpredweak}{\redRNformat{14}}
\newcommand{\redRNpredzero}{\redRNformat{12}}
\newcommand{\redRNpred}{\redRNformat{13}}
\newcommand{\redRNrtestfalse}{\redRNformat{22}}
\newcommand{\redRNrtesttrue}{\redRNformat{21}}
\newcommand{\redRNrtest}{\redRNformat{23}}
\newcommand{\redRNsuccweak}{\redRNformat{11}}
\newcommand{\redRNsucc}{\redRNformat{10}}
\newcommand{\redRNtimesbound}{\redRNformat{19}}
\newcommand{\redRNtimes}{\redRNformat{20}}
\newcommand{\redRNy}{\redRNformat{1}}
\newcommand{\redRTESTfalse}{\Rulec[\redRNrtestfalse]{}{\Rtest(L)\redto \False}{0<\out{L}}}
\newcommand{\redRTESTtrue}{\Rulec[\redRNrtesttrue]{}{\Rtest( L)\redto \True}{\out{L}<1}}
\newcommand{\redRTESTweak}{\Rulec[\redRNrtest]{ L\redto  L'}{\Rtest( L) \redto\Rtest( L')}}
\newcommand{\redSUCCweak}{\Rulec[\redRNsuccweak]{ M\redto M'}{\Succ{(M)}\redto \Succ( M')}}
\newcommand{\redSUCC}{\Rulec[\redRNsucc]{}{\Succ{(\K{n})}\redto \K{n+1}}}
\newcommand{\redTIMESbound}{\Rulec[]{}{p\times\big(\Bound_a(M)\big)\redto\Bound_{p\times{a}}\big(p\times(M)\big)}}
\newcommand{\redTIMESweak}{\Rulec[]{ M\redto M'}{p\times( M)\redto p\times( M')}{\out{ M}=\bot}}
\newcommand{\redY}{\Rulec[\redRNy]{}{\Y_\sigma( M)\redto M(\Y_\sigma( M))}}
\newcommand{\redto}[1][]{\mathrel{\shortrightarrow^{#1}}}
\newcommand{\rel}{\leftrightarrow}
\newcommand{\romero}{Marcial-Romero}
\newcommand{\roota}{\operatorname{root}}
\newcommand{\rtest}{\operatorname{rtest}}
\newcommand{\sbr}{\mathrel{\colon\subseteq}}
\newcommand{\setfunc}[1]{\dot{#1}}
\newcommand{\set}[1]{\left\{#1\right\}}
\newcommand{\sg}{\set}
\newcommand{\singleton}[1]{\llbrace#1\rrbrace}
\newcommand{\smyth}{\mathop{{\mathcal P}^{\mathcal S}}\nolimits}
\newcommand{\sqbrbig}[1]{\big[#1\big]}
\newcommand{\sqbr}[1]{\left[#1\right]}
\newcommand{\sqrta}{\operatorname{sqrt}}
\newcommand{\squarea}{\operatorname{square}}
\newcommand{\squareb}{\operatorname{square'}}
\newcommand{\ssucc}{\operatorname{succ}}
\newcommand{\supa}{\operatorname{sup}\nolimits}
\newcommand{\tail}{\operatorname{tail}\nolimits}
\newcommand{\ttest}{\operatorname{ttest}\nolimits}
\newcommand{\tthen}{\operatorname{then}}
\newcommand{\ttrue}{\operatorname{true}}
\newcommand{\ub}[1]{\overline{#1}}
\newcommand{\uncurry}{\operatorname{uncurry}}
\newcommand{\under}{\unlhd}
\newcommand{\unit}[1]{\mathop{\eta}\left(#1\right)}
\newcommand{\upset}{\mathop{\uparrow}}
\newcommand{\vs}{\vec\sigma}
\newcommand{\vxs}{\vec{x\tp\sigma}}
\newcommand{\vx}{\vec{x}}
\newcommand{\wayabove}{\gg}
\newcommand{\waybelow}{\ll}
\newcommand{\waydown}{\mathop{\waydownarrow}}
\newcommand{\wayup}{\mathop{\wayuparrow}}
\newcommand{\wbound}{\widetilde{bound}}
\newcommand{\wcons}{\widetilde{cons}}
\newcommand{\witnessa}{\operatorname{witness}'}
\newcommand{\witnessb}{\phi}
\newcommand{\witnessc}{\operatorname{witness}}
\newcommand{\wminus}{\mathop{\widetilde{-}}}
\newcommand{\wplus}{\mathop{\widetilde{+}}}
\newcommand{\wpor}{\operatorname{wpor}}
\newcommand{\wtail}{\widetilde{tail}}
\newcommand{\wtest}{\widetilde{ttest}}
\newcommand{\wtimes}{\mathop{\widetilde{\times}}}
\newcommand{\zero}{\operatorname{zero}}
\newcommand{\ztor}{\operatorname{\iota_{\mathbb{Z},\mathbb{R}}}}
\newcommand{{\half}}{\mathop{\operatorname{half}}}


\begin{frontmatter}
\title{Definable quotients in type theory}
\author{TA,TA,LN}
\address{Division of Computer Science, University of Nottingham in Ningbo, China,199 Taikang Road East, 315100, Ningbo, China.}

\begin{abstract}

\end{abstract}

\begin{keyword}

\end{keyword}
\end{frontmatter}

%% Line numbers can be started here if you want.
%\linenumbers

\section{Introduction}\label{sec:introduction}
\subsection{Notation}
We keep obvious quantifications implicit.

Given $B : A \to \Set$, $b : B\,a$, $b' : B\,a'$ and $p : a\equiv a'$, we write $b \simeq_{p} b'$ for $\subst\,B\,p\,b \equiv b'$.
We write $\prop$ for the type of sets with at most one inhabitant.

\todo{define $\Fin$ and Bijection}  

\todo{logical symbols in Prop}


\section{Setoids}\label{sec:setoids}
\begin{definition}
A setoid $(A,\sim)$ is a set $A$ equipped with an equivalence relation $\sim : A \to A \to \prop$.
\end{definition}
\subsection{Examples}\label{sec:setoids:examples}
\subsubsection*{Integers}
The integers can be viewed as the setoid $(\Z_0=\nat\times\nat,\sim)$ where $(a,b)\sim(c,d)$ if{f} $a+d=c+b$ reflecting the idea that $(a,b)$ represents the integer $a-b$.
\subsubsection*{Rational numbers}
The rationals can in turn be defined as $(\Z\times\nat,\sim)$ where $(x,m)\sim(y,n)$ if{f} $x\times(n+1)=y\times(m+1)$, reflecting that $(x,m)$ represents the quotient $\frac {x}{m+1}$.
\subsubsection*{The real numbers}
The real numbers can then be defined as $(R_0,\sim)$ where $R_0$ is the set of Cauchy sequences and two sequences are equivalent if{f} their pointwise difference converges to $0$. 
\begin{align*}
R_0&=\set{s : \nat\to\Q \mid \forall\varepsilon :\Q,\varepsilon>0\to\exists m:\nat, \forall i:\nat, i>m\to |s\,i - s\, m|<\varepsilon}\\
r\sim s &= \forall\varepsilon :\Q,\varepsilon>0\to\exists m:\nat, \forall i:\nat, i>m\to |r\,i - s\,i|<\varepsilon
\end{align*}

\subsubsection*{Unordered pairs}
Given a set $A$, the unordered =airs of elements of $A$ is the setoid $(A\times A,\sim)$ where
$(a,b)\sim(b,a)$.

\subsubsection*{Finite multisets}
Given a set $A$ , the finite multisets of elements in $A$ is the setoid $(\List A,\sim)$ where two lists are equivalent if{f} one is the permutation of the other.
\begin{align*}
\List A &= \Sigma n:\nat.\Fin\,n\to A\\
(m,f)\sim(n,g) &= \exists \varphi : \Fin\,m \to \Fin\,n \cdot\ \mathrm{Bijection}\,\varphi \to g\circ\varphi = f  
\end{align*}

\subsubsection*{Finite sets}
Given a set $A$ , the finite sets of elements in $A$ is the setoid $(\List A,\sim)$ where two lists are equivalent if{f} they contains the same elements.
\begin{align*}
(m,f)\subseteq(n,g) &= \exists \varphi : \Fin\,m \to \Fin\,n \cdot \varphi \to g\circ\varphi = f  \\
(m,f)\sim(n,g)&= (m,f)\subseteq(n,g) \wedge (n,g)\subseteq(m,f)
\end{align*}

\subsubsection*{Partiality monad}
Given a set $A$, the set of partial computations over $A$ is given by $(A_{\bot_0},{\sim})$ where $A_{\bot_0}$ is the set of delayed computations over $A$  and $\sim$ is a weak bisimilarity ignoring finite delays. Using the notation for mixed inductive coinductive  definitions from~\cite{danielson:altenkirch:2010}, where we mark coinductive occurrences of a datatype by using $\infty$, we define $A_{\bot_0} : \Set$ and the relations $\sqsubseteq:A_{\bot_0}\to A_{\bot_0} \to \prop$ by the following constructors:
\begin{align*}
\now  &: A \to A_{\bot_0}\\
\later &: \infty A_{\bot_0} \to  A_{\bot_0}\\
\noweq &: \now\, a \sqsubseteq \now\,a'\\
\latereq &: \infty(d \sqsubseteq d') \to \later\,d \sqsubseteq \later\,d'\\
\laterleft &: d\sqsubseteq d' \to \later\,d \sqsubseteq d'
\end{align*}
and we define $d\sim d'= d\sqsubseteq d' \wedge d'\sqsubseteq d$ .

\section{Quotients and coequalizers}\label{sec:quotients}
 
\begin{definition}
\label{def:quotient}
Given a setoid $(A,\sim)$,  a \emph{quotient} over that setoid is given by

\begin{enumerate}[i.]
\item \label{enum:Q} a set $Q$,
\item \label{enum:box}a function $\bocks\cdot\colon A \to Q$ which is compatible with $\sim$, 
that is \[\sound\colon (a,b : A) \to a\sim b \to [a] \equiv [b],\]
\item \label{enum:dlift} a dependent lifting function for $B:Q\to\Set$
 \[\lift\cdot: (f\colon (a:A) \to B\,\bocks a) \to ((p:a\sim b) \to f\,a \simeq_{\sound\,p}f\,b) 
      \to ((q:Q) \to B\,q) \]
such that $\liftok\colon \lift f \,p\,\bocks a\equiv f a$.
 
\end{enumerate}
If we only have \ref{enum:Q} and \ref{enum:box}, we call the triple $(Q,\bocks\cdot, \sound)$ a \emph{prequotient}.
A quotient is \emph{exact} if additionally
we have 
\begin{enumerate}[i.]
\setcounter{enumi}{3}
\item $\exact :(a,b : A) \to  \bocks a \equiv \bocks b \to a \sim b$.

\end{enumerate}
\end{definition}

There are two special cases of the dependent lift property \ref{enum:dlift}. One is $B$ is not dependent,
 \[\nlift\cdot: (f\colon A \to B) \to (a\sim b \to f\,a \equiv f\,b) \to (Q \to B)\]
and the other is if $B$ is a predicate, i.e. $B : Q\to \prop$, in which case we get an induction principle:
\[\qind \colon((a\colon A)\to B \,\bocks a)\to ((q\colon Q)\to B\,q)\]
since the condition $((p:a\sim b) \to f\,a \simeq_{\sound\,p}f\,b) $  of the dependent lifting function $\lift\cdot$ is trivially satisfied.
On the other hand, these two special cases are sufficient to recover the dependent lifting function:      


\begin{proposition}\label{prop:nlifteq}
A prequotient with non-dependent lift~~$\nlift\cdot$ and an induction principle $\qind$ is a quotient.
\end{proposition}
This characterization was given as a definition of quotients in~\cite{hofmann:thesis}.


Quotients correspond to coequalizers. We remind the reader of the definition of coequalizer in a category. 

\begin{definition}
Given two morphisms $g,h : S\to A$, a \emph{coequalizer} of $g$ and $h$ is a morphism $\bocks\cdot:A\to Q$ such that for any $f:A\to X$ satisfying $f \circ g = f \circ h$, there exists a unique $\lift f$ such that  
\[\xymatrix{
S\ar@<0.5ex>[r]^g\ar@<-0.5ex>[r]_h& A\ar[r]^{\bocks\cdot}\ar[dr]_{f} & Q\ar@{-->}[d]^{\lift f}\\
&&X
}\]
A coequalizer is \emph{exact} if 
\[\xymatrix{
S\pullbackcorner\ar[r]^g\ar[d]_h & A\ar[d]^{\bocks\cdot} \\
A\ar[r]_{\bocks\cdot} & Q
}\]
and it is \emph{split} if the morphism $\bocks\cdot$ is a split epi, that is if it has a right inverse $\emb : Q \to A$.
\end{definition}

We observe that there is an exact correspondence between quotients and coequalizers:
\begin{proposition}\hfill
\begin{enumerate}
\item $Q$ is the quotient on $(S,\sim)$ where $s\sim s'$ if and only if $g\,s=h\,s'$.
This quotient is exact if{f} the coequalizer is exact.
\item Let $R$ be $\Sigma a,a':A,a\sim a'$ and $\pi_0,\pi_1 : R\to A$ the projection functions. The quotient for $(R,\sim)$ is then the coequalizer for those projections and it is exact if and only if the coequalizer is exact.
\[\xymatrix{
R\ar@<0.5ex>[r]^{\pi_0}\ar@<-0.5ex>[r]_{\pi_1}& A\ar[r]^{\bocks\cdot}\ar[dr]_{f} & Q\ar@{-->}[d]^{\lift f}\\
&&X
}\]
\end{enumerate}
\end{proposition}

\section{Definable quotients\\ }\label{sec:defquotients}

We now consider a general construction which allows us to construct quotients in type theory.

\begin{definition}\label{def:defquotients}
A \emph{definable quotient} on the setoid $(A,\sim)$ is a prequotient $(Q,\bocks\cdot,\sound)$ on $(A,\sim)$ with
\begin{align*}
\emb &: Q \to A\\
\compl &: (a : A) \to \emb {\bocks a} \sim a\\
\stable &: (q:Q) \to \bocks{\emb\,q} \equiv q\\
\end{align*}
\end{definition}

\begin{proposition}
All definable quotients are exact quotients.
\end{proposition}
\begin{proof}

Given $(f\colon A \to B)$ and $p : a\sim b \to f\,a \equiv f\,b$, define $\nlift f\, p \,q = f (\emb\,q)$ from which we get $\nlift f \,(p : a \sim b)\,\bocks a\equiv f(\emb\,\bocks a)\equiv f\,a$ because $\emb\,\bocks a\sim a$ by completeness and $f$ respects $\sim$ by $p$. 

To derive $\qind$, let $f:(a\colon A)\to B\,\bocks a$ and $q:Q$. Since $ \bocks{\emb\,q} \equiv q$ by stability, hence from $f (\emb\,q):B\,\bocks{\emb\,q}$ we can derive a proof of $B\,q$. 

It follows from Proposition~\ref{prop:nlifteq} that this defines a quotient. 

Finally, from $\bocks a \equiv \bocks b$ 
we obtain by completeness that $a\sim\emb(\bocks a)\equiv\emb(\bocks b)\sim b$ and hence $a\sim b$. That is, the quotient is exact.
\end{proof}
\subsection{Examples}\label{sec:dquotients:examples}
\subsubsection*{The integers}
Define $\Z =\nat + \nat $ and 

\begin{align*}
\bocks{(a,0)} &= \inl\,a\\
\bocks{(a+1,b+1)} &= \bocks{(a,b)}\\
\bocks{(0,b+1)} &= \inr\,b\\\\
\emb (\inl a) &= (a,0)\\
\emb (\inr b) &= (0,b+1)\\
\end{align*}
The fact that this gives rise to a definable quotient has been verified in Agda~\cite{nuo:report:2010}.
 One could of course just use that $\Z=\nat + \nat$ and define the operations on $\Z$ directly. However, seeing  $\Z$ as a quotient is helpful in proving properties of those operations and reflects the usual mathematical definition of the integers. E.g., to define $+$, we define 
\[(a,b){+_0}(a'b')= (a+a',b+b')\]
on $\Z_0$ and show that it respects $\sim$. Then by lifting $+_0$, we get $+$ on $\Z$, thus avoiding a rather incomprehensible case analysis. This becomes even more relevant when showing other properties such as distributivity of multiplication over addition~\cite{nuo:report:2010}.
\end{document}

