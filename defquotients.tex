% &latex
\documentclass[envcountsame]{llncs}

\usepackage{color}
\usepackage{amsmath}
\usepackage{amsfonts}
\usepackage{amssymb}
\usepackage{xypic}

% Editing and debugging
\hfuzz 0.1pt
\overfullrule=15pt
\brokenpenalty=10000
\newcommand{\todo}[1]{\textcolor{red}{TO~DO:~#1}}

\newtheorem{assumption}[theorem]{Assumption}

\newcommand{\N}{\mathbb{N}}
\newcommand{\Q}{\mathbb{Q}}
\newcommand{\R}{\mathbb{R}}
\newcommand{\Z}{\mathbb{Z}}

\newcommand{\dotph}{\,\cdot\,} % dot place holder as ub [.]

\providecommand{\abs}  [1]{\lvert#1\rvert}
\providecommand{\norm} [1]{\lVert#1\rVert}
\providecommand{\class}[1]{[#1]}
\providecommand{\set}  [1]{\left\{#1\right\}}
\providecommand{\dlift}[1]{\widehat{#1}}

\DeclareMathOperator{\Prop}{\mathbf{Prop}}
\DeclareMathOperator{\Set}{\mathbf{Set}}
\DeclareMathOperator{\Bool}{Bool}
\DeclareMathOperator{\id}{id}
\DeclareMathOperator{\sound}{sound}
\DeclareMathOperator{\qelimbeta}{qelim-\beta}
\DeclareMathOperator{\qind}{qind}
\DeclareMathOperator{\exact}{exact}
\DeclareMathOperator{\subst}{subst}
\DeclareMathOperator{\emb}{emb}
\DeclareMathOperator{\complete}{complete}
\DeclareMathOperator{\stable}{stable}
\DeclareMathOperator{\List}{List}
\DeclareMathOperator{\Fin}{Fin}
\DeclareMathOperator{\now}{now}
\DeclareMathOperator{\later}{later}
\DeclareMathOperator{\nowequal}{now_\sqsubseteq}
\DeclareMathOperator{\laterequal}{later_\sqsubseteq}
\DeclareMathOperator{\laterleft}{later_{left}}
\DeclareMathOperator{\inl}{inl}
\DeclareMathOperator{\inr}{inr}
\DeclareMathOperator{\qelim}{qelim}
\DeclareMathOperator{\lift}{lift}
\DeclareMathOperator{\LC}{LC}
\DeclareMathOperator{\liftbeta}{lift-\beta}

\newcommand{\eqqm}{\overset{\text{\tiny ?}}{=}}
\newcommand{\sep}{\mathrel{\sharp}}
\renewcommand{\equiv}{=}

\newcommand{\fad}{\text{for all definable }}

% For xy matrices
\newcommand{\pullbackcorner}[1][dr]{\save*!/#1-1.2pc/#1:(-1,1)@^{|-}\restore}
\newcommand{\pushoutcorner} [1][dr]{\save*!/#1+1.2pc/#1:(1,-1)@^{|-}\restore}

% FRONTMATTER
\title{Definable Quotients in Type Theory}
\author{Thorsten Altenkirch \inst{1}
   \and Thomas   Anberree   \inst{2}
   \and Nuo      Li         \inst{2}}
\institute{
School of Computer Science, University of Nottingham, Jubilee Campus, Wollaton Road, Nottingham, NG8 1BB, UK
\and
School of Computer Science, University of Nottingham, Ningbo Campus, 199 Taikang East Road, Ningbo, 315100, China}
% END FRONTMATTER

\begin{document}

\maketitle

\begin{abstract}
    The abstract should summarize the contents of the paper
    using at least 70 and at most 150 words. It will be set in 9-point
    font size and be inset 1.0 cm from the right and left margins.
    There will be two blank lines before and after the Abstract.
\end{abstract}

\section{Introduction}\label{sec:introduction}

\todo{Which type theory is used. Meaning of $=$. Meaning of $\Prop$. Meaning of  extensionality (eliminator (99 paper) or axiom?).}

\subsection{Notation}
We keep obvious quantifications implicit.

Given $B : A \to \Set$, $b : B\,a$, $b' : B\,a'$ and $p : a\equiv a'$, we write $b \simeq_{p} b'$ for $\subst\,B\,p\,b \equiv b'$.
We write $\Prop$ for the type of sets with at most one inhabitant.

\todo{define $\Fin$ and Bijection}

\todo{logical symbols in Prop}

\todo{Define subsets as sigma types}

\todo{We implicitly coerce elements of a subset with elements of the underlying set.}

\section{Setoids}\label{sec:setoids}
\begin{definition}
A setoid $(A,\sim)$ is a set $A$ equipped with an equivalence relation $\sim : A \to A \to \Prop$.
\end{definition}
\subsection{Examples}\label{sec:setoids:examples}
\subsubsection*{Integers}
The integers can be viewed as the setoids $(\Z_0=\N\times\N,\sim)$ where $(a,b)\sim(c,d)$ if{f} $a+d=c+b$ reflecting the idea that $(a,b)$ represents the integer $a-b$.
\subsubsection*{Rational numbers}
The rational numbers can in turn be defined as $(\Z\times\N,\sim)$ where $(x,m)\sim(y,n)$ if{f} $x\times(n+1)=y\times(m+1)$, reflecting that $(x,m)$ represents the quotient $\frac {x}{m+1}$.


\subsubsection*{The real numbers}

The real numbers can then be defined as $(\R_0,\sim)$ where $\R_0$ is the set of Cauchy sequences and two sequences are equivalent if{f} their pointwise difference converges to $0$.
\begin{align*}
\R_0&=\set{s : \N\to\Q \mid \forall\varepsilon :\Q,\varepsilon>0\to\exists m:\N, \forall i:\N, i>m\to |s\,i - s\, m|<\varepsilon}\\
r\sim s &= \forall\varepsilon :\Q,\varepsilon>0\to\exists m:\N, \forall i:\N, i>m\to |r\,i - s\,i|<\varepsilon
\end{align*}

\subsubsection*{Unordered pairs}
Given a set $A$, the unordered pairs of elements of $A$ is the setoid $(A\times A,\sim)$ where
$(a,b)\sim(b,a)$.

\subsubsection*{Finite multisets}
Given a set $A$ , the finite multisets of elements in $A$ is the setoid $(\List A,\sim)$ where two lists are equivalent if{f} one is the permutation of the other.
\begin{align*}
\List A &= \Sigma n:\N.\Fin\,n\to A\\
(m,f)\sim(n,g) &= \exists \varphi : \Fin\,m \to \Fin\,n \cdot\ \mathop{Bijection}\,\varphi \to g\circ\varphi = f
\end{align*}

\subsubsection*{Finite sets}
Given a set $A$ , the finite sets of elements in $A$ is the setoid $(\List A,\sim)$ where two lists are equivalent if{f} they contains the same elements.
\begin{align*}
(m,f)\subseteq(n,g) &= \exists \varphi : \Fin\,m \to \Fin\,n \cdot \varphi \to g\circ\varphi = f  \\
(m,f)\sim(n,g)&= (m,f)\subseteq(n,g) \wedge (n,g)\subseteq(m,f)
\end{align*}

\subsubsection*{Partiality monad}
Given a set $A$, the set of partial computations over $A$ is given by $(A_{\bot_0},{\sim})$ where $A_{\bot_0}$ is the set of delayed computations over $A$  and $\sim$ is a weak bisimilarity ignoring finite delays. Using the notation for mixed inductive coinductive  definitions from~\cite{danielsson2010mpc}, where we mark coinductive occurrences of a datatype by using $\infty$, we define $A_{\bot_0} : \Set$ and the relations $\sqsubseteq:A_{\bot_0}\to A_{\bot_0} \to \Prop$ by the following constructors:
\begin{align*}
\now  &: A \to A_{\bot_0}\\
\later &: \infty A_{\bot_0} \to  A_{\bot_0}\\
\nowequal &: \now\, a \sqsubseteq \now\,a'\\
\laterequal &: \infty(d \sqsubseteq d') \to \later\,d \sqsubseteq \later\,d'\\
\laterleft &: d\sqsubseteq d' \to \later\,d \sqsubseteq d'
\end{align*}
and we define $d\sim d'= d\sqsubseteq d' \wedge d'\sqsubseteq d$ .

\section{Quotients and coequalizers}\label{sec:quotients}

\begin{definition}
\label{def:quotient}
Given a setoid $(A,\sim)$,  a \emph{quotient} over that setoid is given by

\begin{enumerate}
\item \label{enum:Q} a set $Q$,
\item \label{enum:box}a function $\class\cdot\colon A \to Q$ which is compatible with $\sim$,
that is \[\sound\colon (a,b : A) \to a\sim b \to [a] = [b],\]
\item \label{enum:elim}   for any $B$ an eliminator : $Q\to\Set$
 \begin{align*}
 \qelim_B\colon &(f\colon (a:A) \to B\,\class a) \\
        {\to}\, &((p:a\sim b) \to f\,a \simeq_{\sound\,p}f\,b)\\
        {\to}\, &((q:Q) \to B\,q)
 \end{align*}
such that $\qelimbeta\colon \qelim_B f \,p\,\class a\equiv f a$.

\end{enumerate}
If we only have \ref{enum:Q} and \ref{enum:box}, we call the triple $(Q,\class\cdot, \sound)$ a \emph{prequotient}.
A quotient is \emph{exact} if additionally
we have
\begin{enumerate}
\setcounter{enumi}{3}
\item $\exact :(\forall a,b : A) \to  \class a \equiv \class b \to a \sim b$.

\end{enumerate}
\end{definition}

There are two special cases of the eliminator \ref{enum:elim}. One is $B$ is not dependent,
 \[\lift\colon (f\colon A \to B) \to (\forall a,b\cdot a\sim b \to f\,a \equiv f\,b) \to (Q \to B)\]
and the other is if $B$ is a predicate, i.e. $B : Q\to \Prop$, in which case we get an induction principle:
\[\qind \colon((a\colon A)\to B \,\class a)\to ((q\colon Q)\to B\,q)\]
since the condition $((p:a\sim b) \to f\,a \simeq_{\sound\,p}f\,b) $  of  the eliminator is trivially satisfied.
On the other hand, these two special cases are sufficient to recover the eliminator :


\begin{proposition}\label{prop:nlifteq}
A prequotient $(Q,\class\cdot,\sound)$ with

\begin{enumerate}
\item a non-dependent eliminator $$\lift_B\colon (f\colon A \to B) \to (\forall a,b\cdot a\sim b \to f\,a \equiv f\,b) \to (Q \to B)$$ for any $B\colon\Set$,
\item a $\beta$-law $$\liftbeta : \lift_B f \,p\,\class a\equiv f a,$$
\item an induction principle $$\qind_P\colon ((a\colon A)\to P \,\class a)\to ((q\colon Q)\to P\,q)$$
\end{enumerate}
gives rise to a quotient $(Q,\class\cdot,\sound,\qelim,\qelimbeta)$.
\end{proposition}
This is reminiscent of the fact that dependent elimination for the natural numbers can be constructed from non-dependent elimination and a induction principle.

The characterization in Proposition~\ref{prop:nlifteq} was given as a definition of quotients in~\cite{hofmann1995thesis}.


Quotients correspond to coequalizers. We remind the reader of the definition of coequalizers in a category.

\begin{definition}
Given two morphisms $g,h : S\to A$, a \emph{coequalizer} of $g$ and $h$ is a morphism $\class\cdot:A\to Q$ such that for any $f:A\to X$ satisfying $f \circ g = f \circ h$, there exists a unique $\dlift f$ such that
\[\xymatrix{
S\ar@<0.5ex>[r]^g\ar@<-0.5ex>[r]_h& A\ar[r]^{\class\cdot}\ar[dr]_{f} & Q\ar@{-->}[d]^{\dlift f}\\
&&X
}\]
A coequalizer is \emph{exact} if
\[\xymatrix{
S\pullbackcorner\ar[r]^g\ar[d]_h & A\ar[d]^{\class\cdot} \\
A\ar[r]_{\class\cdot} & Q
}\]
and it is \emph{split} if the morphism $\class\cdot$ is a split epi, that is if it has a right inverse $\emb : Q \to A$.
\end{definition}

We observe that there is an exact correspondence between quotients and coequalizers:
\begin{proposition}\hfill
\begin{enumerate}
\item $Q$ is the quotient on $(S,\sim)$ where $s\sim s'$ if and only if $g\,s=h\,s'$.
This quotient is exact if{f} the coequalizer is exact.
\item Let $R$ be $\Sigma a,a':A,a\sim a'$ and $\pi_0,\pi_1 : R\to A$ the projection functions. The quotient for $(R,\sim)$ is then the coequalizer for those projections and it is exact if and only if the coequalizer is exact.
\[\xymatrix{
R\ar@<0.5ex>[r]^{\pi_0}\ar@<-0.5ex>[r]_{\pi_1}& A\ar[r]^{\class\cdot}\ar[dr]_{f} & Q\ar@{-->}[d]^{\dlift f}\\
&&X
}\]
where $\dlift f=\lift f p$ and $p \colon \forall a,b\cdot a\sim b \to f\,a \equiv f\,b$ follows from $f \circ \pi_0 = f \circ \pi_1$.
\end{enumerate}
\end{proposition}

\section{Definable quotients}\label{sec:defquotients}

We now consider a general construction which allows us to construct quotients in type theory.

\begin{definition}\label{def:defquotients}
A \emph{definable quotient} is a prequotient $(Q, \class{\dotph}, \sound)$ on a setoid $(A,\sim)$ along with
\begin{align*}
\emb &: Q \to A\\
\complete &: (a : A) \to \emb {\class a} \sim a\\
\stable &: (q:Q) \to \class{\emb\,q} \equiv q\\
\end{align*}
\end{definition}

\begin{proposition}
All definable quotients are exact quotients.
\end{proposition}
\begin{proof}

Given $(f\colon A \to B)$ and $p : a\sim b \to f\,a \equiv f\,b$, define $\lift f\, p \,q = f (\emb\,q)$ from which we get $\lift f \,(p : a \sim b)\,\class a\equiv f(\emb\,\class a)\equiv f\,a$ because $\emb\,\class a\sim a$ by completeness and $f$ respects $\sim$ by $p$.

To derive $\qind$, let $f:(a\colon A)\to B\,\class a$ and $q:Q$. Since $ \class{\emb\,q} \equiv q$ by stability, hence from $f (\emb\,q):B\,\class{\emb\,q}$ we can derive a proof of $B\,q$.

It follows from Proposition~\ref{prop:nlifteq} that this defines a quotient.

Finally, from $\class a \equiv \class b$
we obtain by completeness that $a\sim\emb(\class a)\equiv\emb(\class b)\sim b$ and hence $a\sim b$. That is, the quotient is exact.
\end{proof}


\subsection{Examples}\label{sec:dquotients:examples}

\subsubsection*{The integers}
Define $\Z =\N + \N $ and

\begin{align*}
\class{(a,0)} &= \inl\,a\\
\class{(a+1,b+1)} &= \class{(a,b)}\\
\class{(0,b+1)} &= \inr\,b\\\\
\emb (\inl a) &= (a,0)\\
\emb (\inr b) &= (0,b+1)\\
\end{align*}
The fact that this gives rise to a definable quotient has been verified in Agda~\cite{nuo2010report}.
 One could of course just use that $\Z=\N + \N$ and define the operations on $\Z$ directly. However, seeing  $\Z$ as a quotient is helpful in proving properties of those operations and reflects the usual mathematical definition of the integers. E.g., to define $+$, we define
\[(a,b){+_0}(a', b')= (a+a',b+b')\]
on $\Z_0$ and show that it respects $\sim$. Then by lifting $+_0$, we get $+$ on $\Z$, thus avoiding a rather incomprehensible case analysis. This becomes even more relevant when showing other properties such as distributivity of multiplication over addition~\cite{nuo2010report}.

\subsubsection*{The rational numbers}

Define $\Q = \set{(x,m):\Z\times\N \,|\, \gcd\, x\,  (m+1) = 1}$ and
\begin{align*}
\class{(x,m)}&=\left(\frac{x}{d},\frac{m+1}{d}-1\right) \text{ where } d = \gcd\,x \,(m+1)\\
\emb \,(x,m) &= (x,m)
\end{align*}
Note that the greatest common divisor function ($\gcd$) is definable in type theory. Completeness comes from the fact that, for any common divisor $d$ of $x$ and $m+1$, it is provable that $\left(\frac x d,\frac {m+1} d-1\right)\sim\left(x,m\right)$ because $\frac x d \times (m+1) = x\times(\frac {m+1} d - 1+1)$.  Stability holds because whenever $d=\gcd\, x\, (m+1) = 1$, we have $\left(\frac{x}{d},\frac{m+1}{d}-1\right)=(x,m)$.


\subsubsection*{Unordered pairs}

The construction of the definable quotient depends on the choice of $A$. In general we
require a partial order $\leq : A \to A \to \Prop$ together with functions:
\begin{align*}
\min, \max : A \to A \to A
\end{align*}
calculating the binary minimum/maximum. We define $Q = \set{(a , b) \mid  a \leq b}$ and
\[ [(a,b)] = (\min a\, b, \max a \,b).\]
Clearly $[(a,b)] \sim (a,b)$ and if $a\leq b$ then $[(a,b)]=(a,b)$. Both facts follow from
 the properties of $\min$ and $\max$. We consider two cases:
\begin{description}
\item[$A = \N$] \hfill

We use the standard ordering $\leq : \N \to \N \to \Prop$ and exploit
that it is constructively total $\forall m ,n\cdot m \leq n + n \leq m$ to define $\min$ and
$\max$.
\item[$A=\N\to\N$] \hfill\\
We use the lexicographic ordering ${<},{\leq}:(\N \to \N) \to (\N\to\N)\to\Prop$
\begin{align*}
f < g & = \exists m:\N \cdot f m < g n \wedge \forall i<m\cdot f\,i = g \,i\\
f \leq g &= f < g \vee f=g
\end{align*}
While this order is not constructively total, in the sense that one cannot define a test to decide whether $f<g$, it is still possible to define $\min$ and $\max$.
For instance, the operator $\min : (\N \to \N) \to (\N\to\N) \to (\N \to \N)$ can be defined as :
\begin{align*}
 \min f\,g\,n =\,&\text{if $f \,n$ = $g\,n$ then $f n$}\\
                 &\text{else }\\
                 &\text{let}\,i = \min \{ j \leq n \mid f\,j \not= g\,j \}\\
                 &\text{in }\text{if}\, f\,i< g\,i\,\text{ then } f\,n \text{ else } g\,n
\end{align*}
\end{description}


\subsubsection*{Finite multisets}

\subsubsection*{Finite sets}


\section{Undefinable quotients}
However there are setoids specifications for which it is impossible to construct a definable quotient in type theory. Examples include the real numbers and the partiality monad described in Section~\ref{sec:setoids:examples}.

To show that this is the case, we show some properties of type theory in a classical metatheory.

We write $\vdash a\colon A$ if $a\colon A$ is derivable in type theory. In case that $\vdash P\colon \Prop$, we write $\vdash P$ to indicate that there is a proof $\vdash p\colon P$
\begin{definition}\hfill
\begin{enumerate}
\item Two elements $a$ and $b$ of a definable set are separable, written $a \sep b$, if there exists a test definable test $P\colon A\to \Bool$ such that $\vdash P\,a \neq P\,b$.
\item A definable set $A$ is \emph{discrete} whenever $\vdash a, b\colon A$ and   $\vdash a\not= b$
entails that $a$ and $b$ are separable.
\end{enumerate}
\end{definition}
\begin{proposition}\label{prop:NtoNdiscrete}
The set $\N\to\N$ is discrete.
\end{proposition}
\begin{proof}
Assume $\vdash f, g\colon \N \to \N$ and $\vdash f\neq g$. By soundness, $f$ and $g$ must denote different functions and hence there is a natural number $i$ such that $\vdash f\,i\neq g\,i$. Hence we can define $P\,h = {h\,i} \eqqm {f\,i}$ where ${\eqqm}\colon \N\to\N\to\Bool$ is a decision procedure for equality on $\N$.
\end{proof}

Note that we have used classical reasoning in the proof above. We don't think it is necessary, i.e. it should be possible to extract the witness $i$ from the proof that $f\neq g$.

\begin{proposition}\label{prop:splitepidiscrete}
Assume $ e\colon A\to B$ is a definable split~epi.  If $A$ is discrete then $B$ is discrete.
\end{proposition}
\begin{proof}
Let $\vdash s\colon B\to A$ such that $\vdash e\circ s=\id_B$. Let $\vdash b\neq b'\colon B$. Then $\vdash s\,b\neq s\,b'$, because $s$ is a right inverse of $e$, and hence there exists $\vdash P\colon A\to\Bool$ such that $\vdash P\,(s\,b)\neq P\,(s\,b')$, because $A$ is discrete, and hence $\vdash P'\colon B\to\Bool$ defined by $P' = P\circ s$ provably separates $b$ and $b'$.
Therefore $B$ is discrete.
\end{proof}

\begin{proposition}
 $\R_0$ is discrete.
\end{proposition}
\begin{proof}
Left to the reader as it is essentially the same as the proof for Proposition~\ref{prop:NtoNdiscrete}.
\end{proof}
To show that any set $\R$ which is a definable quotient of the setoid $(\R_0,\sim)$ given earlier in~\ref{sec:setoids:examples} is not discrete, we need

\todo{Define local continuity instead of assuming it and argue later that it is reasonable and if it holds then bla bla.}
\begin{assumption}
Local continuity  for $(\N \to \N) \to \N$ is admissible, i.e.
\begin{align*}
   &\fad \varphi : (\N \to \N) \to \N,\\
   &\fad f : \N \to \N,\\
   &\text{there exists }  n:\N\text{ such that }\\
   &\fad g : \N \to \N \text{ such that } (\forall i\leq n,\, \vdash f\,i = g\,i),\\
   &{\,\vdash \varphi\, f} = \varphi\, g.
\end{align*}
\end{assumption}

In the presence of local continuity we have
\begin{proposition}\label{prop:Rnotdiscrete} The set $\R$ is not discrete.
\end{proposition}
\begin{proof}
Assume that $\R$ is discrete. By exactness of the quotient defined in~\ref{sec:setoids:examples}, $\class{\vec 0} \neq \class{\vec 1}$ where $\vec 0$ and $\vec 1$ are the elements representing the constant $0$ or $1$ Cauchy sequences in $\R_0$. By discreteness of $\R$, there is a definable $P:\R\to\Bool$
such that $\vdash P\class{\vec 0}\neq P\class{\vec 1 }$. We define $P^\prime:\R_0\to\Bool$ by $P^\prime\,s = P\class{s}$, which has the property that $P^\prime\,\vec0\neq P^\prime\,\vec 1$ and that $P^\prime$ is closed under $\sim$. By local continuity, there is a number $n$  such that if $\forall i\leq n,\, \vdash f\,i = 0$ then $P^\prime \, f = P^\prime\,\vec 0$. Define $g\,i=\text{if } i\leq n \text{ then } 0 \text{ else } 1$, such that $P^\prime g =P^\prime \vec 0$ by local continuity. However $g \sim \vec 1$ and hence $P^\prime\,g=P^\prime \vec 1$, which contradicts $P^\prime \vec 1 \neq P^\prime \vec 0$.
\todo{Remove abuses of notation and refine local continuity to Cauchy sequences.}
\end{proof}

It seems that all sets definable in ordinary type theory~(using only the set formers $\Pi$, $\Sigma$, $=$, finite sets, $W$, see e.g.~\cite{nordstrom1990programming}) are discrete. This observation shows that the reals are not  definable as an exact quotient in ordinary type theory while Proposition~\ref{prop:Rnotdiscrete} shows that reals are not a definable  quotient in any extension of ordinary type theory, as long as local continuity is admissible.

\todo{See whether we can use the same counter for different classes of theorems.}

\begin{corollary}
$\R$ is not a definable quotient of $\R_0$.
\end{corollary}
\begin{proof}
Directly follows from Propositions~\ref{prop:Rnotdiscrete} and~\ref{prop:splitepidiscrete}.
\end{proof}

\bibliographystyle{plain}
\bibliography{biblio}
\end{document}














