% &latex
\documentclass[envcountsame]{llncs}
% Editing and debugging
\hfuzz 0.1pt
\overfullrule=15pt
\brokenpenalty=10000
\newcommand{\todo}[1]%{}%{\textcolor{red}{TO~DO:~#1}}


\usepackage{color}
\usepackage{amsmath}
\usepackage{amsfonts}
\usepackage{amssymb}
\usepackage{xypic}
\usepackage{Appendix/mylhstex}



% Used for Agda prettty printing with lhs2tex --agda (with the %include agda.fmt directive)
\RequirePackage[T1]{fontenc}
\RequirePackage[utf8x]{inputenc}
\RequirePackage{ucs}
\RequirePackage{amsfonts}
\providecommand\mathbbm{\mathbb}
\usepackage[utf8x]{inputenc}
\usepackage{ucs}
\usepackage{cite}


% ---------
% save and then undefine the conflicting command \vec
% we need \makeatletter because \@undefined uses the special @ character.
\newcommand{\mydummy}{}                % change if you need 
\makeatletter
\let\vec\mydummy                       % make \vec do same as \mydummy
\let\vec\@undefined                    % undefine \proof
\makeatother
\usepackage{MnSymbol}
% ---------

\DeclareUnicodeCharacter{"03BB}{\ensuremath{\lambda}}
\DeclareUnicodeCharacter{"03A3}{\ensuremath{\Sigma}}
\DeclareUnicodeCharacter{"03B2}{\ensuremath{\beta}}
\DeclareUnicodeCharacter{"03C8}{\ensuremath{\psi}}
\DeclareUnicodeCharacter{"231C}{\ensuremath{\ulcorner}}
\DeclareUnicodeCharacter{"231D}{\ensuremath{\urcorner}}
\DeclareUnicodeCharacter{"25B6}{\ensuremath{\filledmedtriangleright}}
%

% Proof trees (Paul Taylor's package)
\message{<Paul Taylor's Proof Trees, 2 August 1996>}
%% Build proof tree for Natural Deduction, Sequent Calculus, etc.
%% WITH SHORTENING OF PROOF RULES!
%% Paul Taylor, begun 10 Oct 1989
%% *** THIS IS ONLY A PRELIMINARY VERSION AND THINGS MAY CHANGE! ***
%%
%% 2 Aug 1996: fixed \mscount and \proofdotnumber
%%
%%      \prooftree
%%              hyp1            produces:
%%              hyp2
%%              hyp3            hyp1    hyp2    hyp3
%%      \justifies              -------------------- rulename
%%              concl                   concl
%%      \thickness=0.08em
%%      \shiftright 2em
%%      \using
%%              rulename
%%      \endprooftree
%%
%% where the hypotheses may be similar structures or just formulae.
%%
%% To get a vertical string of dots instead of the proof rule, do
%%
%%      \prooftree                      which produces:
%%              [hyp]
%%      \using                                  [hyp]
%%              name                              .
%%      \proofdotseparation=1.2ex                 .name
%%      \proofdotnumber=4                         .
%%      \leadsto                                  .
%%              concl                           concl
%%      \endprooftree
%%
%% Within a prooftree, \[ and \] may be used instead of \prooftree and
%% \endprooftree; this is not permitted at the outer level because it
%% conflicts with LaTeX. Also,
%%      \Justifies
%% produces a double line. In LaTeX you can use \begin{prooftree} and
%% \end{prootree} at the outer level (however this will not work for the inner
%% levels, but in any case why would you want to be so verbose?).
%%
%% All of of the keywords except \prooftree and \endprooftree are optional
%% and may appear in any order. They may also be combined in \newcommand's
%% eg "\def\Cut{\using\sf cut\thickness.08em\justifies}" with the abbreviation
%% "\prooftree hyp1 hyp2 \Cut \concl \endprooftree". This is recommended and
%% some standard abbreviations will be found at the end of this file.
%%
%% \thickness specifies the breadth of the rule in any units, although
%% font-relative units such as "ex" or "em" are preferable.
%% It may optionally be followed by "=".
%% \proofrulebreadth=.08em or \setlength\proofrulebreadth{.08em} may also be
%% used either in place of \thickness or globally; the default is 0.04em.
%% \proofdotseparation and \proofdotnumber control the size of the
%% string of dots
%%
%% If proof trees and formulae are mixed, some explicit spacing is needed,
%% but don't put anything to the left of the left-most (or the right of
%% the right-most) hypothesis, or put it in braces, because this will cause
%% the indentation to be lost.
%%
%% By default the conclusion is centered wrt the left-most and right-most
%% immediate hypotheses (not their proofs); \shiftright or \shiftleft moves
%% it relative to this position. (Not sure about this specification or how
%% it should affect spreading of proof tree.)
%
% global assignments to dimensions seem to have the effect of stretching
% diagrams horizontally.
%
%%==========================================================================

\def\introrule{{\cal I}}\def\elimrule{{\cal E}}%%
\def\andintro{\using{\land}\introrule\justifies}%%
\def\impelim{\using{\Rightarrow}\elimrule\justifies}%%
\def\allintro{\using{\forall}\introrule\justifies}%%
\def\allelim{\using{\forall}\elimrule\justifies}%%
\def\falseelim{\using{\bot}\elimrule\justifies}%%
\def\existsintro{\using{\exists}\introrule\justifies}%%

%% #1 is meant to be 1 or 2 for the first or second formula
\def\andelim#1{\using{\land}#1\elimrule\justifies}%%
\def\orintro#1{\using{\lor}#1\introrule\justifies}%%

%% #1 is meant to be a label corresponding to the discharged hypothesis/es
\def\impintro#1{\using{\Rightarrow}\introrule_{#1}\justifies}%%
\def\orelim#1{\using{\lor}\elimrule_{#1}\justifies}%%
\def\existselim#1{\using{\exists}\elimrule_{#1}\justifies}

%%==========================================================================

\newdimen\proofrulebreadth \proofrulebreadth=.05em
\newdimen\proofdotseparation \proofdotseparation=1.25ex
\newdimen\proofrulebaseline \proofrulebaseline=2ex
\newcount\proofdotnumber \proofdotnumber=3
\let\then\relax
\def\hfi{\hskip0pt plus.0001fil}
\mathchardef\squigto="3A3B
%
% flag where we are
\newif\ifinsideprooftree\insideprooftreefalse
\newif\ifonleftofproofrule\onleftofproofrulefalse
\newif\ifproofdots\proofdotsfalse
\newif\ifdoubleproof\doubleprooffalse
\let\wereinproofbit\relax
%
% dimensions and boxes of bits
\newdimen\shortenproofleft
\newdimen\shortenproofright
\newdimen\proofbelowshift
\newbox\proofabove
\newbox\proofbelow
\newbox\proofrulename
%
% miscellaneous commands for setting values
\def\shiftproofbelow{\let\next\relax\afterassignment\setshiftproofbelow\dimen0 }
\def\shiftproofbelowneg{\def\next{\multiply\dimen0 by-1 }%
\afterassignment\setshiftproofbelow\dimen0 }
\def\setshiftproofbelow{\next\proofbelowshift=\dimen0 }
\def\setproofrulebreadth{\proofrulebreadth}

%=============================================================================
\def\prooftree{% NESTED ZERO (\ifonleftofproofrule)
%
% first find out whether we're at the left-hand end of a proof rule
\ifnum  \lastpenalty=1
\then   \unpenalty
\else   \onleftofproofrulefalse
\fi
%
% some space on left (except if we're on left, and no infinity for outermost)
\ifonleftofproofrule
\else   \ifinsideprooftree
        \then   \hskip.5em plus1fil
        \fi
\fi
%
% begin our proof tree environment
\bgroup% NESTED ONE (\proofbelow, \proofrulename, \proofabove,
%               \shortenproofleft, \shortenproofright, \proofrulebreadth)
\setbox\proofbelow=\hbox{}\setbox\proofrulename=\hbox{}%
\let\justifies\proofover\let\leadsto\proofoverdots\let\Justifies\proofoverdbl
\let\using\proofusing\let\[\prooftree
\ifinsideprooftree\let\]\endprooftree\fi
\proofdotsfalse\doubleprooffalse
\let\thickness\setproofrulebreadth
\let\shiftright\shiftproofbelow \let\shift\shiftproofbelow
\let\shiftleft\shiftproofbelowneg
\let\ifwasinsideprooftree\ifinsideprooftree
\insideprooftreetrue
%
% now begin to set the top of the rule (definitions local to it)
\setbox\proofabove=\hbox\bgroup$\displaystyle % NESTED TWO
\let\wereinproofbit\prooftree
%
% these local variables will be copied out:
\shortenproofleft=0pt \shortenproofright=0pt \proofbelowshift=0pt
%
% flags to enable inner proof tree to detect if on left:
\onleftofproofruletrue\penalty1
}

%=============================================================================
% end whatever box and copy crucial values out of it
\def\eproofbit{% NESTED TWO
%
% various hacks applicable to hypothesis list 
\ifx    \wereinproofbit\prooftree
\then   \ifcase \lastpenalty
        \then   \shortenproofright=0pt  % 0: some other object, no indentation
        \or     \unpenalty\hfil         % 1: empty hypotheses, just glue
        \or     \unpenalty\unskip       % 2: just had a tree, remove glue
        \else   \shortenproofright=0pt  % eh?
        \fi
\fi
%
% pass out crucial values from scope
\global\dimen0=\shortenproofleft
\global\dimen1=\shortenproofright
\global\dimen2=\proofrulebreadth
\global\dimen3=\proofbelowshift
\global\dimen4=\proofdotseparation
\global\count255=\proofdotnumber
%
% end the box
$\egroup  % NESTED ONE
%
% restore the values
\shortenproofleft=\dimen0
\shortenproofright=\dimen1
\proofrulebreadth=\dimen2
\proofbelowshift=\dimen3
\proofdotseparation=\dimen4
\proofdotnumber=\count255
}

%=============================================================================
\def\proofover{% NESTED TWO
\eproofbit % NESTED ONE
\setbox\proofbelow=\hbox\bgroup % NESTED TWO
\let\wereinproofbit\proofover
$\displaystyle
}%
%
%=============================================================================
\def\proofoverdbl{% NESTED TWO
\eproofbit % NESTED ONE
\doubleprooftrue
\setbox\proofbelow=\hbox\bgroup % NESTED TWO
\let\wereinproofbit\proofoverdbl
$\displaystyle
}%
%
%=============================================================================
\def\proofoverdots{% NESTED TWO
\eproofbit % NESTED ONE
\proofdotstrue
\setbox\proofbelow=\hbox\bgroup % NESTED TWO
\let\wereinproofbit\proofoverdots
$\displaystyle
}%
%
%=============================================================================
\def\proofusing{% NESTED TWO
\eproofbit % NESTED ONE
\setbox\proofrulename=\hbox\bgroup % NESTED TWO
\let\wereinproofbit\proofusing
\kern0.3em$
}

%=============================================================================
\def\endprooftree{% NESTED TWO
\eproofbit % NESTED ONE
% \dimen0 =     length of proof rule
% \dimen1 =     indentation of conclusion wrt rule
% \dimen2 =     new \shortenproofleft, ie indentation of conclusion
% \dimen3 =     new \shortenproofright, ie
%                space on right of conclusion to end of tree
% \dimen4 =     space on right of conclusion below rule
  \dimen5 =0pt% spread of hypotheses
% \dimen6, \dimen7 = height & depth of rule
%
% length of rule needed by proof above
\dimen0=\wd\proofabove \advance\dimen0-\shortenproofleft
\advance\dimen0-\shortenproofright
%
% amount of spare space below
\dimen1=.5\dimen0 \advance\dimen1-.5\wd\proofbelow
\dimen4=\dimen1
\advance\dimen1\proofbelowshift \advance\dimen4-\proofbelowshift
%
% conclusion sticks out to left of immediate hypotheses
\ifdim  \dimen1<0pt
\then   \advance\shortenproofleft\dimen1
        \advance\dimen0-\dimen1
        \dimen1=0pt
%       now it sticks out to left of tree!
        \ifdim  \shortenproofleft<0pt
        \then   \setbox\proofabove=\hbox{%
                        \kern-\shortenproofleft\unhbox\proofabove}%
                \shortenproofleft=0pt
        \fi
\fi
%
% and to the right
\ifdim  \dimen4<0pt
\then   \advance\shortenproofright\dimen4
        \advance\dimen0-\dimen4
        \dimen4=0pt
\fi
%
% make sure enough space for label
\ifdim  \shortenproofright<\wd\proofrulename
\then   \shortenproofright=\wd\proofrulename
\fi
%
% calculate new indentations
\dimen2=\shortenproofleft \advance\dimen2 by\dimen1
\dimen3=\shortenproofright\advance\dimen3 by\dimen4
%
% make the rule or dots, with name attached
\ifproofdots
\then
        \dimen6=\shortenproofleft \advance\dimen6 .5\dimen0
        \setbox1=\vbox to\proofdotseparation{\vss\hbox{$\cdot$}\vss}%
        \setbox0=\hbox{%
                \advance\dimen6-.5\wd1
                \kern\dimen6
                $\vcenter to\proofdotnumber\proofdotseparation
                        {\leaders\box1\vfill}$%
                \unhbox\proofrulename}%
\else   \dimen6=\fontdimen22\the\textfont2 % height of maths axis
        \dimen7=\dimen6
        \advance\dimen6by.5\proofrulebreadth
        \advance\dimen7by-.5\proofrulebreadth
        \setbox0=\hbox{%
                \kern\shortenproofleft
                \ifdoubleproof
                \then   \hbox to\dimen0{%
                        $\mathsurround0pt\mathord=\mkern-6mu%
                        \cleaders\hbox{$\mkern-2mu=\mkern-2mu$}\hfill
                        \mkern-6mu\mathord=$}%
                \else   \vrule height\dimen6 depth-\dimen7 width\dimen0
                \fi
                \unhbox\proofrulename}%
        \ht0=\dimen6 \dp0=-\dimen7
\fi
%
% set up to centre outermost tree only
\let\doll\relax
\ifwasinsideprooftree
\then   \let\VBOX\vbox
\else   \ifmmode\else$\let\doll=$\fi
        \let\VBOX\vcenter
\fi
% this \vbox or \vcenter is the actual output:
\VBOX   {\baselineskip\proofrulebaseline \lineskip.2ex
        \expandafter\lineskiplimit\ifproofdots0ex\else-0.6ex\fi
        \hbox   spread\dimen5   {\hfi\unhbox\proofabove\hfi}%
        \hbox{\box0}%
        \hbox   {\kern\dimen2 \box\proofbelow}}\doll%
%
% pass new indentations out of scope
\global\dimen2=\dimen2
\global\dimen3=\dimen3
\egroup % NESTED ZERO
\ifonleftofproofrule
\then   \shortenproofleft=\dimen2
\fi
\shortenproofright=\dimen3
%
% some space on right and flag we've just made a tree
\onleftofproofrulefalse
\ifinsideprooftree
\then   \hskip.5em plus 1fil \penalty2
\fi
}

%==========================================================================
% IDEAS
% 1.    Specification of \shiftright and how to spread trees.
% 2.    Spacing command \m which causes 1em+1fil spacing, over-riding
%       exisiting space on sides of trees and not affecting the
%       detection of being on the left or right.
% 3.    Hack using \@currenvir to detect LaTeX environment; have to
%       use \aftergroup to pass \shortenproofleft/right out.
% 4.    (Pie in the sky) detect how much trees can be "tucked in"
% 5.    Discharged hypotheses (diagonal lines).
\newcommand{\ru}[2]{\vspace{1ex}
\begin{prooftree}
#1 \justifies #2
\end{prooftree}\vspace{1ex}}
\newcommand{\ax}[1]{
\ru{}{#1} }
\newcommand{\Ru}[3]{\vspace{1ex}
\begin{prooftree}
#1 \justifies #2 \using{\rm{#3}}
\end{prooftree}\vspace{1ex}}
\newcommand{\Ax}[2]{
\Ru{}{#1}{#2} }
%





\newtheorem{assumption}[theorem]{Assumption}

\newcommand{\N}{\mathbb{N}}
\newcommand{\Q}{\mathbb{Q}}
\newcommand{\R}{\mathbb{R}}
\newcommand{\Z}{\mathbb{Z}}

\newcommand{\dotph}{\,\cdot\,} % dot place holder as ub [.]
\newcommand{\dotop}{\mathrel{.}}
\providecommand{\abs}  [1]{\lvert#1\rvert}
\providecommand{\norm} [1]{\lVert#1\rVert}
\providecommand{\class}[1]{[#1]}
\providecommand{\set}  [1]{\left\{#1\right\}}
\providecommand{\dlift}[1]{\widehat{#1}}


\DeclareMathOperator{\Prop}{\mathbf{Prop}}
\DeclareMathOperator{\Set}{\mathbf{Set}}
\DeclareMathOperator{\Ext}{Ext}
\DeclareMathOperator{\Bool}{Bool}
\DeclareMathOperator{\id}{id}
\DeclareMathOperator{\sound}{sound}
\DeclareMathOperator{\qelimbeta}{qelim-\beta}
\DeclareMathOperator{\qind}{qind}
\DeclareMathOperator{\exact}{exact}
\DeclareMathOperator{\subst}{subst}
\DeclareMathOperator{\emb}{emb}
\DeclareMathOperator{\complete}{complete}
\DeclareMathOperator{\stable}{stable}
\DeclareMathOperator{\List}{List}
\DeclareMathOperator{\Fin}{Fin}
\DeclareMathOperator{\now}{now}
\DeclareMathOperator{\later}{later}
\DeclareMathOperator{\nowequal}{now_\sqsubseteq}
\DeclareMathOperator{\laterequal}{later_\sqsubseteq}
\DeclareMathOperator{\laterleft}{later_{left}}
\DeclareMathOperator{\inl}{inl}
\DeclareMathOperator{\inr}{inr}
\DeclareMathOperator{\qelim}{qelim}
\DeclareMathOperator{\lift}{lift}
\DeclareMathOperator{\LC}{LC}
\DeclareMathOperator{\liftbeta}{lift-\beta}
\DeclareMathOperator{\Bijection}{Bijection}
\DeclareMathOperator{\true}{true}
\DeclareMathOperator{\false}{false}
\DeclareMathOperator{\sort}{sort}
\DeclareMathOperator{\length}{length}
\DeclareMathOperator{\nub}{nub}
\newcommand{\eqqm}{\overset{\text{\tiny ?}}{=}}
\newcommand{\sep}{\mathrel{\sharp}}
\renewcommand{\equiv}{=}

\newcommand{\fad}{\text{for all definable }}

% For xy matrices
\newcommand{\pullbackcorner}[1][dr]{\save*!/#1-1.2pc/#1:(-1,1)@^{|-}\restore}
\newcommand{\pushoutcorner} [1][dr]{\save*!/#1+1.2pc/#1:(1,-1)@^{|-}\restore}

% FRONTMATTER
\title{Definable Quotients in Type Theory}
\author{Thorsten Altenkirch     \inst{1}
   \and Thomas   Anberr\'{e}e   \inst{2}
   \and Nuo      Li             \inst{2}}
\institute{
School of Computer Science, University of Nottingham, Jubilee Campus, Wollaton Road, Nottingham, NG8 1BB, UK
\and
School of Computer Science, University of Nottingham, Ningbo Campus, 199 Taikang East Road, Ningbo, 315100, China}
% END FRONTMATTER

\begin{document}

\maketitle

\begin{abstract}
  In Type Theory, a quotient set is a set representing a setoid.
  Categorically, this corresponds to the concept of an exact
  coequalizer. In the present paper we consider the case of a
  \emph{definable quotients}, where the quotient set arises as the
  codomain of a normalization function --- this corresponds to the
  notion of a split coequalizer.  We give a number of examples of
  definable quotients and notice that it is preferable to use the
  setoid structure when reasoning about the quotient set. We also show
  that there are examples where setoids cannot be represented in
  ordinary Type Theory such as the real numbers or the partiality
  monad under the assumption that local continuity is admissible in
  Type Theory.
\end{abstract}

\section{Introduction}\label{sec:introduction}

In Intensional Type Theory \cite{nordstrom1990programming}, quotient
types are unavailable and we use setoids \cite{barthe2003setoids}
instead. Setoids are just sets together with an equivalence
relation. However, the disadvantage of using setoids is that we have
now to lift any set-based operation on setoids. E.g. we need lists as
an operation on setoids and not just on sets. Moreover, setoids are
not safe in the sense that any consumer of a setoid may access the
underlying representation. One way out is to use a Type Theory which
supports genuine quotients such as the forthcoming Epigram 2 system (based on
\cite{alti:ott-conf}).  However, in many cases this is not
necessary because the quotient is actually definable. This is the
subject of the current paper.

An example is the case of integers. We can define integers as a
setoid, namely as the setoid given by pairs of natural numbers
$\Z_0=\N\times\N$, where the equivalence relation identifies pairs
representing the same difference, that is  $(a,b)\sim(c,d)$ if{f} $a+d=c+b$.
However, as it is well known, using a setoid here is unnecessary; we
can use a set, namely $\N+\N$ where the first injection represents the
positive numbers including $0$ whereas the second injection represents
the proper negative numbers. We can now define operations like
addition and multiplication and show algebraic properties,
such as verifying that the structure is a ring. However, this is
quite complicated and uses many unnecessary case
distinctions. E.g. try to prove distributivity within this setting!
It is easier to define the operations on the setoid and the required
algebraic properties are direct consequences of the semiring structure
of the natural numbers.

Hence we propose to use both the setoid
and the associated set, but to use the setoid structure to define
operations on the quotient set and to reason about it. In the present
paper we introduce the formal framework to do this, i.e. we give the definitions of  quotient as well as definable quotients and show the
equivalence of alternative definitions of quotients. We also verify that quotients
correspond precisely to the notion of coequalizers, and that an
additional condition, exactness, can equivalently be expressed in Type
Theory or in category theory. We present a number of examples for
definable quotients which are the base of a library of definable
quotients.

However, not all setoids can be represented as definable
quotients. Under the (reasonable) assumption of local continuity, we
show that the real numbers are not a definable quotient. Another
important example is the partiality monad. These counterexamples
suggest that it pays off to move to a type theory where all quotient
types exist --- i.e. the Type Theory corresponding to a Heyting
pretopos \cite{maietti1999effective}. In this context our work can be
seen as an exploration of the use of quotients within the settings of
Intensional Type Theory.

\subsection{Type Theory basics}
\label{sec:type-theory-basics}

We use standard type theoretic notation, inspired by Agda
\cite{norell-phd}. We write $(x : A) \to B$ for dependent function
types ($\Pi$-types) and $\Sigma x:A.B$ for dependent product types
($\Sigma$-types). We assume that strictly positive inductive and
coinductive types such as natural numbers $\N$, booleans $\Bool$,
disjoint union $A + B$ and lists $\List\,A$ are defined. We also use
the family of finite sets $\Fin : \N \to \Set$ with $\Fin\,n = \{ 0 ,
1, \dots, n-1\}$ which can be inductively generated from $0 :
\Fin\,(n+1)$ and $+1 : \Fin\,n \to \Fin\,(n+1) $. We write $\Set$ for
the universe of small sets.%  Indeed, all our definitions are universe
% polymorphic and could have taken place in any (larger) universe.  
We
write $\Prop$ for the subuniverse of propositions that are sets which
(extensionally) have at most one inhabitant (proof-irrelevance). We assume that $\Prop$
contains the equality type $a = b : \Prop$ for any $a,b : A : \Set$
\footnote{This is consistent with Voevodsky's univalence interpretation of type theory \cite{voevodsky}, if one identify sets with types whose h-level is 2.}  
and is closed under
universal ($\forall$) and existential ($\exists$)
quantification. While $\forall$ exactly corresponds to a $\Pi$-type,
$\exists$ is the squashing \cite{mendler1990quotient} of the
corresponding $\Sigma$-type.  $\Prop$ is also closed under conjunction
$\wedge$ which is interpreted as a $\Sigma$-type where both components
are propositional (and can be dependent) while $P \vee Q$ can be
defined using $\exists$ and $\Bool$.  Subset comprehension over a
predicate $P : A \to \Prop$  is interpreted as the corresponding $\Sigma$-type,
i.e. $\set{a:A\mid P\,a}=\Sigma a:A \dotop P$. Due to proof
irrelevance, the projection $\set{a:A\mid P\,a} \to A$ is an injection
and we will omit it if it is obvious from the context. We also omit
implicit arguments and to improve readability, we will even omit the
declaration of implicitly quantified arguments, assuming that the
human reader, unlike a machine, can reconstruct those. Given elements $b : B\,a$ and $b' : B\,a'$ with a proof $p : a = a'$,
we write $b \simeq_{p} b'$ for  the \emph{heterogeneous equality} $\subst\,B\,p\,b \equiv b'$. Most of
our examples do not require functional extensionality, but if we do we
assume that it is present in form of an uninterpreted constant $\Ext : (x : A) \to f\,x = g\,x \to f = g$,
which is justified by Hofmann's observation that extensional Type
Theory is a conservative extension of the theory considered in the present paper~\cite{hofmann1995thesis}. Alternatively we can eliminate $\Ext$ as suggested in
\cite{alti:lics99}. 
% which can then also be extended to quotient types.

%\todo{Bijection to be defined elsewhere}

\subsection{Related Work}
\label{sec:related-work}

Quotient types were introduced by Mendler in
\cite{mendler1990quotient} and subsequently investigated in Hofmann's
PhD dissertation~\cite{hofmann1995thesis}. An extensive investigation of setoids
can be found in~\cite{barthe2003setoids}. Maetti considers extensions
of both intensional and extensional Type Theory by quotient types
\cite{maietti1999effective}. Courtieu considers an extension of CIC
(an intensional type theory) by \emph{normalized types} corresponding
to our definable quotients~\cite{courtieulcs01}. Nogin describes a
modular implementation of quotient types in NuPRL (an extensional Type
Theory)~\cite{Nogin02quotienttypes}.

\subsection{Main results}
\label{sec:main-results}

We develop the notion of a definable quotient within an existing
intensional Type Theory instead of an extension by a new type former.
This enables us to formally verify a number of basic results in Agda
(see appendix), such as the relation between exact quotients,
coequalizers and definable quotients. We give a number of examples for
definable quotients, some which might seem surprising such as the presentation
of multisets over higher order types. Finally, we show that certain
quotient types are not definable quotients in our sense.

\section{Setoids}\label{sec:setoids}

We review the notion of a setoid and give a number of examples which
we are going to use subsequently.

\begin{definition}
A setoid $(A,\sim)$ is a set $A$ equipped with an equivalence relation ${\,\sim\,}\colon A \to A \to \Prop$.
\end{definition}
\todo{words}
\subsection{Examples}\label{sec:setoids:examples}
\subsubsection*{Integers}
The integers can be viewed as the setoid $(\Z_0=\N\times\N,\sim)$ where $(a,b)\sim(c,d)$ if{f} $a+d=c+b$ reflecting the idea that $(a,b)$ represents the integer $a-b$.
\subsubsection*{Rational numbers}
The rational numbers can in turn be defined as $(\Z\times\N,\sim)$ where $(x,m)\sim(y,n)$ if{f} $x\times(n+1)=y\times(m+1)$, reflecting that $(x,m)$ represents the quotient $\frac {x}{m+1}$.


\subsubsection*{The real numbers}

The real numbers can then be defined as $(\R_0,\sim)$ where $\R_0$ is the set of Cauchy sequences and two sequences are equivalent if{f} their pointwise difference converges to $0$.
\begin{align*}
\R_0&=\set{s : \N\to\Q \mid \forall\varepsilon :\Q,\varepsilon>0\to\exists m:\N, \forall i:\N, i>m\to |s\,i - s\, m|<\varepsilon}\\
r\sim s &= \forall\varepsilon :\Q,\varepsilon>0\to\exists m:\N, \forall i:\N, i>m\to |r\,i - s\,i|<\varepsilon
\end{align*}

\subsubsection*{Unordered pairs}
Given a set $A$, the unordered pairs of elements of $A$ is the setoid $(A\times A,\sim)$ where
$\sim$ is reflexive and $(a,b)\sim(b,a)$.

\subsubsection*{Finite multisets}
Given a set $A$ , the finite multisets of elements in $A$ is the setoid $(\List A,\sim)$ where two lists are equivalent if{f} one is the permutation of the other.
\begin{align*}
\List A &= \Sigma n:\N.\Fin\,n\to A\\
(m,f)\sim(n,g) &= \exists \varphi : \Fin\,m \to \Fin\,n \cdot\ \Bijection\,\varphi \land g\circ\varphi = f
\end{align*}
Notice that $(m,g)\sim(n,f)\implies m=n$ is provable in type theory, based on the definition of $\Bijection : (A\to B)\to \Prop$ which we omit here.

\subsubsection*{Finite sets}
Given a set $A$, the finite sets of elements in $A$ is the setoid $(\List A,{\sim})$ where two lists are equivalent if{f} they contain the same elements :\begin{align*}
(m,f)\subseteq(n,g) &= \exists \varphi : \Fin\,m \to \Fin\,n \cdot  g\circ\varphi = f  \\
(m,f)\sim(n,g)&= (m,f)\subseteq(n,g) \wedge (n,g)\subseteq(m,f).
\end{align*}
For example the lists $[1,2,1]$ and $[1,2]$ are equivalent and both represent the set $\set{1,2}$.
\subsubsection*{Partiality monad}
Given a set $A$, the set of partial computations over $A$ is given by
$(A_{\bot_0},{\sim})$ where $A_{\bot_0}$ is the set of delayed
computations over $A$  and $\sim$ is a weak bisimilarity ignoring
finite delays. We define $A_{\bot_0}$ as generated by the constructors
\begin{align*}
\now  &: A \to A_{\bot_0}\\
\later &: \infty A_{\bot_0} \to  A_{\bot_0}
\end{align*}
where $\infty$ indicates a coinductive premise --- categorically this is
the terminal coalgebra of $F\,X = A + X$. We inductively define the
relation $- \downarrow - : A_{\bot_0} \to A \to \Prop$ with the idea
that $d \downarrow a$ means that the computation $d$ terminates with
$a$, by the following rules:
\[ \ax{\now\,a \downarrow a}
\qquad
\ru{d \downarrow a}{\later\,d \downarrow a}
\]
We define the termination order ${\sqsubseteq} : A_{\bot_0}\to A_{\bot_0}
\to \Prop$ as $d \sqsubseteq d' = \forall a:A . d \downarrow a \to d'
\downarrow a$ and $d\sim d'= d\sqsubseteq d' \wedge d'\sqsubseteq d$ .
See \cite{capretta:2005}.

%  Using the notation for mixed inductive coinductive  definitions from~\cite{danielsson2010mpc}, where we mark coinductive occurrences of a datatype by using $\infty$, we define $A_{\bot_0} : \Set$ and the relations $\sqsubseteq:A_{\bot_0}\to A_{\bot_0} \to \Prop$ by the following constructors:
% \begin{align*}
% \now  &: A \to A_{\bot_0}\\
% \later &: \infty A_{\bot_0} \to  A_{\bot_0}\\
% \nowequal &: \now\, a \sqsubseteq \now\,a'\\
% \laterequal &: \infty(d \sqsubseteq d') \to \later\,d \sqsubseteq \later\,d'\\
% \laterleft &: d\sqsubseteq d' \to \later\,d \sqsubseteq d'
% \end{align*}
% and we define $d\sim d'= d\sqsubseteq d' \wedge d'\sqsubseteq d$ .

\section{Quotients and coequalizers}\label{sec:quotients}

We define what an (exact) quotient over a setoid is and relate this to
an alternative definition given by Hofmann and to the categorical
definition. All the concepts have been formalized in Agda (see Appendix~\ref{app:agda:quotients}).

\begin{definition}[prequotient, quotient, exact quotient]
\label{def:quotient}

\noindent
Given a setoid $(A,\sim)$,  a \emph{prequotient} $(Q,\class\dotph,\sound)$ over that setoid consists in
\begin{enumerate}
\item \label{enum:Q} a set $Q$,
\item \label{enum:box}a function $\class\dotph: A \to Q$,
\item \label{enum:sound} a proof $\sound$ that  the function $\class\dotph$ is compatible with the relation $\sim$,
that is \[\sound\colon (a,b : A) \to a\sim b \to [a] = [b],\]
\end{enumerate}
Such a prequotient is a \emph{quotient} if  we also have\begin{enumerate}
\setcounter{enumi}{3}
\item \label{enum:elim}
for any $B: Q\to\Set$, an eliminator
 \begin{align*}
 \qelim_B\,\,:\,\,\,&(f\colon (a:A) \to B\,\class a) \\
        {\to}\, &((p:a\sim b) \to f\,a \simeq_{\sound\,p}f\,b)\\
        {\to}\, &((q:Q) \to B\,q)
 \end{align*}
such that $\qelimbeta : \qelim_B\,f \,p\,\class a\equiv f\,a$.

\end{enumerate}
Finally, such a quotient is \emph{exact} if additionally
we have
a proof\begin{enumerate}
\setcounter{enumi}{4}
\item $\exact :(\forall a,b : A) \to  \class a \equiv \class b \to a \sim b$.

\end{enumerate}
\end{definition}

There are two special cases of the eliminator $\qelim_B$ described in item~\ref{enum:elim}. One is if $B$ is not dependent,
 \[\lift\colon (f\colon A \to B) \to (\forall a,b\cdot a\sim b \to f\,a \equiv f\,b) \to (Q \to B)\]
and the other is if $B$ is a predicate, i.e. $B : Q\to \Prop$, in which case we get an induction principle:
\[\qind \colon((a: A)\to B \,\class a)\to ((q : Q)\to B\,q)\]
since the condition $((p:a\sim b) \to f\,a \simeq_{\sound\,p}f\,b) $  of  the eliminator is trivially satisfied.
These two special cases are in fact sufficient to recover the eliminator, which is reminiscent of the fact that dependent elimination for the natural numbers can be constructed from non-dependent elimination and an induction principle.

\begin{proposition}\label{prop:nlifteq}
A prequotient $(Q,\class\dotph,\sound)$ with

\begin{enumerate}
\item a non-dependent eliminator $$\lift_B\colon (f\colon A \to B) \to (\forall a,b\cdot a\sim b \to f\,a \equiv f\,b) \to (Q \to B)$$ for any $B\colon\Set$,
\item a $\beta$-law $$\liftbeta  : \lift_B f \,p\,\class a\equiv f\,a,$$
\item an induction principle $$\qind_P\colon ((a\colon A)\to P \,\class a)\to ((q\colon Q)\to P\,q)$$
\end{enumerate}
gives rise to a quotient $(Q,\class\dotph,\sound,\qelim,\qelimbeta)$.
\end{proposition}
We refer to Appendix~\ref{app:agda:quotients} for a formal proof of Proposition~\ref{prop:nlifteq} and its converse. The characterization in Proposition~\ref{prop:nlifteq} was given as a definition of quotients in~\cite{hofmann1995thesis}.

\bigskip

Quotients correspond to coequalizers in category theory. Let us recall the definition.
\begin{definition}\label{def:coequalizer}
Given two morphisms $g,h : S\to A$, a \emph{coequalizer} of $g$ and $h$ is a morphism $\class\dotph:A\to Q$ such that for any $f:A\to X$ satisfying $f \circ g = f \circ h$, there exists a unique $\dlift f$ such that
\[\xymatrix{
S\ar@<0.5ex>[r]^g\ar@<-0.5ex>[r]_h& A\ar[r]^{\class\dotph}\ar[dr]_{f} & Q\ar@{-->}[d]^{\dlift f}\\
&&X
}\]
A coequalizer is \emph{exact} if
\[\xymatrix{
S\pullbackcorner\ar[r]^g\ar[d]_h & A\ar[d]^{\class\dotph} \\
A\ar[r]_{\class\dotph} & Q
}\]
and it is \emph{split} if the morphism $\class\dotph$ is a split epi, that is if it has a right inverse $\emb : Q \to A$.
\end{definition}

We observe that there is an exact correspondence between quotients and coequalizers:
\begin{proposition} In the context of Definition~\ref{def:coequalizer} above :
\begin{enumerate}
\item $Q$ is the quotient on $(S,\sim)$ where $s\sim s'$ if and only if $g\,s=h\,s'$.
This quotient is exact if{f} the coequalizer is exact.
\item Let $R$ be $\Sigma a,a':A,a\sim a'$ and $\pi_0,\pi_1 : R\to A$ the projection functions. The quotient for $(R,\sim)$ is then the coequalizer for those projections and it is exact if and only if the coequalizer is exact.
\[\xymatrix{
R\ar@<0.5ex>[r]^{\pi_0}\ar@<-0.5ex>[r]_{\pi_1}& A\ar[r]^{\class\dotph}\ar[dr]_{f} & Q\ar@{-->}[d]^{\dlift f}\\
&&X
}\]
where $\dlift f=\lift f p$ and $p \colon \forall a,b\cdot a\sim b \to f\,a \equiv f\,b$ follows from $f \circ \pi_0 = f \circ \pi_1$.
\end{enumerate}
\end{proposition}

\section{Definable quotients}\label{sec:defquotients}

We now consider a general construction which allows us to construct quotients in type theory.

\begin{definition}\label{def:defquotients}
A \emph{definable quotient} is a prequotient $(Q, \class{\dotph}, \sound)$ on a setoid $(A,\sim)$ along with
\begin{align*}
\emb &: Q \to A\\
\complete &: (a : A) \to \emb {\class a} \sim a\\
\stable &: (q:Q) \to \class{\emb\,q} \equiv q\\
\end{align*}
\end{definition}

This is exactly the specification of  $\class{-}$ as a
normalisation function with respect to $\emb$ (see~\cite{txa:jtait}).

\begin{proposition}\label{prop:definableimpliesexact}
All definable quotients are exact quotients.
\end{proposition}
\begin{proof}

Given $(f\colon A \to B)$ and $p : a\sim b \to f\,a \equiv f\,b$, define $\lift f\, p \,q = f (\emb\,q)$ from which we get $\lift f \,(p : a \sim b)\,\class a\equiv f(\emb\,\class a)\equiv f\,a$ because $\emb\,\class a\sim a$ by completeness and $f$ respects $\sim$ by $p$.

To derive $\qind$, let $f:(a\colon A)\to B\,\class a$ and $q:Q$. Since $ \class{\emb\,q} \equiv q$ by stability, hence from $f (\emb\,q):B\,\class{\emb\,q}$ we can derive a proof of $B\,q$.

It follows from Proposition~\ref{prop:nlifteq} that this defines a quotient.

Finally, from $\class a \equiv \class b$
we obtain by completeness that $a\sim\emb(\class a)\equiv\emb(\class b)\sim b$ and hence $a\sim b$. That is, the quotient is exact.
\end{proof}


\subsection{Examples}\label{sec:dquotients:examples}

We revisit the examples of setoids which turn out to correspond to
definable quotients.

\subsubsection*{The integers}
Define $\Z =\N + \N $ and

\begin{align*}
\class{(a,0)} &= \inl\,a\\
\class{(a+1,b+1)} &= \class{(a,b)}\\
\class{(0,b+1)} &= \inr\,b\\\\
\emb (\inl a) &= (a,0)\\
\emb (\inr b) &= (0,b+1)\\
\end{align*}
The fact that this gives rise to a definable quotient has been verified in Agda~\cite{nuo2010report}.
 One could of course just use that $\Z=\N + \N$ and define the operations on $\Z$ directly. However, seeing  $\Z$ as a quotient is helpful in proving properties of those operations and reflects the usual mathematical definition of the integers. E.g., to define $+$, we define
\[(a,b){+_0}(a', b')= (a+a',b+b')\]
on $\Z_0$ and show that it respects $\sim$. Then by lifting $+_0$, we get $+$ on $\Z$, thus avoiding a rather incomprehensible case analysis. This becomes even more relevant when showing other properties such as distributivity of multiplication over addition~\cite{nuo2010report}.

\subsubsection*{The rational numbers}

Define $\Q = \set{(x,m):\Z\times\N \,|\, \gcd\, x\,  (m+1) = 1}$ and
\begin{align*}
\class{(x,m)}&=\left(\frac{x}{d},\frac{m+1}{d}-1\right) \text{ where } d = \gcd\,x \,(m+1)\\
\emb \,(x,m) &= (x,m)
\end{align*}
Note that the greatest common divisor function ($\gcd$) is definable in type theory. Completeness comes from the fact that, for any common divisor $d$ of $x$ and $m+1$, it is provable that $\left(\frac x d,\frac {m+1} d-1\right)\sim\left(x,m\right)$ because $\frac x d \times (m+1) = x\times(\frac {m+1} d - 1+1)$.  Stability holds because whenever $d=\gcd\, x\, (m+1) = 1$, we have $\left(\frac{x}{d},\frac{m+1}{d}-1\right)=(x,m)$.


\subsubsection*{Unordered pairs}

The construction of a definable quotient over the setoids of unordered pairs $(A\times A,\sim)$ as defined in Section~~\ref{sec:setoids:examples} depends on the choice of $A$. In general we
require an order $\leq : A \to A \to \Prop$ together with functions:
\begin{align*}
\min, \max : A \to A \to A
\end{align*}
calculating the binary minimum and maximum for that order. This allows us to  define
\[
Q = \set{(a , b) \mid  a \leq b}
\]
and
\[ [(a,b)] = (\min a\, b, \max a \,b).\]
Soundness is obviously satisfied. An embedding of $Q$ into  $A\times A$ is simply the first projection --- for recall that an element in $Q$ is of the form $((a,b),p)$ where $p$ is a proof that $a\leq b$ (see Section~\ref{sec:introduction}) :
\begin{align*}
\emb &: Q \to A\times A\\
\emb &= \pi_0
\end{align*}
from which completeness and stability as stated in Definition~\ref{def:defquotients} clearly ensue : $[(a,b)] \sim (a,b)$ and if $a\leq b$ then $[(a,b)]=(a,b)$. Both facts follow from
 the properties of $\min$ and $\max$.
Thus $(Q,\class\dotph)$ gives rise to a definable quotient.

We consider three examples in which  $A$ is taken to be the set $\N$, $\N\to\N$ and $(\N\to\N)\to\N$ respectively :
\begin{description}
\item[$A = \N$] \hfill

We use the standard ordering $\leq : \N \to \N \to \Prop$ and exploit
that it is constructively total $\forall m ,n\cdot m \leq n \lor n \leq m$ to define $\min$ and
$\max$.

\bigskip
\item[$A=\N\to\N$] \hfill\\
We use the lexicographic ordering ${<},{\leq}:(\N \to \N) \to (\N\to\N)\to\Prop$
\begin{align*}
f < g & = \exists m:\N \cdot f m < g n \wedge \forall i<m\cdot f\,i = g \,i\\
f \leq g &= f < g \vee f=g
\end{align*}
While this order is not constructively total, in the sense that one cannot define a test to decide whether $f<g$, it is still possible to define $\min$ and $\max$.
For instance, the operator $\min : (\N \to \N) \to (\N\to\N) \to (\N \to \N)$ can be defined as :
\begin{align*}
 \min\,f\,g\,n =\,&\text{if $f \,n$ = $g\,n$ then $f n$}\\
                 &\text{else }\\
                 &\text{let}\,i = \min \{ j \leq n \mid f\,j \not= g\,j \}\\
                 &\text{in }\text{if}\, f\,i< g\,i\,\text{ then } f\,n \text{ else } g\,n
\end{align*}
Notice that both the definition of $i$ and the test $f\,i< g\,i$ do not depend on $n$ but only on $f$ and $g$.  Thus, in the case where $f$ and $g$ are different, $\min\,f\,g$ consistently returns the same function $f$ or $g$, whichever is the smallest in lexicographical order. In the case where the two functions $f$ and $g$ are equal, then the second branch of the top level if\dots then\dots else\dots is never chosen.

\bigskip
\item[$A=(\N\to\N)\to\N$] \hfill\\
The general idea to define the operator $\min$ is the same as in the case where $A=\N\to\N$. Let $\varphi : \N \to (\N\to\N)$ be an enumeration of natural sequences such that any finite sequence $[x_0,\dots,x_k]$ of
natural numbers is the prefix of $\varphi_i$ for some $i$  in the sense that $\varphi_i \,0 = x_0$, \dots, $\varphi_i\,k = x_k$ (see Appendix~\ref{app:haskell:phi} for a definition of $\varphi$).
We define :
\begin{align*}
 \min\,f\,g\,u =\,&\text{if $f \,u$ = $g\,u$ then $f u$}\\
                 &\text{else }\\
                 &\text{let}\,i = \min \{ i : \N\mid f\,\varphi_ i \not= g\,\varphi_ i \}\\
                 &\text{in }\text{if}\, f\,\varphi_ i< g\,\varphi_ i\,\text{ then } f\,u \text{ else } g\,u.
\end{align*}
Notice as previously that both the definition of $i$ and the test $f\,\varphi_i < g\,\varphi_ i$ do not depend on $u$ but only on $f$ and $g$. Under the assumption that local continuity holds~(see Definition~\ref{def:localcontinuity} below), we know that if $f\,u\not=g\,u$ then there must exist some $\varphi_i$, sharing a long enough prefix with $u$, such that $f\,\varphi_ i \not= g\,\varphi_ i$. However, if one works in a type theory where type checking is decidable, local continuity needs to be derivable and not just admissible for the system to accept the above definition. As an alternative, one may  postulate
\begin{align*}
&\text{local\_continuity}:\\
&\forall f,g: (\N\to\N)\to\N\\
&\to(\exists u:\N\to\N\,.\,f\,u\neq g\,u)\\
&\to (\exists n:\N \quad \forall v:\N\to\N\quad(\forall i\leq n\,.\,v_i= u_i\implies f\,v\neq g\,v))
\end{align*}

\end{description}
\todo{More details on this? Topic for future work: relation to axiom of choice and well orderedness.}

\subsubsection*{Finite multisets}

As in the case of unordered pairs, the construction of a definable quotient over the setoid of multisets $(\List A,\sim)$ defined in Section~\ref{sec:setoids:examples} depends on the choice of $A$.  We again require an order $A\to A \to \Prop$ to define the set of finite multisets of elements of $A$ as
\[
Q = \set{(m , s) : \List A\mid  \forall i,j:\Fin\,m \cdot i\leq j\implies s\,i \leq s\,j}
\]
and a sorting function $\sort: \List A \to \List A$ from which we define
\[
\class{(m,s)} = (m,\sort s).
\]
Notice that the function $\sort$ can be defined from the functions $\min$ and $\max : A \to A \to A$ used in the previous example about unordered pairs. However, we use a more direct method in our exploration of the case where $A$ is the set $\N\to\N$ of natural sequences. At first glance, it might seem counterintuitive that one can constructively sort sequences of infinite natural sequences and thus obtain a definable quotient of the setoids of multisets of natural sequences.
As with unordered pairs, the first projection defines an embedding from $Q$ to $\List A$ which clearly gives rise to a definable quotient.
\begin{description}
\item[$A=\N\to\N$] \hfill\\
First we define a family of preorders $\set{{\leq_k}}_{k:\N}$
on sequences of natural numbers by requesting that $u\leq_k v$
if and only if  the finite sequence $[u_0,\dots, u_k]$ comes before the finite sequence $[v_0,\dots, v_k]$ in the lexicographic order. Writing $u \leq_k v $ for $(\leq) \,k\,u\,v$ :


\begin{align*}
&{{-}\leq_{-}{-}} : \N\to (\N\to\N) \to (\N\to\N)\to \Bool\\
&{u\leq_k v} =  u_i\leq v_i \\
&\text{\phantom{$f\leq_n g =$} where }i =\min\set{i:\N\mid i>k\lor u_i\neq v_i}.
\end{align*}
Notice that if $u<_k v$ for some $k$ then $u<_l v$ for all $l$ greater than $k$.

Now, given a finite sequence of natural sequences $\varphi:\Fin m\to (\N\to\N)$, we can order it using any algorithm
\[\sort_{m,k} :(\Fin m \to (\N\to\N)) \to (\Fin m \to (\N\to\N))\]
which sorts $m$ sequences according to the preorder $\leq_k$.
We are then able to define :\begin{align*}
\class{(m,\varphi)} &= (m,\psi)\\
&\text{where $\psi\,i\,j = (\sort_{m,j}\,\varphi)\,i\,j$,}
\end{align*}
so that the finite sequence  $[\psi\,0,\dots,\psi\,(m-1)]$ thus defined is the finite sequence $[\varphi\,0,\dots,\varphi\,(m-1)]$ ordered in lexicographic order. The key point justifying that claim is that
\begin{equation}\label{eq:ms:order}
(\sort_{m, j}\,\varphi)\,i\,k=(\sort^\ast_{m}\,\varphi)\,i\,k
\end{equation}
for all $i:\Fin m$ and all $k\leq j$ where $\sort^\ast_m\,\varphi$ is the finite sequence whose elements are the functions $\varphi\,i:\N\to\N$ ordered in full lexicographical order --- we do not assume $\sort^\ast_m$ to be definable a priori although it is as a consequence of the definability of $\sort_{m,j}$.
We omit further details of the proof,  the intuition drawn from the case of unordered pairs above being more interesting.


\bigskip
\item[$A=(\N\to\N)\to\N$] \hfill\\
This example will be the subject of a separate paper as its formalization is quite involved.


\end{description}

\subsubsection*{Finite sets}\hfill\\

For types $A$ over which equality is decidable,  the set of finite subsets of $A$ can easily be defined as a quotient of the setoid of finite sets $(\List A, \sim)$ considered in Section~\ref{sec:setoids:examples}:
\begin{align*}
Q &= \set{(m , s) : \List A\mid  \forall i,j:\Fin\,m \cdot i\leq j\implies s\,i < s\,j}\\
\class{(m,as)} &=\nub(\sort\,as)
\end{align*}
where $\nub:\List\,A\to\List\,A$ takes advantage of the decidability of equality on $A$ to remove duplicates in a list.

However, there is no hope to define   $\class\dotph : \List\,A \to Q$ when equality on $A$ is not decidable, as  $a=b$ is equivalent to $\length\,[(a,b)] = 1$, which is decidable.





\section{Undefinable quotients}
However there are interesting setoid specifications for which it is impossible to construct a definable quotient in type theory. Examples include the real numbers and the partiality monad described in Section~\ref{sec:setoids:examples}.
To prove that these are indeed undefinable quotients, we first establish some properties of type theory in a classical metatheory.
We write $\vdash a : A$ if $a : A$ is derivable in the type theory under consideration. In case that $\vdash P : \Prop$, we simply  write $\vdash P$ to indicate that there is a proof $p$ of $P$ which is derivable, that is $\vdash p : P$.
\begin{definition}[separable elements, discrete sets]\hfill
\begin{enumerate}
\item Two elements $a$ and $b$ of a definable set are \emph{separable}, written $a \sep b$, if there exists a definable test $P\colon A\to \Bool$ such that $\vdash P\,a \neq P\,b$.
\item A definable set $A$ is \emph{discrete} whenever $\vdash a, b :A$ and   $\vdash a\not= b$
entails that $a$ and $b$ are separable.
\end{enumerate}
\end{definition}

\begin{proposition}\label{prop:NtoNdiscrete}
The set $\N\to\N$ is discrete.
\end{proposition}
\begin{proof}
Assume $\vdash f, g\colon \N \to \N$ and $\vdash f\neq g$. By soundness, $f$ and $g$ must denote different functions and hence there is a natural number $i$ such that $\vdash f\,i\neq g\,i$. Hence we can define $P\,h = {h\,i} \eqqm {f\,i}$ where ${\eqqm}\colon \N\to\N\to\Bool$ is a decision procedure for equality on $\N$.
\end{proof}

Note that we have used classical reasoning in the proof of Proposition~\ref{prop:NtoNdiscrete}. However, we do not think it is necessary because it should be possible to extract the witness $i$ from the proof that $f\neq g$.

\begin{proposition}\label{prop:splitepidiscrete}
Assume $ e\colon A\to B$ is a definable split~epi.  If $A$ is discrete then $B$ is discrete.
\end{proposition}
\begin{proof}
Let $\vdash s\colon B\to A$ such that $\vdash e \circ s=\id_B$ and let $\vdash b\neq b'\colon B$. Then $\vdash s\,b\neq s\,b'$ because  $s$ is a right inverse of $e$ :
\begin{align*}
&\vdash s\,b = s\,b'\to (e\circ s)\, b = (e\circ s)\,b'&\text{by congruence}  \\
&\vdash s\,b = s\,b'\to \id_B\, b = \id_B\,b'&e\circ s = \id_B  \\
&\vdash s\,b = s\,b'\to b = b'&\text{by definition of $\id_B$}\\
&\vdash s\,b = s\,b'\to \bot&\text{by modus ponens with $b=b'\to\bot.$}\\
\end{align*}Hence there exists $\vdash P\colon A\to\Bool$ such that $\vdash P\,(s\,b)\neq P\,(s\,b')$, because $A$ is discrete, and  $\vdash P'\colon B\to\Bool$ defined by $P' = P\circ s$ provably separates $b$ and $b'$.
Therefore $B$ is discrete.
\end{proof}

\begin{proposition}\label{prop:RZdiscrete}
 $\R_0$ is discrete.
\end{proposition}
\begin{proof}
Left to the reader as it is essentially the same as the proof for Proposition~\ref{prop:NtoNdiscrete}.
\end{proof}
To show that any set $\R$ which is a definable quotient of the setoid $(\R_0,\sim)$ given earlier in~\ref{sec:setoids:examples} is not discrete, we need


\begin{definition}[local continuity]\label{def:localcontinuity}
\emph{Local continuity} at type $(\N \to \N) \to \N$ is the property that
\begin{align*}
   &\fad\text{functions }\varphi : (\N \to \N) \to \N,\\
   &\fad\text{sequences }f : \N \to \N,\\
   &\text{there exists }  n:\N\text{ such that }\\
   &\fad\text{sequences } g : \N \to \N \text{ satisfying } (\forall i\leq n,\, \vdash f\,i = g\,i),\\
   &\text{we have that }{\,\vdash \varphi\, f} = \varphi\, g.
\end{align*}
\end{definition}
Local continuity expresses the fact that, to compute $\varphi\,f$, the reduction relation defining the operational semantics of type theory only inspects finitely many terms of the input sequence $f$. We have stated local continuity in its perhaps simplest form, at a particular type. However, we conjecture that it can be expressed and proved at all types. Whatever the case, it is easily shown that local continuity at type $(\N\to\N)\to\N$  entails local continuity at some other types, in particular at type $(\N\to\Q)\to\Bool$, which we next use to show that no set is a definable quotient of the setoid $(\R_0,\sim)$ described in Section~\ref{sec:setoids:examples}.

\begin{lemma}(local continuity for tests on rational sequences)
\label{lem:lcratseq}

In the presence of local continuity as in Definition~\ref{def:localcontinuity}, the following property holds :
\begin{align*}
   &\fad\text{functions }\varphi \colon (\N \to \Q) \to \Bool,\\
   &\fad\text{sequences }f  \colon  \N \to \Q,\\
   &\text{there exists }  n:\N\text{ such that }\\
   &\fad\text{sequences } g  \colon  \N \to \Q \text{ satisfying } (\forall i\leq n,\, \vdash f\,i = g\,i),\\
   &\text{we have that }{\,\vdash \varphi\, f} = \varphi\, g.
\end{align*}
\end{lemma}
\begin{proof}
Let $\eta\colon \N\to\Q$ be a definable bijection from $\N$ to $\Q$ and $\iota\colon\Bool\to\N$ a definable monomorphism, e.g. $\iota(\true)=0$, $\iota(\false)=1$. Let $\varphi\colon (\N \to \Q) \to \Bool$ and $f  \colon  \N \to \Q$ be as in the statement of Lemma~\ref{lem:lcratseq}.
Define $\varphi'\colon (\N\to\N)\to\N$ by $\varphi'\,f=\iota(\varphi(\eta\,f))$. By local continuity at type $(\N\to\N)\to\N$, there exists $n:\N$ such that for all definable sequences $g'\colon\N\to\N$ satisfying $(\forall i\leq n,\, \vdash (\eta^{-1}\circ f)\,i = g'\,i)$, we have that $\,\vdash\varphi'\,(\eta^{-1}\circ f) = \varphi'\,g'$. Now suppose that some definable function $g\colon(\N\to\Q)\to\Bool$ is
such that $(\forall i\leq n,\, \vdash f\,i = g\,i)$. Then we also have that $(\forall i\leq n,\, \vdash (\eta^{-1}\circ f)\,i = (\eta^{-1}\circ g)\,i)$ and hence that $\vdash\varphi'\,(\eta^{-1}\circ f) = \varphi'\,(\eta^{-1}\circ g)$,
that is $\vdash(\iota\circ \varphi)\,f = (\iota\circ \varphi)\,g$, by definition of $\varphi'$. Since $\iota$ is mono, we then have $\vdash \varphi\,f=\varphi\,g$, as expected.
\end{proof}


\begin{proposition}\label{prop:Rnotdiscrete} In the presence of local continuity, no set $R$ is a definable quotient of the setoid $(\R_0,\sim)$.
\end{proposition}
\begin{proof}
Suppose for the sake of contradiction that $(R,\class\dotph, \sound)$ is a definable quotient of the setoid $(\R_0,\sim)$. The function $\class\dotph\colon \R_0\to R$ is a split epi, as it has a right inverse $\emb$, and hence by propositions~\ref{prop:RZdiscrete} and~\ref{prop:splitepidiscrete}, the set $R$ is discrete. By exactness of the quotient,  we have that $\class{\vec 0} \neq \class{\vec 1}$ where $\vec 0$ and $\vec 1$ are  elements of $\R_0$ representing the Cauchy sequences $\lambda x.0$ and $\lambda x.1$, respectively. By discreteness of $R$, there exists a definable function $P:R\to\Bool$
such that $\vdash P\class{\vec 0}\neq P\class{\vec 1 }$. It follows that the  function $P^\prime:\R_0\to\Bool$ defined by $P^\prime\,s = P\class{s}$ has the property that $\vdash P^\prime\,\vec0\neq P^\prime\,\vec 1$ and that $P^\prime$ is closed under $\sim$. By local continuity at type $(\N\to\Q)\to\Bool$ (Lemma~\ref{lem:lcratseq}) and by proof irrelevance in the second component of the pairs in $\R_0$, there is a number $n_P$  such that, for all definable sequences $f\colon\N\to\Q$,
\[
\left(\forall i\leq n_P,\, \vdash f\,i = 0_\Q\right)\text{ entails } P^\prime \, f = P^\prime\,(\pi_0\,\vec0).
\]
Define $g\,i=\text{if } i\leq n \text{ then } 0_\Q \text{ else } 1_\Q$, such that $P^\prime g =P^\prime \vec 0$ by local continuity. However $g \sim \vec 1$ and hence $P^\prime\,g=P^\prime\,\vec 1$, which contradicts $P^\prime\,\vec 1 \neq P^\prime\,\vec 0$.

\end{proof}

Using very similar reasoning it can be shown that $\N_\bot$ is not definable either.

It seems that all sets definable in ordinary type theory~(using only the set formers $\Pi$, $\Sigma$, $=$, finite sets, $W$, see e.g.~\cite{nordstrom1990programming}) are discrete. This observation shows that the reals are not  definable as an exact quotient in ordinary type theory while Proposition~\ref{prop:Rnotdiscrete} shows that reals are not a definable  quotient in any extension of ordinary type theory, as long as local continuity is admissible.


\section{Conclusions}
\label{sec:conclusions}

The main result of the present work is that the notion of a definable
quotient in intensional type theory is useful and doesn't require any
extension of the theory. We hope that our formalisation of the notion
and the examples help to popularize this notion among people using
type theory. Some of the examples are maybe surprising, i.e. the
possibility to define unordered pairs and multisets for 1st order
function types, even though the order (and equality) of the elements
are undecidable. Assuming an internal proof of local continuity, this
can be even extended beyond 1st order. We also show that under the
assumption of local continuity the set of real numbers cannot be
defined by normalisation. This also extends to other examples such as
the partiality monad. These natural examples strongly suggest that
while the notion of a definable quotient is useful, we would also like
to be able to use quotient sets which do not fall in this category.
In the present work we have only covered the notion of a quotient by a
propositional family. It seems interesting, especially in the
context of higher dimensional type theory inspired by Voevodsky's proposal \cite{voevodsky}, to consider
non-propositional quotients, e.g. the quotient of a set by a
groupoid. An example for a definable quotient of this kind would be
the quotient of a non-canonical notion of finite sets by isomorphism.

\bibliographystyle{plain}
\bibliography{biblio,alti}

\appendix

\section{Formalization in Agda}\label{app:agda:quotients}

\def\textmu{}
%% ODER: format ==         = "\mathrel{==}"
%% ODER: format /=         = "\neq "
%
%
\makeatletter
\@ifundefined{lhs2tex.lhs2tex.sty.read}%
  {\@namedef{lhs2tex.lhs2tex.sty.read}{}%
   \newcommand\SkipToFmtEnd{}%
   \newcommand\EndFmtInput{}%
   \long\def\SkipToFmtEnd#1\EndFmtInput{}%
  }\SkipToFmtEnd

\newcommand\ReadOnlyOnce[1]{\@ifundefined{#1}{\@namedef{#1}{}}\SkipToFmtEnd}
\DeclareFontFamily{OT1}{cmtex}{}
\DeclareFontShape{OT1}{cmtex}{m}{n}
  {<5><6><7><8>cmtex8
   <9>cmtex9
   <10><10.95><12><14.4><17.28><20.74><24.88>cmtex10}{}
\DeclareFontShape{OT1}{cmtex}{m}{it}
  {<-> ssub * cmtt/m/it}{}
\newcommand{\texfamily}{\fontfamily{cmtex}\selectfont}
\DeclareFontShape{OT1}{cmtt}{bx}{n}
  {<5><6><7><8>cmtt8
   <9>cmbtt9
   <10><10.95><12><14.4><17.28><20.74><24.88>cmbtt10}{}
\DeclareFontShape{OT1}{cmtex}{bx}{n}
  {<-> ssub * cmtt/bx/n}{}
\newcommand{\tex}[1]{\text{\texfamily#1}}       % NEU

\newcommand{\Sp}{\hskip.33334em\relax}


\newcommand{\Conid}[1]{\mathit{#1}}
\newcommand{\Varid}[1]{\mathit{#1}}
\newcommand{\anonymous}{\kern0.06em \vbox{\hrule\@width.5em}}
\newcommand{\plus}{\mathbin{+\!\!\!+}}
\newcommand{\bind}{\mathbin{>\!\!\!>\mkern-6.7mu=}}
\newcommand{\rbind}{\mathbin{=\mkern-6.7mu<\!\!\!<}}% suggested by Neil Mitchell
\newcommand{\sequ}{\mathbin{>\!\!\!>}}
\renewcommand{\leq}{\leqslant}
\renewcommand{\geq}{\geqslant}
\usepackage{polytable}

%mathindent has to be defined
\@ifundefined{mathindent}%
  {\newdimen\mathindent\mathindent\leftmargini}%
  {}%

\def\resethooks{%
  \global\let\SaveRestoreHook\empty
  \global\let\ColumnHook\empty}
\newcommand*{\savecolumns}[1][default]%
  {\g@addto@macro\SaveRestoreHook{\savecolumns[#1]}}
\newcommand*{\restorecolumns}[1][default]%
  {\g@addto@macro\SaveRestoreHook{\restorecolumns[#1]}}
\newcommand*{\aligncolumn}[2]%
  {\g@addto@macro\ColumnHook{\column{#1}{#2}}}

\resethooks

\newcommand{\onelinecommentchars}{\quad-{}- }
\newcommand{\commentbeginchars}{\enskip\{-}
\newcommand{\commentendchars}{-\}\enskip}

\newcommand{\visiblecomments}{%
  \let\onelinecomment=\onelinecommentchars
  \let\commentbegin=\commentbeginchars
  \let\commentend=\commentendchars}

\newcommand{\invisiblecomments}{%
  \let\onelinecomment=\empty
  \let\commentbegin=\empty
  \let\commentend=\empty}

\visiblecomments

\newlength{\blanklineskip}
\setlength{\blanklineskip}{0.66084ex}

\newcommand{\hsindent}[1]{\quad}% default is fixed indentation
\let\hspre\empty
\let\hspost\empty
\newcommand{\NB}{\textbf{NB}}
\newcommand{\Todo}[1]{$\langle$\textbf{To do:}~#1$\rangle$}

\EndFmtInput
\makeatother
%
%
%
%
%
%
% This package provides two environments suitable to take the place
% of hscode, called "plainhscode" and "arrayhscode". 
%
% The plain environment surrounds each code block by vertical space,
% and it uses \abovedisplayskip and \belowdisplayskip to get spacing
% similar to formulas. Note that if these dimensions are changed,
% the spacing around displayed math formulas changes as well.
% All code is indented using \leftskip.
%
% Changed 19.08.2004 to reflect changes in colorcode. Should work with
% CodeGroup.sty.
%
\ReadOnlyOnce{polycode.fmt}%
\makeatletter

\newcommand{\hsnewpar}[1]%
  {{\parskip=0pt\parindent=0pt\par\vskip #1\noindent}}

% can be used, for instance, to redefine the code size, by setting the
% command to \small or something alike
\newcommand{\hscodestyle}{}

% The command \sethscode can be used to switch the code formatting
% behaviour by mapping the hscode environment in the subst directive
% to a new LaTeX environment.

\newcommand{\sethscode}[1]%
  {\expandafter\let\expandafter\hscode\csname #1\endcsname
   \expandafter\let\expandafter\endhscode\csname end#1\endcsname}

% "compatibility" mode restores the non-polycode.fmt layout.

\newenvironment{compathscode}%
  {\par\noindent
   \advance\leftskip\mathindent
   \hscodestyle
   \let\\=\@normalcr
   \let\hspre\(\let\hspost\)%
   \pboxed}%
  {\endpboxed\)%
   \par\noindent
   \ignorespacesafterend}

\newcommand{\compaths}{\sethscode{compathscode}}

% "plain" mode is the proposed default.
% It should now work with \centering.
% This required some changes. The old version
% is still available for reference as oldplainhscode.

\newenvironment{plainhscode}%
  {\hsnewpar\abovedisplayskip
   \advance\leftskip\mathindent
   \hscodestyle
   \let\hspre\(\let\hspost\)%
   \pboxed}%
  {\endpboxed%
   \hsnewpar\belowdisplayskip
   \ignorespacesafterend}

\newenvironment{oldplainhscode}%
  {\hsnewpar\abovedisplayskip
   \advance\leftskip\mathindent
   \hscodestyle
   \let\\=\@normalcr
   \(\pboxed}%
  {\endpboxed\)%
   \hsnewpar\belowdisplayskip
   \ignorespacesafterend}

% Here, we make plainhscode the default environment.

\newcommand{\plainhs}{\sethscode{plainhscode}}
\newcommand{\oldplainhs}{\sethscode{oldplainhscode}}
\plainhs

% The arrayhscode is like plain, but makes use of polytable's
% parray environment which disallows page breaks in code blocks.

\newenvironment{arrayhscode}%
  {\hsnewpar\abovedisplayskip
   \advance\leftskip\mathindent
   \hscodestyle
   \let\\=\@normalcr
   \(\parray}%
  {\endparray\)%
   \hsnewpar\belowdisplayskip
   \ignorespacesafterend}

\newcommand{\arrayhs}{\sethscode{arrayhscode}}

% The mathhscode environment also makes use of polytable's parray 
% environment. It is supposed to be used only inside math mode 
% (I used it to typeset the type rules in my thesis).

\newenvironment{mathhscode}%
  {\parray}{\endparray}

\newcommand{\mathhs}{\sethscode{mathhscode}}

% texths is similar to mathhs, but works in text mode.

\newenvironment{texthscode}%
  {\(\parray}{\endparray\)}

\newcommand{\texths}{\sethscode{texthscode}}

% The framed environment places code in a framed box.

\def\codeframewidth{\arrayrulewidth}
\RequirePackage{calc}

\newenvironment{framedhscode}%
  {\parskip=\abovedisplayskip\par\noindent
   \hscodestyle
   \arrayrulewidth=\codeframewidth
   \tabular{@{}|p{\linewidth-2\arraycolsep-2\arrayrulewidth-2pt}|@{}}%
   \hline\framedhslinecorrect\\{-1.5ex}%
   \let\endoflinesave=\\
   \let\\=\@normalcr
   \(\pboxed}%
  {\endpboxed\)%
   \framedhslinecorrect\endoflinesave{.5ex}\hline
   \endtabular
   \parskip=\belowdisplayskip\par\noindent
   \ignorespacesafterend}

\newcommand{\framedhslinecorrect}[2]%
  {#1[#2]}

\newcommand{\framedhs}{\sethscode{framedhscode}}

% The inlinehscode environment is an experimental environment
% that can be used to typeset displayed code inline.

\newenvironment{inlinehscode}%
  {\(\def\column##1##2{}%
   \let\>\undefined\let\<\undefined\let\\\undefined
   \newcommand\>[1][]{}\newcommand\<[1][]{}\newcommand\\[1][]{}%
   \def\fromto##1##2##3{##3}%
   \def\nextline{}}{\) }%

\newcommand{\inlinehs}{\sethscode{inlinehscode}}

% The joincode environment is a separate environment that
% can be used to surround and thereby connect multiple code
% blocks.

\newenvironment{joincode}%
  {\let\orighscode=\hscode
   \let\origendhscode=\endhscode
   \def\endhscode{\def\hscode{\endgroup\def\@currenvir{hscode}\\}\begingroup}
   %\let\SaveRestoreHook=\empty
   %\let\ColumnHook=\empty
   %\let\resethooks=\empty
   \orighscode\def\hscode{\endgroup\def\@currenvir{hscode}}}%
  {\origendhscode
   \global\let\hscode=\orighscode
   \global\let\endhscode=\origendhscode}%

\makeatother
\EndFmtInput
%
%
\ReadOnlyOnce{agda.fmt}%




\providecommand\mathbbm{\mathbb}

% TODO: Define more of these ...
\DeclareUnicodeCharacter{737}{\textsuperscript{l}}
\DeclareUnicodeCharacter{8718}{\ensuremath{\blacksquare}}
\DeclareUnicodeCharacter{8759}{::}
\DeclareUnicodeCharacter{9669}{\ensuremath{\triangleleft}}
\DeclareUnicodeCharacter{8799}{\ensuremath{\stackrel{\scriptscriptstyle ?}{=}}}
\DeclareUnicodeCharacter{10214}{\ensuremath{\llbracket}}
\DeclareUnicodeCharacter{10215}{\ensuremath{\rrbracket}}

% TODO: This is in general not a good idea.
\providecommand\textepsilon{$\epsilon$}
\providecommand\textmu{$\mu$}


%Actually, varsyms should not occur in Agda output.

% TODO: Make this configurable. IMHO, italics doesn't work well
% for Agda code.

\renewcommand\Varid[1]{\mathord{\textsf{#1}}}
\let\Conid\Varid
\newcommand\Keyword[1]{\textsf{\textbf{#1}}}
\EndFmtInput






\begin{hscode}\SaveRestoreHook
\column{B}{@{}>{\hspre}l<{\hspost}@{}}%
\column{3}{@{}>{\hspre}l<{\hspost}@{}}%
\column{E}{@{}>{\hspre}l<{\hspost}@{}}%
\>[B]{}\Keyword{module}\;\Conid{Quotient}\;\Keyword{where}{}\<[E]%
\\[\blanklineskip]%
\>[B]{}\Keyword{open}\;\Keyword{import}\;\Conid{Data.Product}{}\<[E]%
\\
\>[B]{}\Keyword{open}\;\Keyword{import}\;\Conid{Function}{}\<[E]%
\\[\blanklineskip]%
\>[B]{}\Keyword{open}\;\Keyword{import}\;\Conid{Relation.Binary.Core}{}\<[E]%
\\
\>[B]{}\Keyword{open}\;\Keyword{import}\;\Conid{Relation.Binary.PropositionalEquality}{}\<[E]%
\\
\>[B]{}\hsindent{3}{}\<[3]%
\>[3]{}\Keyword{hiding}\;(\Varid{isEquivalence}){}\<[E]%
\\[\blanklineskip]%
\>[B]{}\Keyword{open}\;\Keyword{import}\;\Conid{ThomasProperties}{}\<[E]%
\ColumnHook
\end{hscode}\resethooks

\emph{Definition of setoids}

\begin{hscode}\SaveRestoreHook
\column{B}{@{}>{\hspre}l<{\hspost}@{}}%
\column{3}{@{}>{\hspre}l<{\hspost}@{}}%
\column{5}{@{}>{\hspre}l<{\hspost}@{}}%
\column{19}{@{}>{\hspre}l<{\hspost}@{}}%
\column{E}{@{}>{\hspre}l<{\hspost}@{}}%
\>[B]{}\Keyword{record}\;\Conid{Setoid}\;\mathbin{:}\;\Conid{Set₁}\;\Keyword{where}{}\<[E]%
\\
\>[B]{}\hsindent{3}{}\<[3]%
\>[3]{}\Keyword{infix}\;\Varid{4}\;\Varid{\char95 ≈\char95 }{}\<[E]%
\\
\>[B]{}\hsindent{3}{}\<[3]%
\>[3]{}\Keyword{field}{}\<[E]%
\\
\>[3]{}\hsindent{2}{}\<[5]%
\>[5]{}\Conid{Carrier}\;{}\<[19]%
\>[19]{}\mathbin{:}\;\Conid{Set}{}\<[E]%
\\
\>[3]{}\hsindent{2}{}\<[5]%
\>[5]{}\Varid{\char95 ≈\char95 }\;{}\<[19]%
\>[19]{}\mathbin{:}\;\Conid{Carrier}\;\Varid{→}\;\Conid{Carrier}\;\Varid{→}\;\Conid{Set}{}\<[E]%
\\
\>[3]{}\hsindent{2}{}\<[5]%
\>[5]{}\Varid{isEquivalence}\;\mathbin{:}\;\Conid{IsEquivalence}\;\Varid{\char95 ≈\char95 }{}\<[E]%
\\
\>[B]{}\hsindent{3}{}\<[3]%
\>[3]{}\Keyword{open}\;\Conid{IsEquivalence}\;\Varid{isEquivalence}\;\Keyword{public}{}\<[E]%
\\[\blanklineskip]%
\>[B]{}\Keyword{open}\;\Conid{Setoid}\;\Keyword{renaming}{}\<[E]%
\\
\>[B]{}\hsindent{3}{}\<[3]%
\>[3]{}(\Varid{refl}\;\Varid{to}\;\Varid{reflexive};\Varid{sym}\;\Varid{to}\;\Varid{symmetric};\Varid{trans}\;\Varid{to}\;\Varid{transitive}){}\<[E]%
\ColumnHook
\end{hscode}\resethooks

\emph{Prequotients}

\begin{hscode}\SaveRestoreHook
\column{B}{@{}>{\hspre}l<{\hspost}@{}}%
\column{3}{@{}>{\hspre}l<{\hspost}@{}}%
\column{5}{@{}>{\hspre}l<{\hspost}@{}}%
\column{9}{@{}>{\hspre}l<{\hspost}@{}}%
\column{11}{@{}>{\hspre}l<{\hspost}@{}}%
\column{E}{@{}>{\hspre}l<{\hspost}@{}}%
\>[B]{}\Keyword{record}\;\Conid{PreQu}\;(\Conid{S}\;\mathbin{:}\;\Conid{Setoid})\;\mathbin{:}\;\Conid{Set₁}\;\Keyword{where}{}\<[E]%
\\
\>[B]{}\hsindent{3}{}\<[3]%
\>[3]{}\Varid{constructor}\;{}\<[E]%
\\
\>[3]{}\hsindent{2}{}\<[5]%
\>[5]{}\Conid{Q:\char95 }\;[\mskip1.5mu \mskip1.5mu]\;\Conid{:\char95 sound:\char95 }{}\<[E]%
\\
\>[B]{}\hsindent{3}{}\<[3]%
\>[3]{}\Keyword{private}{}\<[E]%
\\
\>[3]{}\hsindent{2}{}\<[5]%
\>[5]{}\Conid{A}\;{}\<[9]%
\>[9]{}\mathrel{=}\;\Conid{Carrier}\;\Conid{S}{}\<[E]%
\\
\>[3]{}\hsindent{2}{}\<[5]%
\>[5]{}\Varid{\char95 ∼\char95 }\;\mathrel{=}\;\Varid{\char95 ≈\char95 }\;\Conid{S}{}\<[E]%
\\
\>[B]{}\hsindent{3}{}\<[3]%
\>[3]{}\Keyword{field}{}\<[E]%
\\
\>[3]{}\hsindent{2}{}\<[5]%
\>[5]{}\Conid{Q}\;{}\<[11]%
\>[11]{}\mathbin{:}\;\Conid{Set}{}\<[E]%
\\
\>[3]{}\hsindent{2}{}\<[5]%
\>[5]{}[\mskip1.5mu \anonymous \mskip1.5mu]\;{}\<[11]%
\>[11]{}\mathbin{:}\;\Conid{A}\;\Varid{→}\;\Conid{Q}{}\<[E]%
\\
\>[3]{}\hsindent{2}{}\<[5]%
\>[5]{}\Varid{sound}\;\mathbin{:}\;\Varid{∀}\;\{\mskip1.5mu \Varid{a}\;\Varid{b}\;\mathbin{:}\;\Conid{A}\mskip1.5mu\}\;\Varid{→}\;\Varid{a}\;\Varid{∼}\;\Varid{b}\;\Varid{→}\;[\mskip1.5mu \Varid{a}\mskip1.5mu]\;\Varid{≡}\;[\mskip1.5mu \Varid{b}\mskip1.5mu]{}\<[E]%
\\[\blanklineskip]%
\>[B]{}\Keyword{open}\;\Conid{PreQu}\;\Keyword{renaming}{}\<[E]%
\\
\>[B]{}\hsindent{3}{}\<[3]%
\>[3]{}(\Conid{Q}\;\Varid{to}\;\Conid{Q'};[\mskip1.5mu \anonymous \mskip1.5mu]\;\Varid{to}\;\Varid{nf};\Varid{sound}\;\Varid{to}\;\Varid{sound'}){}\<[E]%
\ColumnHook
\end{hscode}\resethooks

\emph{Quotients as prequotients with a dependent eliminator.}

\begin{hscode}\SaveRestoreHook
\column{B}{@{}>{\hspre}l<{\hspost}@{}}%
\column{3}{@{}>{\hspre}l<{\hspost}@{}}%
\column{5}{@{}>{\hspre}l<{\hspost}@{}}%
\column{11}{@{}>{\hspre}l<{\hspost}@{}}%
\column{13}{@{}>{\hspre}l<{\hspost}@{}}%
\column{17}{@{}>{\hspre}l<{\hspost}@{}}%
\column{50}{@{}>{\hspre}l<{\hspost}@{}}%
\column{E}{@{}>{\hspre}l<{\hspost}@{}}%
\>[B]{}\Keyword{record}\;\Conid{Qu}\;\{\mskip1.5mu \Conid{S}\;\mathbin{:}\;\Conid{Setoid}\mskip1.5mu\}\;(\Conid{PQ}\;\mathbin{:}\;\Conid{PreQu}\;\Conid{S})\;\mathbin{:}\;\Conid{Set₁}\;\Keyword{where}{}\<[E]%
\\
\>[B]{}\hsindent{3}{}\<[3]%
\>[3]{}\Varid{constructor}\;{}\<[E]%
\\
\>[3]{}\hsindent{2}{}\<[5]%
\>[5]{}\Varid{qelim:\char95 qelim-β:\char95 }{}\<[E]%
\\
\>[B]{}\hsindent{3}{}\<[3]%
\>[3]{}\Keyword{private}{}\<[E]%
\\
\>[3]{}\hsindent{2}{}\<[5]%
\>[5]{}\Conid{A}\;{}\<[11]%
\>[11]{}\mathrel{=}\;\Conid{Carrier}\;\Conid{S}{}\<[E]%
\\
\>[3]{}\hsindent{2}{}\<[5]%
\>[5]{}\Varid{\char95 ∼\char95 }\;{}\<[11]%
\>[11]{}\mathrel{=}\;\Varid{\char95 ≈\char95 }\;\Conid{S}{}\<[E]%
\\
\>[3]{}\hsindent{2}{}\<[5]%
\>[5]{}\Conid{Q}\;{}\<[11]%
\>[11]{}\mathrel{=}\;\Conid{Q'}\;\Conid{PQ}{}\<[E]%
\\
\>[3]{}\hsindent{2}{}\<[5]%
\>[5]{}[\mskip1.5mu \anonymous \mskip1.5mu]\;{}\<[11]%
\>[11]{}\mathrel{=}\;\Varid{nf}\;\Conid{PQ}{}\<[E]%
\\
\>[3]{}\hsindent{2}{}\<[5]%
\>[5]{}\Varid{sound}\;\mathbin{:}\;\Varid{∀}\;\{\mskip1.5mu \Varid{a}\;\Varid{b}\;\mathbin{:}\;\Conid{A}\mskip1.5mu\}\;\Varid{→}\;(\Varid{a}\;\Varid{∼}\;\Varid{b})\;\Varid{→}\;[\mskip1.5mu \Varid{a}\mskip1.5mu]\;\Varid{≡}\;[\mskip1.5mu \Varid{b}\mskip1.5mu]{}\<[E]%
\\
\>[3]{}\hsindent{2}{}\<[5]%
\>[5]{}\Varid{sound}\;\mathrel{=}\;\Varid{sound'}\;\Conid{PQ}{}\<[E]%
\\
\>[B]{}\hsindent{3}{}\<[3]%
\>[3]{}\Keyword{field}{}\<[E]%
\\
\>[3]{}\hsindent{2}{}\<[5]%
\>[5]{}\Varid{qelim}\;{}\<[13]%
\>[13]{}\mathbin{:}\;\{\mskip1.5mu \Conid{B}\;\mathbin{:}\;\Conid{Q}\;\Varid{→}\;\Conid{Set}\mskip1.5mu\}\;{}\<[E]%
\\
\>[13]{}\Varid{→}\;(\Varid{f}\;\mathbin{:}\;(\Varid{a}\;\mathbin{:}\;\Conid{A})\;\Varid{→}\;\Conid{B}\;[\mskip1.5mu \Varid{a}\mskip1.5mu])\;{}\<[E]%
\\
\>[13]{}\Varid{→}\;((\Varid{a}\;\Varid{b}\;\mathbin{:}\;\Conid{A})\;\Varid{→}\;(\Varid{p}\;\mathbin{:}\;\Varid{a}\;\Varid{∼}\;\Varid{b})\;{}\<[E]%
\\
\>[13]{}\hsindent{4}{}\<[17]%
\>[17]{}\Varid{→}\;\Varid{subst}\;\Conid{B}\;(\Varid{sound}\;\Varid{p})\;(\Varid{f}\;\Varid{a})\;\Varid{≡}\;\Varid{f}\;\Varid{b})\;{}\<[E]%
\\
\>[13]{}\Varid{→}\;(\Varid{q}\;\mathbin{:}\;\Conid{Q})\;\Varid{→}\;\Conid{B}\;\Varid{q}{}\<[E]%
\\
\>[3]{}\hsindent{2}{}\<[5]%
\>[5]{}\Varid{qelim-β}\;\mathbin{:}\;\Varid{∀}\;\{\mskip1.5mu \Conid{B}\;\Varid{a}\;\Varid{f}\mskip1.5mu\}\;\Varid{q}\;\Varid{→}\;\Varid{qelim}\;\{\mskip1.5mu \Conid{B}\mskip1.5mu\}\;\Varid{f}\;\Varid{q}\;[\mskip1.5mu \Varid{a}\mskip1.5mu]\;{}\<[50]%
\>[50]{}\Varid{≡}\;\Varid{f}\;\Varid{a}{}\<[E]%
\\
\>[B]{}\Keyword{open}\;\Conid{Qu}{}\<[E]%
\ColumnHook
\end{hscode}\resethooks

\emph{Proof irrelevance of qelim}
\begin{hscode}\SaveRestoreHook
\column{B}{@{}>{\hspre}l<{\hspost}@{}}%
\column{10}{@{}>{\hspre}l<{\hspost}@{}}%
\column{12}{@{}>{\hspre}l<{\hspost}@{}}%
\column{33}{@{}>{\hspre}l<{\hspost}@{}}%
\column{E}{@{}>{\hspre}l<{\hspost}@{}}%
\>[B]{}\Varid{qelimIrr}\;\mathbin{:}\;\{\mskip1.5mu \Conid{S}\;\mathbin{:}\;\Conid{Setoid}\mskip1.5mu\}\;\{\mskip1.5mu \Conid{PQ}\;\mathbin{:}\;\Conid{PreQu}\;\Conid{S}\mskip1.5mu\}\;(\Varid{x}\;\mathbin{:}\;\Conid{Qu}\;\Conid{PQ})\;{}\<[E]%
\\
\>[B]{}\hsindent{10}{}\<[10]%
\>[10]{}\Varid{→}\;\Varid{∀}\;\{\mskip1.5mu \Conid{B}\;\Varid{a}\;\Varid{f}\;\Varid{q}\;\Varid{q'}\mskip1.5mu\}\;{}\<[E]%
\\
\>[B]{}\hsindent{10}{}\<[10]%
\>[10]{}\Varid{→}\;\Varid{qelim}\;\Varid{x}\;\{\mskip1.5mu \Conid{B}\mskip1.5mu\}\;\Varid{f}\;\Varid{q}\;(\Varid{nf}\;\Conid{PQ}\;\Varid{a})\;{}\<[E]%
\\
\>[10]{}\hsindent{2}{}\<[12]%
\>[12]{}\Varid{≡}\;\Varid{qelim}\;\Varid{x}\;\{\mskip1.5mu \Conid{B}\mskip1.5mu\}\;\Varid{f}\;\Varid{q'}\;(\Varid{nf}\;\Conid{PQ}\;\Varid{a}){}\<[E]%
\\
\>[B]{}\Varid{qelimIrr}\;\Varid{x}\;\{\mskip1.5mu \Conid{B}\mskip1.5mu\}\;\{\mskip1.5mu \Varid{a}\mskip1.5mu\}\;\{\mskip1.5mu \Varid{f}\mskip1.5mu\}\;\{\mskip1.5mu \Varid{q}\mskip1.5mu\}\;\{\mskip1.5mu \Varid{q'}\mskip1.5mu\}\;\mathrel{=}\;(\Varid{qelim-β}\;\Varid{x}\;\{\mskip1.5mu \Conid{B}\mskip1.5mu\}\;\{\mskip1.5mu \Varid{a}\mskip1.5mu\}\;\{\mskip1.5mu \Varid{f}\mskip1.5mu\}\;\Varid{q})\;{}\<[E]%
\\
\>[B]{}\hsindent{33}{}\<[33]%
\>[33]{}\Varid{▶}\;\Varid{⟨}\;\Varid{qelim-β}\;\Varid{x}\;\{\mskip1.5mu \Conid{B}\mskip1.5mu\}\;\{\mskip1.5mu \Varid{a}\mskip1.5mu\}\;\{\mskip1.5mu \Varid{f}\mskip1.5mu\}\;\Varid{q'}\;\Varid{⟩}{}\<[E]%
\ColumnHook
\end{hscode}\resethooks

\emph{Exact quotients}
\begin{hscode}\SaveRestoreHook
\column{B}{@{}>{\hspre}l<{\hspost}@{}}%
\column{3}{@{}>{\hspre}l<{\hspost}@{}}%
\column{5}{@{}>{\hspre}l<{\hspost}@{}}%
\column{10}{@{}>{\hspre}l<{\hspost}@{}}%
\column{11}{@{}>{\hspre}l<{\hspost}@{}}%
\column{E}{@{}>{\hspre}l<{\hspost}@{}}%
\>[B]{}\Keyword{record}\;\Conid{QuE}\;\{\mskip1.5mu \Conid{S}\;\mathbin{:}\;\Conid{Setoid}\mskip1.5mu\}\;\{\mskip1.5mu \Conid{PQ}\;\mathbin{:}\;\Conid{PreQu}\;\Conid{S}\mskip1.5mu\}\;(\Conid{QU}\;\mathbin{:}\;\Conid{Qu}\;\Conid{PQ})\;\mathbin{:}\;\Conid{Set₁}\;\Keyword{where}{}\<[E]%
\\
\>[B]{}\hsindent{3}{}\<[3]%
\>[3]{}\Varid{constructor}\;{}\<[E]%
\\
\>[3]{}\hsindent{2}{}\<[5]%
\>[5]{}\Varid{exact:\char95 }{}\<[E]%
\\
\>[B]{}\hsindent{3}{}\<[3]%
\>[3]{}\Keyword{private}{}\<[E]%
\\
\>[3]{}\hsindent{2}{}\<[5]%
\>[5]{}\Conid{A}\;{}\<[11]%
\>[11]{}\mathrel{=}\;\Conid{Carrier}\;\Conid{S}{}\<[E]%
\\
\>[3]{}\hsindent{2}{}\<[5]%
\>[5]{}\Varid{\char95 ∼\char95 }\;{}\<[10]%
\>[10]{}\mathrel{=}\;\Varid{\char95 ≈\char95 }\;\Conid{S}{}\<[E]%
\\
\>[3]{}\hsindent{2}{}\<[5]%
\>[5]{}[\mskip1.5mu \anonymous \mskip1.5mu]\;{}\<[10]%
\>[10]{}\mathrel{=}\;\Varid{nf}\;\Conid{PQ}{}\<[E]%
\\
\>[B]{}\hsindent{3}{}\<[3]%
\>[3]{}\Keyword{field}{}\<[E]%
\\
\>[3]{}\hsindent{2}{}\<[5]%
\>[5]{}\Varid{exact}\;\mathbin{:}\;\Varid{∀}\;\{\mskip1.5mu \Varid{a}\;\Varid{b}\;\mathbin{:}\;\Conid{A}\mskip1.5mu\}\;\Varid{→}\;[\mskip1.5mu \Varid{a}\mskip1.5mu]\;\Varid{≡}\;[\mskip1.5mu \Varid{b}\mskip1.5mu]\;\Varid{→}\;\Varid{a}\;\Varid{∼}\;\Varid{b}{}\<[E]%
\\
\>[B]{}\Keyword{open}\;\Conid{QuE}{}\<[E]%
\ColumnHook
\end{hscode}\resethooks

\emph{Quotients as prequotients with a non-dependent eliminator (lift).}

\emph{(As in Hofmann's PhD dissertation.)}

\begin{hscode}\SaveRestoreHook
\column{B}{@{}>{\hspre}l<{\hspost}@{}}%
\column{3}{@{}>{\hspre}l<{\hspost}@{}}%
\column{5}{@{}>{\hspre}l<{\hspost}@{}}%
\column{11}{@{}>{\hspre}l<{\hspost}@{}}%
\column{12}{@{}>{\hspre}l<{\hspost}@{}}%
\column{42}{@{}>{\hspre}l<{\hspost}@{}}%
\column{45}{@{}>{\hspre}l<{\hspost}@{}}%
\column{48}{@{}>{\hspre}l<{\hspost}@{}}%
\column{E}{@{}>{\hspre}l<{\hspost}@{}}%
\>[B]{}\Keyword{record}\;\Conid{QuH}\;\{\mskip1.5mu \Conid{S}\;\mathbin{:}\;\Conid{Setoid}\mskip1.5mu\}\;(\Conid{PQ}\;\mathbin{:}\;\Conid{PreQu}\;\Conid{S})\;\mathbin{:}\;\Conid{Set₁}\;\Keyword{where}{}\<[E]%
\\
\>[B]{}\hsindent{3}{}\<[3]%
\>[3]{}\Varid{constructor}\;{}\<[E]%
\\
\>[3]{}\hsindent{2}{}\<[5]%
\>[5]{}\Varid{lift:\char95 lift-β:\char95 qind:\char95 }{}\<[E]%
\\
\>[B]{}\hsindent{3}{}\<[3]%
\>[3]{}\Keyword{private}{}\<[E]%
\\
\>[3]{}\hsindent{2}{}\<[5]%
\>[5]{}\Conid{A}\;{}\<[12]%
\>[12]{}\mathrel{=}\;\Conid{Carrier}\;\Conid{S}{}\<[E]%
\\
\>[3]{}\hsindent{2}{}\<[5]%
\>[5]{}\Varid{\char95 ∼\char95 }\;{}\<[12]%
\>[12]{}\mathrel{=}\;\Varid{\char95 ≈\char95 }\;\Conid{S}{}\<[E]%
\\
\>[3]{}\hsindent{2}{}\<[5]%
\>[5]{}\Conid{Q}\;{}\<[12]%
\>[12]{}\mathrel{=}\;\Conid{Q'}\;\Conid{PQ}{}\<[E]%
\\
\>[3]{}\hsindent{2}{}\<[5]%
\>[5]{}[\mskip1.5mu \anonymous \mskip1.5mu]\;{}\<[12]%
\>[12]{}\mathrel{=}\;\Varid{nf}\;\Conid{PQ}{}\<[E]%
\\
\>[B]{}\hsindent{3}{}\<[3]%
\>[3]{}\Keyword{field}{}\<[E]%
\\
\>[3]{}\hsindent{2}{}\<[5]%
\>[5]{}\Varid{lift}\;{}\<[12]%
\>[12]{}\mathbin{:}\;\{\mskip1.5mu \Conid{B}\;\mathbin{:}\;\Conid{Set}\mskip1.5mu\}\;{}\<[E]%
\\
\>[12]{}\Varid{→}\;(\Varid{f}\;\mathbin{:}\;\Conid{A}\;\Varid{→}\;\Conid{B})\;{}\<[E]%
\\
\>[12]{}\Varid{→}\;((\Varid{a}\;\Varid{b}\;\mathbin{:}\;\Conid{A})\;\Varid{→}\;(\Varid{a}\;\Varid{∼}\;\Varid{b})\;\Varid{→}\;\Varid{f}\;\Varid{a}\;{}\<[42]%
\>[42]{}\Varid{≡}\;{}\<[45]%
\>[45]{}\Varid{f}\;\Varid{b})\;{}\<[E]%
\\
\>[12]{}\Varid{→}\;\Conid{Q}\;\Varid{→}\;\Conid{B}{}\<[E]%
\\
\>[3]{}\hsindent{2}{}\<[5]%
\>[5]{}\Varid{lift-β}\;\mathbin{:}\;\Varid{∀}\;\{\mskip1.5mu \Conid{B}\;\Varid{a}\;\Varid{f}\;\Varid{q}\mskip1.5mu\}\;\Varid{→}\;\Varid{lift}\;\{\mskip1.5mu \Conid{B}\mskip1.5mu\}\;\Varid{f}\;\Varid{q}\;[\mskip1.5mu \Varid{a}\mskip1.5mu]\;{}\<[48]%
\>[48]{}\Varid{≡}\;\Varid{f}\;\Varid{a}{}\<[E]%
\\
\>[3]{}\hsindent{2}{}\<[5]%
\>[5]{}\Varid{qind}\;{}\<[11]%
\>[11]{}\mathbin{:}\;(\Conid{P}\;\mathbin{:}\;\Conid{Q}\;\Varid{→}\;\Conid{Set})\;{}\<[E]%
\\
\>[11]{}\hsindent{1}{}\<[12]%
\>[12]{}\Varid{→}\;(\Varid{∀}\;\Varid{x}\;\Varid{→}\;(\Varid{p}\;\Varid{p'}\;\mathbin{:}\;\Conid{P}\;\Varid{x})\;\Varid{→}\;\Varid{p}\;\Varid{≡}\;\Varid{p'})\;{}\<[E]%
\\
\>[11]{}\hsindent{1}{}\<[12]%
\>[12]{}\Varid{→}\;(\Varid{∀}\;\Varid{a}\;\Varid{→}\;\Conid{P}\;[\mskip1.5mu \Varid{a}\mskip1.5mu])\;{}\<[E]%
\\
\>[11]{}\hsindent{1}{}\<[12]%
\>[12]{}\Varid{→}\;(\Varid{∀}\;\Varid{x}\;\Varid{→}\;\Conid{P}\;\Varid{x}){}\<[E]%
\\[\blanklineskip]%
\>[B]{}\Keyword{open}\;\Conid{QuH}\;\Keyword{renaming}\;(\Varid{lift}\;\Varid{to}\;\Varid{lift'};\Varid{lift-β}\;\Varid{to}\;\Varid{lift-β'}){}\<[E]%
\ColumnHook
\end{hscode}\resethooks

\emph{Definable quotients}
\begin{hscode}\SaveRestoreHook
\column{B}{@{}>{\hspre}l<{\hspost}@{}}%
\column{3}{@{}>{\hspre}l<{\hspost}@{}}%
\column{5}{@{}>{\hspre}l<{\hspost}@{}}%
\column{11}{@{}>{\hspre}l<{\hspost}@{}}%
\column{14}{@{}>{\hspre}l<{\hspost}@{}}%
\column{E}{@{}>{\hspre}l<{\hspost}@{}}%
\>[B]{}\Keyword{record}\;\Conid{QuD}\;\{\mskip1.5mu \Conid{S}\;\mathbin{:}\;\Conid{Setoid}\mskip1.5mu\}\;(\Conid{PQ}\;\mathbin{:}\;\Conid{PreQu}\;\Conid{S})\;\mathbin{:}\;\Conid{Set₁}\;\Keyword{where}{}\<[E]%
\\
\>[B]{}\hsindent{3}{}\<[3]%
\>[3]{}\Varid{constructor}\;{}\<[E]%
\\
\>[3]{}\hsindent{2}{}\<[5]%
\>[5]{}\Varid{emb:\char95 complete:\char95 stable:\char95 }{}\<[E]%
\\
\>[B]{}\hsindent{3}{}\<[3]%
\>[3]{}\Keyword{private}{}\<[E]%
\\
\>[3]{}\hsindent{2}{}\<[5]%
\>[5]{}\Conid{A}\;{}\<[11]%
\>[11]{}\mathrel{=}\;\Conid{Carrier}\;\Conid{S}{}\<[E]%
\\
\>[3]{}\hsindent{2}{}\<[5]%
\>[5]{}\Varid{\char95 ∼\char95 }\;{}\<[11]%
\>[11]{}\mathrel{=}\;\Varid{\char95 ≈\char95 }\;\Conid{S}{}\<[E]%
\\
\>[3]{}\hsindent{2}{}\<[5]%
\>[5]{}\Conid{Q}\;{}\<[11]%
\>[11]{}\mathrel{=}\;\Conid{Q'}\;\Conid{PQ}{}\<[E]%
\\
\>[3]{}\hsindent{2}{}\<[5]%
\>[5]{}[\mskip1.5mu \anonymous \mskip1.5mu]\;{}\<[11]%
\>[11]{}\mathrel{=}\;\Varid{nf}\;\Conid{PQ}{}\<[E]%
\\
\>[B]{}\hsindent{3}{}\<[3]%
\>[3]{}\Keyword{field}{}\<[E]%
\\
\>[3]{}\hsindent{2}{}\<[5]%
\>[5]{}\Varid{emb}\;{}\<[14]%
\>[14]{}\mathbin{:}\;\Conid{Q}\;\Varid{→}\;\Conid{A}{}\<[E]%
\\
\>[3]{}\hsindent{2}{}\<[5]%
\>[5]{}\Varid{complete}\;\mathbin{:}\;\Varid{∀}\;\Varid{a}\;\Varid{→}\;\Varid{emb}\;[\mskip1.5mu \Varid{a}\mskip1.5mu]\;\Varid{∼}\;\Varid{a}{}\<[E]%
\\
\>[3]{}\hsindent{2}{}\<[5]%
\>[5]{}\Varid{stable}\;{}\<[14]%
\>[14]{}\mathbin{:}\;\Varid{∀}\;\Varid{q}\;\Varid{→}\;[\mskip1.5mu \Varid{emb}\;\Varid{q}\mskip1.5mu]\;\Varid{≡}\;\Varid{q}{}\<[E]%
\\
\>[B]{}\Keyword{open}\;\Conid{QuD}{}\<[E]%
\ColumnHook
\end{hscode}\resethooks

\emph{Relations between types of quotients:}

Below, we show the following, where the arrow → means "gives rise to" :

\ensuremath{\Conid{QuH}\;\Varid{→}\;\Conid{Qu}} (Proposition 3 in the paper)

\ensuremath{\Conid{Qu}\;\Varid{→}\;\Conid{QuH}} (Reverse of Proposition 3)

\ensuremath{\Conid{QuD}\;\Varid{→}\;\Conid{QuE}} (A definable quotient is always exact)

\ensuremath{\Conid{QuD}\;\Varid{→}\;\Conid{Qu}}

\ensuremath{\Conid{QuD}\;\Varid{→}\;\Conid{QuH}} (Also a consequence of \ensuremath{\Conid{QuD}\;\Varid{→}\;\Conid{Qu}} and \ensuremath{\Conid{Qu}\;\Varid{→}\;\Conid{QuH}})

\begin{hscode}\SaveRestoreHook
\column{B}{@{}>{\hspre}l<{\hspost}@{}}%
\column{3}{@{}>{\hspre}l<{\hspost}@{}}%
\column{5}{@{}>{\hspre}l<{\hspost}@{}}%
\column{7}{@{}>{\hspre}l<{\hspost}@{}}%
\column{8}{@{}>{\hspre}l<{\hspost}@{}}%
\column{9}{@{}>{\hspre}l<{\hspost}@{}}%
\column{10}{@{}>{\hspre}l<{\hspost}@{}}%
\column{12}{@{}>{\hspre}l<{\hspost}@{}}%
\column{13}{@{}>{\hspre}l<{\hspost}@{}}%
\column{15}{@{}>{\hspre}l<{\hspost}@{}}%
\column{23}{@{}>{\hspre}l<{\hspost}@{}}%
\column{28}{@{}>{\hspre}l<{\hspost}@{}}%
\column{31}{@{}>{\hspre}l<{\hspost}@{}}%
\column{E}{@{}>{\hspre}l<{\hspost}@{}}%
\>[B]{}\Conid{QuH→Qu}\;\mathbin{:}\;\{\mskip1.5mu \Conid{S}\;\mathbin{:}\;\Conid{Setoid}\mskip1.5mu\}\;\Varid{→}\;\{\mskip1.5mu \Conid{PQ}\;\mathbin{:}\;\Conid{PreQu}\;\Conid{S}\mskip1.5mu\}\;{}\<[E]%
\\
\>[B]{}\hsindent{8}{}\<[8]%
\>[8]{}\Varid{→}\;(\Conid{QuH}\;\Conid{PQ})\;\Varid{→}\;(\Conid{Qu}\;\Conid{PQ}){}\<[E]%
\\
\>[B]{}\Conid{QuH→Qu}\;\{\mskip1.5mu \Conid{S}\mskip1.5mu\}\;\{\mskip1.5mu \Conid{Q:}\;\Conid{Q}\;[\mskip1.5mu \mskip1.5mu]\;\mathbin{:}\;[\mskip1.5mu \anonymous \mskip1.5mu]\;\Varid{sound:}\;\Varid{sound}\mskip1.5mu\}\;{}\<[E]%
\\
\>[B]{}\hsindent{8}{}\<[8]%
\>[8]{}(\Varid{lift:}\;\Varid{lift}\;\Varid{lift-β:}\;\Varid{β}\;\Varid{qind:}\;\Varid{qind})\;\mathrel{=}{}\<[E]%
\\
\>[B]{}\hsindent{3}{}\<[3]%
\>[3]{}\Keyword{record}{}\<[E]%
\\
\>[3]{}\hsindent{2}{}\<[5]%
\>[5]{}\{\mskip1.5mu \Varid{qelim}\;{}\<[15]%
\>[15]{}\mathrel{=}\;\Varid{λ}\;\{\mskip1.5mu \Conid{B}\mskip1.5mu\}\;\Varid{→}\;\Varid{qelim₁}\;\{\mskip1.5mu \Conid{B}\mskip1.5mu\}{}\<[E]%
\\
\>[3]{}\hsindent{2}{}\<[5]%
\>[5]{};\Varid{qelim-β}\;\mathrel{=}\;\Varid{λ}\;\{\mskip1.5mu \Conid{B}\mskip1.5mu\}\;\{\mskip1.5mu \Varid{a}\mskip1.5mu\}\;\{\mskip1.5mu \Varid{f}\mskip1.5mu\}\;\Varid{→}\;\Varid{qelim-β₁}\;\{\mskip1.5mu \Conid{B}\mskip1.5mu\}\;\Varid{a}\;\Varid{f}{}\<[E]%
\\
\>[3]{}\hsindent{2}{}\<[5]%
\>[5]{}\mskip1.5mu\}{}\<[E]%
\\
\>[B]{}\hsindent{3}{}\<[3]%
\>[3]{}\Keyword{where}{}\<[E]%
\\
\>[3]{}\hsindent{2}{}\<[5]%
\>[5]{}\Conid{A}\;{}\<[12]%
\>[12]{}\mathrel{=}\;\Conid{Carrier}\;\Conid{S}{}\<[E]%
\\
\>[3]{}\hsindent{2}{}\<[5]%
\>[5]{}\Varid{\char95 ∼\char95 }\;{}\<[12]%
\>[12]{}\mathrel{=}\;\Varid{\char95 ≈\char95 }\;\Conid{S}{}\<[E]%
\\[\blanklineskip]%
\>[3]{}\hsindent{2}{}\<[5]%
\>[5]{}\mbox{\onelinecomment  the dependent function f is made independent}{}\<[E]%
\\
\>[3]{}\hsindent{2}{}\<[5]%
\>[5]{}\Varid{indep}\;\mathbin{:}\;\{\mskip1.5mu \Conid{B}\;\mathbin{:}\;\Conid{Q}\;\Varid{→}\;\Conid{Set}\mskip1.5mu\}\;{}\<[28]%
\>[28]{}\Varid{→}\;((\Varid{a}\;\mathbin{:}\;\Conid{A})\;\Varid{→}\;\Conid{B}\;[\mskip1.5mu \Varid{a}\mskip1.5mu])\;\Varid{→}\;\Conid{A}\;\Varid{→}\;\Conid{Σ}\;\Conid{Q}\;\Conid{B}{}\<[E]%
\\
\>[3]{}\hsindent{2}{}\<[5]%
\>[5]{}\Varid{indep}\;\Varid{f}\;\Varid{a}\;\mathrel{=}\;[\mskip1.5mu \Varid{a}\mskip1.5mu],\Varid{f}\;\Varid{a}{}\<[E]%
\\[\blanklineskip]%
\>[3]{}\hsindent{2}{}\<[5]%
\>[5]{}\Varid{indep-β}\;\mathbin{:}\;\{\mskip1.5mu \Conid{B}\;\mathbin{:}\;\Conid{Q}\;\Varid{→}\;\Conid{Set}\mskip1.5mu\}\;{}\<[E]%
\\
\>[5]{}\hsindent{8}{}\<[13]%
\>[13]{}\Varid{→}\;(\Varid{f}\;\mathbin{:}\;(\Varid{a}\;\mathbin{:}\;\Conid{A})\;\Varid{→}\;\Conid{B}\;[\mskip1.5mu \Varid{a}\mskip1.5mu])\;{}\<[E]%
\\
\>[5]{}\hsindent{8}{}\<[13]%
\>[13]{}\Varid{→}\;(\Varid{∀}\;\Varid{a}\;\Varid{b}\;\Varid{→}\;(\Varid{p}\;\mathbin{:}\;\Varid{a}\;\Varid{∼}\;\Varid{b})\;\Varid{→}\;\Varid{subst}\;\Conid{B}\;(\Varid{sound}\;\Varid{p})\;(\Varid{f}\;\Varid{a})\;\Varid{≡}\;\Varid{f}\;\Varid{b})\;{}\<[E]%
\\
\>[5]{}\hsindent{8}{}\<[13]%
\>[13]{}\Varid{→}\;\Varid{∀}\;\Varid{a}\;\Varid{a'}\;\Varid{→}\;(\Varid{a}\;\Varid{∼}\;\Varid{a'})\;\Varid{→}\;\Varid{indep}\;\{\mskip1.5mu \Conid{B}\mskip1.5mu\}\;\Varid{f}\;\Varid{a}\;\Varid{≡}\;\Varid{indep}\;\Varid{f}\;\Varid{a'}{}\<[E]%
\\
\>[3]{}\hsindent{2}{}\<[5]%
\>[5]{}\Varid{indep-β}\;\{\mskip1.5mu \Conid{B}\mskip1.5mu\}\;\Varid{f}\;\Varid{q}\;\Varid{a}\;\Varid{a'}\;\Varid{p}\;\mathrel{=}\;(\Varid{cong\char95 },\anonymous \;[\mskip1.5mu \Varid{a}\mskip1.5mu]\;[\mskip1.5mu \Varid{a'}\mskip1.5mu]\;(\Varid{sound}\;\Varid{p})\;(\Varid{f}\;\Varid{a}))\;{}\<[E]%
\\
\>[5]{}\hsindent{23}{}\<[28]%
\>[28]{}\Varid{▶}\;((\Varid{λ}\;\Varid{b}\;\Varid{→}\;[\mskip1.5mu \Varid{a'}\mskip1.5mu],\Varid{b})\;\Varid{⋆}\;(\Varid{q}\;\Varid{a}\;\Varid{a'}\;\Varid{p})){}\<[E]%
\\[\blanklineskip]%
\>[3]{}\hsindent{2}{}\<[5]%
\>[5]{}\Varid{lift₀}\;\mathbin{:}\;\{\mskip1.5mu \Conid{B}\;\mathbin{:}\;\Conid{Q}\;\Varid{→}\;\Conid{Set}\mskip1.5mu\}\;{}\<[E]%
\\
\>[5]{}\hsindent{5}{}\<[10]%
\>[10]{}\Varid{→}\;(\Varid{f}\;\mathbin{:}\;(\Varid{a}\;\mathbin{:}\;\Conid{A})\;\Varid{→}\;(\Conid{B}\;[\mskip1.5mu \Varid{a}\mskip1.5mu]))\;{}\<[E]%
\\
\>[5]{}\hsindent{5}{}\<[10]%
\>[10]{}\Varid{→}\;((\Varid{a}\;\Varid{a'}\;\mathbin{:}\;\Conid{A})\;\Varid{→}\;(\Varid{p}\;\mathbin{:}\;\Varid{a}\;\Varid{∼}\;\Varid{a'}){}\<[E]%
\\
\>[5]{}\hsindent{5}{}\<[10]%
\>[10]{}\Varid{→}\;\Varid{subst}\;\Conid{B}\;(\Varid{sound}\;\Varid{p})\;{}\<[31]%
\>[31]{}(\Varid{f}\;\Varid{a})\;\Varid{≡}\;\Varid{f}\;\Varid{a'})\;{}\<[E]%
\\
\>[5]{}\hsindent{5}{}\<[10]%
\>[10]{}\Varid{→}\;\Conid{Q}\;\Varid{→}\;\Conid{Σ}\;\Conid{Q}\;\Conid{B}{}\<[E]%
\\
\>[3]{}\hsindent{2}{}\<[5]%
\>[5]{}\Varid{lift₀}\;\Varid{f}\;\Varid{q}\;\mathrel{=}\;\Varid{lift}\;(\Varid{indep}\;\Varid{f})\;(\Varid{indep-β}\;\Varid{f}\;\Varid{q}){}\<[E]%
\\[\blanklineskip]%
\>[3]{}\hsindent{2}{}\<[5]%
\>[5]{}\Varid{qind₁}\;\mathbin{:}\;\{\mskip1.5mu \Conid{B}\;\mathbin{:}\;\Conid{Q}\;\Varid{→}\;\Conid{Set}\mskip1.5mu\}\;{}\<[E]%
\\
\>[5]{}\hsindent{5}{}\<[10]%
\>[10]{}\Varid{→}\;(\Varid{f}\;\mathbin{:}\;(\Varid{a}\;\mathbin{:}\;\Conid{A})\;\Varid{→}\;\Conid{B}\;[\mskip1.5mu \Varid{a}\mskip1.5mu])\;{}\<[E]%
\\
\>[5]{}\hsindent{5}{}\<[10]%
\>[10]{}\Varid{→}\;(\Varid{q}\;\mathbin{:}\;\Varid{∀}\;\Varid{a}\;\Varid{b}\;\Varid{→}\;(\Varid{p}\;\mathbin{:}\;\Varid{a}\;\Varid{∼}\;\Varid{b})\;\Varid{→}\;\Varid{subst}\;\Conid{B}\;(\Varid{sound}\;\Varid{p})\;(\Varid{f}\;\Varid{a})\;\Varid{≡}\;\Varid{f}\;\Varid{b})\;{}\<[E]%
\\
\>[5]{}\hsindent{5}{}\<[10]%
\>[10]{}\Varid{→}\;\Varid{∀}\;(\Varid{c}\;\mathbin{:}\;\Conid{Q})\;\Varid{→}\;\Varid{proj₁}\;(\Varid{lift₀}\;\Varid{f}\;\Varid{q}\;\Varid{c})\;\Varid{≡}\;\Varid{c}{}\<[E]%
\\
\>[3]{}\hsindent{2}{}\<[5]%
\>[5]{}\Varid{qind₁}\;\{\mskip1.5mu \Conid{B}\mskip1.5mu\}\;\Varid{f}\;\Varid{q}\;\mathrel{=}\;\Varid{qind}\;\Conid{P}\;\Varid{heredity}\;\Varid{base}{}\<[E]%
\\
\>[5]{}\hsindent{2}{}\<[7]%
\>[7]{}\Keyword{where}{}\<[E]%
\\
\>[7]{}\hsindent{2}{}\<[9]%
\>[9]{}\Varid{f'}\;\mathbin{:}\;\Conid{Q}\;\Varid{→}\;\Conid{Σ}\;\Conid{Q}\;\Conid{B}{}\<[E]%
\\
\>[7]{}\hsindent{2}{}\<[9]%
\>[9]{}\Varid{f'}\;\mathrel{=}\;\Varid{lift₀}\;\Varid{f}\;\Varid{q}{}\<[E]%
\\
\>[7]{}\hsindent{2}{}\<[9]%
\>[9]{}\Conid{P}\;\mathbin{:}\;\Conid{Q}\;\Varid{→}\;\Conid{Set}{}\<[E]%
\\
\>[7]{}\hsindent{2}{}\<[9]%
\>[9]{}\Conid{P}\;\Varid{c}\;\mathrel{=}\;\Varid{proj₁}\;\{\mskip1.5mu \anonymous \mskip1.5mu\}\;\{\mskip1.5mu \anonymous \mskip1.5mu\}\;\{\mskip1.5mu \Conid{Q}\mskip1.5mu\}\;\{\mskip1.5mu \Conid{B}\mskip1.5mu\}\;(\Varid{lift₀}\;\Varid{f}\;\Varid{q}\;\Varid{c})\;\Varid{≡}\;\Varid{c}{}\<[E]%
\\
\>[7]{}\hsindent{2}{}\<[9]%
\>[9]{}\Varid{heredity}\;\mathbin{:}\;\Varid{∀}\;\Varid{x}\;\Varid{→}\;(\Varid{p}\;\Varid{p'}\;\mathbin{:}\;\Conid{P}\;\Varid{x})\;\Varid{→}\;\Varid{p}\;\Varid{≡}\;\Varid{p'}{}\<[E]%
\\
\>[7]{}\hsindent{2}{}\<[9]%
\>[9]{}\Varid{heredity}\;\Varid{x}\;\Varid{p}\;\Varid{p'}\;\mathrel{=}\;\Varid{≡-prfIrr}\;((\Varid{lift₀}\;\Varid{f}\;\Varid{q}\;\Varid{x})\;\Varid{₁})\;\Varid{x}\;\Varid{p}\;\Varid{p'}{}\<[E]%
\\
\>[7]{}\hsindent{2}{}\<[9]%
\>[9]{}\Varid{base}\;\mathbin{:}\;\Varid{∀}\;\Varid{a}\;\Varid{→}\;\Conid{P}\;[\mskip1.5mu \Varid{a}\mskip1.5mu]{}\<[E]%
\\
\>[7]{}\hsindent{2}{}\<[9]%
\>[9]{}\Varid{base}\;\Varid{a}\;\mathrel{=}\;\Varid{proj₁}\;\Varid{⋆}\;\Varid{β}{}\<[E]%
\\[\blanklineskip]%
\>[3]{}\hsindent{2}{}\<[5]%
\>[5]{}\Varid{qelim₁}\;{}\<[13]%
\>[13]{}\mathbin{:}\;\{\mskip1.5mu \Conid{B}\;\mathbin{:}\;\Conid{Q}\;\Varid{→}\;\Conid{Set}\mskip1.5mu\}\;{}\<[E]%
\\
\>[5]{}\hsindent{7}{}\<[12]%
\>[12]{}\Varid{→}\;(\Varid{f}\;\mathbin{:}\;(\Varid{a}\;\mathbin{:}\;\Conid{A})\;\Varid{→}\;(\Conid{B}\;[\mskip1.5mu \Varid{a}\mskip1.5mu]))\;{}\<[E]%
\\
\>[5]{}\hsindent{7}{}\<[12]%
\>[12]{}\Varid{→}\;(\Varid{∀}\;\Varid{a}\;\Varid{b}\;\Varid{→}\;(\Varid{p}\;\mathbin{:}\;\Varid{a}\;\Varid{∼}\;\Varid{b})\;\Varid{→}\;\Varid{subst}\;\Conid{B}\;(\Varid{sound}\;\Varid{p})\;(\Varid{f}\;\Varid{a})\;\Varid{≡}\;\Varid{f}\;\Varid{b})\;{}\<[E]%
\\
\>[5]{}\hsindent{7}{}\<[12]%
\>[12]{}\Varid{→}\;(\Varid{c}\;\mathbin{:}\;\Conid{Q})\;\Varid{→}\;\Conid{B}\;\Varid{c}{}\<[E]%
\\
\>[3]{}\hsindent{2}{}\<[5]%
\>[5]{}\Varid{qelim₁}\;\{\mskip1.5mu \Conid{B}\mskip1.5mu\}\;\Varid{f}\;\Varid{q}\;\Varid{c}\;\mathrel{=}\;\Varid{subst}\;\Conid{B}\;(\Varid{qind₁}\;\Varid{f}\;\Varid{q}\;\Varid{c})\;{}\<[E]%
\\
\>[5]{}\hsindent{18}{}\<[23]%
\>[23]{}(\Varid{proj₂}\;\{\mskip1.5mu \anonymous \mskip1.5mu\}\;\{\mskip1.5mu \anonymous \mskip1.5mu\}\;\{\mskip1.5mu \Conid{Q}\mskip1.5mu\}\;\{\mskip1.5mu \Conid{B}\mskip1.5mu\}\;(\Varid{lift₀}\;\Varid{f}\;\Varid{q}\;\Varid{c})){}\<[E]%
\\[\blanklineskip]%
\>[3]{}\hsindent{2}{}\<[5]%
\>[5]{}\Varid{qelim-β₁}\;\mathbin{:}\;\Varid{∀}\;\{\mskip1.5mu \Conid{B}\mskip1.5mu\}\;\Varid{a}\;\Varid{f}\;\Varid{q}\;\Varid{→}\;\Varid{qelim₁}\;\{\mskip1.5mu \Conid{B}\mskip1.5mu\}\;\Varid{f}\;\Varid{q}\;[\mskip1.5mu \Varid{a}\mskip1.5mu]\;\Varid{≡}\;\Varid{f}\;\Varid{a}{}\<[E]%
\\
\>[3]{}\hsindent{2}{}\<[5]%
\>[5]{}\Varid{qelim-β₁}\;\{\mskip1.5mu \Conid{B}\mskip1.5mu\}\;\Varid{a}\;\Varid{f}\;\Varid{q}\;\mathrel{=}\;{}\<[E]%
\\
\>[5]{}\hsindent{2}{}\<[7]%
\>[7]{}(\Varid{substIrr}\;\Conid{B}\;(\Varid{qind₁}\;\Varid{f}\;\Varid{q}\;[\mskip1.5mu \Varid{a}\mskip1.5mu])\;{}\<[E]%
\\
\>[7]{}\hsindent{2}{}\<[9]%
\>[9]{}(\Varid{cong-proj₁}\;\{\mskip1.5mu \Conid{Q}\mskip1.5mu\}\;\{\mskip1.5mu \Conid{B}\mskip1.5mu\}\;(\Varid{lift₀}\;\Varid{f}\;\Varid{q}\;[\mskip1.5mu \Varid{a}\mskip1.5mu])\;(\Varid{indep}\;\Varid{f}\;\Varid{a})\;\Varid{β})\;{}\<[E]%
\\
\>[7]{}\hsindent{2}{}\<[9]%
\>[9]{}(\Varid{proj₂}\;\{\mskip1.5mu \anonymous \mskip1.5mu\}\;\{\mskip1.5mu \anonymous \mskip1.5mu\}\;\{\mskip1.5mu \Conid{Q}\mskip1.5mu\}\;\{\mskip1.5mu \Conid{B}\mskip1.5mu\}\;(\Varid{lift₀}\;\Varid{f}\;\Varid{q}\;[\mskip1.5mu \Varid{a}\mskip1.5mu])))\;\Varid{▶}\;{}\<[E]%
\\
\>[5]{}\hsindent{2}{}\<[7]%
\>[7]{}(\Varid{cong-proj₂}\;\{\mskip1.5mu \Conid{Q}\mskip1.5mu\}\;\{\mskip1.5mu \Conid{B}\mskip1.5mu\}\;(\Varid{lift₀}\;\Varid{f}\;\Varid{q}\;[\mskip1.5mu \Varid{a}\mskip1.5mu])\;(\Varid{indep}\;\Varid{f}\;\Varid{a})\;\Varid{β}){}\<[E]%
\ColumnHook
\end{hscode}\resethooks


\begin{hscode}\SaveRestoreHook
\column{B}{@{}>{\hspre}l<{\hspost}@{}}%
\column{3}{@{}>{\hspre}l<{\hspost}@{}}%
\column{5}{@{}>{\hspre}l<{\hspost}@{}}%
\column{8}{@{}>{\hspre}l<{\hspost}@{}}%
\column{10}{@{}>{\hspre}l<{\hspost}@{}}%
\column{12}{@{}>{\hspre}l<{\hspost}@{}}%
\column{E}{@{}>{\hspre}l<{\hspost}@{}}%
\>[B]{}\Conid{Qu→QuH}\;\mathbin{:}\;\{\mskip1.5mu \Conid{S}\;\mathbin{:}\;\Conid{Setoid}\mskip1.5mu\}\;\Varid{→}\;\{\mskip1.5mu \Conid{PQ}\;\mathbin{:}\;\Conid{PreQu}\;\Conid{S}\mskip1.5mu\}\;{}\<[E]%
\\
\>[B]{}\hsindent{8}{}\<[8]%
\>[8]{}\Varid{→}\;(\Conid{Qu}\;\Conid{PQ})\;\Varid{→}\;(\Conid{QuH}\;\Conid{PQ}){}\<[E]%
\\
\>[B]{}\Conid{Qu→QuH}\;\{\mskip1.5mu \Conid{S}\mskip1.5mu\}\;\{\mskip1.5mu \Conid{Q:}\;\Conid{Q}\;[\mskip1.5mu \mskip1.5mu]\;\mathbin{:}\;[\mskip1.5mu \anonymous \mskip1.5mu]\;\Varid{sound:}\;\Varid{sound}\mskip1.5mu\}\;(\Varid{qelim:}\;\Varid{qelim}\;\Varid{qelim-β:}\;\Varid{β})\;\mathrel{=}{}\<[E]%
\\
\>[B]{}\hsindent{3}{}\<[3]%
\>[3]{}\Keyword{record}{}\<[E]%
\\
\>[B]{}\hsindent{3}{}\<[3]%
\>[3]{}\{\mskip1.5mu \Varid{lift}\;{}\<[12]%
\>[12]{}\mathrel{=}\;\Varid{λ}\;\{\mskip1.5mu \Conid{B}\mskip1.5mu\}\;\Varid{f}\;\Varid{s}\;\Varid{→}\;\Varid{qelim}\;\{\mskip1.5mu \Varid{λ}\;\anonymous \;\Varid{→}\;\Conid{B}\mskip1.5mu\}\;\Varid{f}\;(\Varid{λ}\;\Varid{a}\;\Varid{b}\;\Varid{p}{}\<[E]%
\\
\>[12]{}\Varid{→}\;(\Varid{subFix}\;(\Varid{sound}\;\Varid{p})\;\Conid{B}\;(\Varid{f}\;\Varid{a}))\;\Varid{▶}\;(\Varid{s}\;\Varid{a}\;\Varid{b}\;\Varid{p})){}\<[E]%
\\
\>[B]{}\hsindent{3}{}\<[3]%
\>[3]{};\Varid{lift-β}\;\mathrel{=}\;\Varid{λ}\;\{\mskip1.5mu \Conid{B}\mskip1.5mu\}\;\{\mskip1.5mu \Varid{a'}\mskip1.5mu\}\;\{\mskip1.5mu \Varid{f}\mskip1.5mu\}\;\{\mskip1.5mu \Varid{s}\mskip1.5mu\}\;\Varid{→}\;\Varid{β}\;\{\mskip1.5mu \Varid{λ}\;\anonymous \;\Varid{→}\;\Conid{B}\mskip1.5mu\}\;\{\mskip1.5mu \Varid{a'}\mskip1.5mu\}\;\{\mskip1.5mu \Varid{f}\mskip1.5mu\}\;(\Varid{λ}\;\Varid{a}\;\Varid{b}\;\Varid{p}{}\<[E]%
\\
\>[3]{}\hsindent{9}{}\<[12]%
\>[12]{}\Varid{→}\;(\Varid{subFix}\;(\Varid{sound}\;\Varid{p})\;\Conid{B}\;(\Varid{f}\;\Varid{a}))\;\Varid{▶}\;(\Varid{s}\;\Varid{a}\;\Varid{b}\;\Varid{p})){}\<[E]%
\\
\>[B]{}\hsindent{3}{}\<[3]%
\>[3]{};\Varid{qind}\;\mathrel{=}\;\Varid{λ}\;\Conid{P}\;\Varid{irr}\;\Varid{f}\;{}\<[E]%
\\
\>[3]{}\hsindent{7}{}\<[10]%
\>[10]{}\Varid{→}\;\Varid{qelim}\;\{\mskip1.5mu \Conid{P}\mskip1.5mu\}\;\Varid{f}\;(\Varid{λ}\;\Varid{a}\;\Varid{b}\;\Varid{p}\;\Varid{→}\;\Varid{irr}\;[\mskip1.5mu \Varid{b}\mskip1.5mu]\;(\Varid{subst}\;\Conid{P}\;(\Varid{sound}\;\Varid{p})\;(\Varid{f}\;\Varid{a}))\;(\Varid{f}\;\Varid{b})){}\<[E]%
\\
\>[B]{}\hsindent{3}{}\<[3]%
\>[3]{}\mskip1.5mu\}{}\<[E]%
\\
\>[B]{}\hsindent{3}{}\<[3]%
\>[3]{}\Keyword{where}{}\<[E]%
\\
\>[3]{}\hsindent{2}{}\<[5]%
\>[5]{}\Varid{subFix}\;\mathbin{:}\;\Varid{∀}\;\{\mskip1.5mu \Conid{A}\;\mathbin{:}\;\Conid{Set}\mskip1.5mu\}\;\{\mskip1.5mu \Varid{c}\;\Varid{d}\;\mathbin{:}\;\Conid{A}\mskip1.5mu\}\;(\Varid{x}\;\mathbin{:}\;\Varid{c}\;\Varid{≡}\;\Varid{d})\;(\Conid{B}\;\mathbin{:}\;\Conid{Set})\;(\Varid{p}\;\mathbin{:}\;\Conid{B})\;{}\<[E]%
\\
\>[5]{}\hsindent{7}{}\<[12]%
\>[12]{}\Varid{→}\;\Varid{subst}\;(\Varid{λ}\;\anonymous \;\Varid{→}\;\Conid{B})\;\Varid{x}\;\Varid{p}\;\Varid{≡}\;\Varid{p}{}\<[E]%
\\
\>[3]{}\hsindent{2}{}\<[5]%
\>[5]{}\Varid{subFix}\;\Varid{refl}\;\anonymous \;\anonymous \;\mathrel{=}\;\Varid{refl}{}\<[E]%
\ColumnHook
\end{hscode}\resethooks


\begin{hscode}\SaveRestoreHook
\column{B}{@{}>{\hspre}l<{\hspost}@{}}%
\column{3}{@{}>{\hspre}l<{\hspost}@{}}%
\column{5}{@{}>{\hspre}l<{\hspost}@{}}%
\column{7}{@{}>{\hspre}l<{\hspost}@{}}%
\column{9}{@{}>{\hspre}l<{\hspost}@{}}%
\column{10}{@{}>{\hspre}l<{\hspost}@{}}%
\column{12}{@{}>{\hspre}l<{\hspost}@{}}%
\column{13}{@{}>{\hspre}l<{\hspost}@{}}%
\column{14}{@{}>{\hspre}l<{\hspost}@{}}%
\column{21}{@{}>{\hspre}l<{\hspost}@{}}%
\column{E}{@{}>{\hspre}l<{\hspost}@{}}%
\>[B]{}\Conid{QuD→QuE}\;\mathbin{:}\;\{\mskip1.5mu \Conid{S}\;\mathbin{:}\;\Conid{Setoid}\mskip1.5mu\}\;\{\mskip1.5mu \Conid{PQ}\;\mathbin{:}\;\Conid{PreQu}\;\Conid{S}\mskip1.5mu\}\;\{\mskip1.5mu \Conid{QU}\;\mathbin{:}\;\Conid{Qu}\;\Conid{PQ}\mskip1.5mu\}\;{}\<[E]%
\\
\>[B]{}\hsindent{9}{}\<[9]%
\>[9]{}\Varid{→}\;(\Conid{QuD}\;\Conid{PQ})\;\Varid{→}\;(\Conid{QuE}\;\Conid{QU}){}\<[E]%
\\
\>[B]{}\Conid{QuD→QuE}\;\{\mskip1.5mu \Conid{S}\mskip1.5mu\}\;\{\mskip1.5mu \Conid{Q:}\;\Conid{Q}\;[\mskip1.5mu \mskip1.5mu]\;\mathbin{:}\;[\mskip1.5mu \anonymous \mskip1.5mu]\;\Varid{sound:}\;\anonymous \mskip1.5mu\}\;{}\<[E]%
\\
\>[B]{}\hsindent{9}{}\<[9]%
\>[9]{}(\Varid{emb:}\;\Varid{emb}\;\Varid{complete:}\;\Varid{complete}\;\Varid{stable:}\;\anonymous )\;\mathrel{=}{}\<[E]%
\\
\>[B]{}\hsindent{3}{}\<[3]%
\>[3]{}\Keyword{record}\;\{\mskip1.5mu \Varid{exact}\;\mathrel{=}\;{}\<[21]%
\>[21]{}\Varid{λ}\;\{\mskip1.5mu \Varid{a}\mskip1.5mu\}\;\{\mskip1.5mu \Varid{b}\mskip1.5mu\}\;[\mskip1.5mu \Varid{a}\mskip1.5mu]\;\Varid{≡}\;[\mskip1.5mu \Varid{b}\mskip1.5mu]{}\<[E]%
\\
\>[3]{}\hsindent{7}{}\<[10]%
\>[10]{}\Varid{→}\;\Varid{⟨}\;\Varid{complete}\;\Varid{a}\;\Varid{⟩₀}\;{}\<[E]%
\\
\>[10]{}\hsindent{2}{}\<[12]%
\>[12]{}\Varid{▶₀}\;\Varid{subst}\;(\Varid{λ}\;\Varid{x}\;\Varid{→}\;\Varid{x}\;\Varid{∼}\;\Varid{b})\;(\Varid{emb}\;\Varid{⋆}\;\Varid{⟨}\;[\mskip1.5mu \Varid{a}\mskip1.5mu]\;\Varid{≡}\;[\mskip1.5mu \Varid{b}\mskip1.5mu]\;\Varid{⟩})\;(\Varid{complete}\;\Varid{b}){}\<[E]%
\\
\>[3]{}\hsindent{7}{}\<[10]%
\>[10]{}\mskip1.5mu\}{}\<[E]%
\\
\>[3]{}\hsindent{2}{}\<[5]%
\>[5]{}\Keyword{where}{}\<[E]%
\\
\>[5]{}\hsindent{2}{}\<[7]%
\>[7]{}\Conid{A}\;{}\<[13]%
\>[13]{}\mathrel{=}\;\Conid{Carrier}\;\Conid{S}{}\<[E]%
\\
\>[5]{}\hsindent{2}{}\<[7]%
\>[7]{}\Varid{\char95 ∼\char95 }\;{}\<[13]%
\>[13]{}\mathrel{=}\;\Varid{\char95 ≈\char95 }\;\Conid{S}{}\<[E]%
\\
\>[5]{}\hsindent{2}{}\<[7]%
\>[7]{}\Varid{⟨\char95 ⟩₀}\;{}\<[14]%
\>[14]{}\mathbin{:}\;\Conid{Symmetric}\;\Varid{\char95 ∼\char95 }{}\<[E]%
\\
\>[5]{}\hsindent{2}{}\<[7]%
\>[7]{}\Varid{⟨\char95 ⟩₀}\;{}\<[14]%
\>[14]{}\mathrel{=}\;\Varid{symmetric}\;\Conid{S}{}\<[E]%
\\
\>[5]{}\hsindent{2}{}\<[7]%
\>[7]{}\Varid{\char95 ▶₀\char95 }\;{}\<[13]%
\>[13]{}\mathbin{:}\;\Conid{Transitive}\;\Varid{\char95 ∼\char95 }{}\<[E]%
\\
\>[5]{}\hsindent{2}{}\<[7]%
\>[7]{}\Varid{\char95 ▶₀\char95 }\;{}\<[13]%
\>[13]{}\mathrel{=}\;\Varid{transitive}\;\Conid{S}{}\<[E]%
\ColumnHook
\end{hscode}\resethooks


\begin{hscode}\SaveRestoreHook
\column{B}{@{}>{\hspre}l<{\hspost}@{}}%
\column{3}{@{}>{\hspre}l<{\hspost}@{}}%
\column{8}{@{}>{\hspre}l<{\hspost}@{}}%
\column{13}{@{}>{\hspre}l<{\hspost}@{}}%
\column{15}{@{}>{\hspre}l<{\hspost}@{}}%
\column{E}{@{}>{\hspre}l<{\hspost}@{}}%
\>[B]{}\Conid{QuD→Qu}\;\mathbin{:}\;\{\mskip1.5mu \Conid{S}\;\mathbin{:}\;\Conid{Setoid}\mskip1.5mu\}\;\Varid{→}\;\{\mskip1.5mu \Conid{PQ}\;\mathbin{:}\;\Conid{PreQu}\;\Conid{S}\mskip1.5mu\}\;{}\<[E]%
\\
\>[B]{}\hsindent{8}{}\<[8]%
\>[8]{}\Varid{→}\;(\Conid{QuD}\;\Conid{PQ})\;\Varid{→}\;(\Conid{Qu}\;\Conid{PQ}){}\<[E]%
\\
\>[B]{}\Conid{QuD→Qu}\;\{\mskip1.5mu \Conid{S}\mskip1.5mu\}\;\{\mskip1.5mu \Conid{Q:}\;\Conid{Q}\;[\mskip1.5mu \mskip1.5mu]\;\mathbin{:}\;[\mskip1.5mu \anonymous \mskip1.5mu]\;\Varid{sound:}\;\Varid{sound}\mskip1.5mu\}\;{}\<[E]%
\\
\>[B]{}\hsindent{8}{}\<[8]%
\>[8]{}(\Varid{emb:}\;\Varid{⌜\char95 ⌝}\;\Varid{complete:}\;\Varid{complete}\;\Varid{stable:}\;\Varid{stable})\;\mathrel{=}{}\<[E]%
\\
\>[B]{}\hsindent{3}{}\<[3]%
\>[3]{}\Keyword{record}{}\<[E]%
\\
\>[B]{}\hsindent{3}{}\<[3]%
\>[3]{}\{\mskip1.5mu \Varid{qelim}\;{}\<[13]%
\>[13]{}\mathrel{=}\;\Varid{λ}\;\{\mskip1.5mu \Conid{B}\mskip1.5mu\}\;\Varid{f}\;\anonymous \;\Varid{a}\;\Varid{→}\;\Varid{subst}\;\Conid{B}\;(\Varid{stable}\;\Varid{a})\;(\Varid{f}\;\Varid{⌜}\;\Varid{a}\;\Varid{⌝}){}\<[E]%
\\
\>[B]{}\hsindent{3}{}\<[3]%
\>[3]{};\Varid{qelim-β}\;\mathrel{=}\;\Varid{λ}\;\{\mskip1.5mu \Conid{B}\mskip1.5mu\}\;\{\mskip1.5mu \Varid{a}\mskip1.5mu\}\;\{\mskip1.5mu \Varid{f}\mskip1.5mu\}\;\Varid{s}\;{}\<[E]%
\\
\>[3]{}\hsindent{12}{}\<[15]%
\>[15]{}\Varid{→}\;\Varid{substIrr}\;\Conid{B}\;(\Varid{stable}\;[\mskip1.5mu \Varid{a}\mskip1.5mu])\;(\Varid{sound}\;(\Varid{complete}\;\Varid{a}))\;(\Varid{f}\;\Varid{⌜}\;[\mskip1.5mu \Varid{a}\mskip1.5mu]\;\Varid{⌝})\;{}\<[E]%
\\
\>[3]{}\hsindent{12}{}\<[15]%
\>[15]{}\Varid{▶}\;\Varid{s}\;\anonymous \;\anonymous \;(\Varid{complete}\;\Varid{a}){}\<[E]%
\\
\>[B]{}\hsindent{3}{}\<[3]%
\>[3]{}\mskip1.5mu\}{}\<[E]%
\ColumnHook
\end{hscode}\resethooks


\begin{hscode}\SaveRestoreHook
\column{B}{@{}>{\hspre}l<{\hspost}@{}}%
\column{3}{@{}>{\hspre}l<{\hspost}@{}}%
\column{9}{@{}>{\hspre}l<{\hspost}@{}}%
\column{12}{@{}>{\hspre}l<{\hspost}@{}}%
\column{E}{@{}>{\hspre}l<{\hspost}@{}}%
\>[B]{}\Conid{QuD→QuH}\;\mathbin{:}\;\{\mskip1.5mu \Conid{S}\;\mathbin{:}\;\Conid{Setoid}\mskip1.5mu\}\;\Varid{→}\;\{\mskip1.5mu \Conid{PQ}\;\mathbin{:}\;\Conid{PreQu}\;\Conid{S}\mskip1.5mu\}\;{}\<[E]%
\\
\>[B]{}\hsindent{9}{}\<[9]%
\>[9]{}\Varid{→}\;(\Conid{QuD}\;\Conid{PQ})\;\Varid{→}\;(\Conid{QuH}\;\Conid{PQ}){}\<[E]%
\\
\>[B]{}\Conid{QuD→QuH}\;\{\mskip1.5mu \Conid{S}\mskip1.5mu\}\;\{\mskip1.5mu \Conid{Q:}\;\Conid{Q}\;[\mskip1.5mu \mskip1.5mu]\;\mathbin{:}\;[\mskip1.5mu \anonymous \mskip1.5mu]\;\Varid{sound:}\;\Varid{sound}\mskip1.5mu\}\;{}\<[E]%
\\
\>[B]{}\hsindent{9}{}\<[9]%
\>[9]{}(\Varid{emb:}\;\Varid{⌜\char95 ⌝}\;\Varid{complete:}\;\Varid{complete}\;\Varid{stable:}\;\Varid{stable})\;\mathrel{=}{}\<[E]%
\\
\>[B]{}\hsindent{3}{}\<[3]%
\>[3]{}\Keyword{record}{}\<[E]%
\\
\>[B]{}\hsindent{3}{}\<[3]%
\>[3]{}\{\mskip1.5mu \Varid{lift}\;{}\<[12]%
\>[12]{}\mathrel{=}\;\Varid{λ}\;\Varid{f}\;\anonymous \;\Varid{q}\;\Varid{→}\;\Varid{f}\;\Varid{⌜}\;\Varid{q}\;\Varid{⌝}{}\<[E]%
\\
\>[B]{}\hsindent{3}{}\<[3]%
\>[3]{};\Varid{lift-β}\;\mathrel{=}\;\Varid{λ}\;\{\mskip1.5mu \Conid{B}\mskip1.5mu\}\;\{\mskip1.5mu \Varid{a}\mskip1.5mu\}\;\{\mskip1.5mu \Varid{f}\mskip1.5mu\}\;\{\mskip1.5mu \Varid{s}\mskip1.5mu\}\;\Varid{→}\;\Varid{s}\;\Varid{⌜}\;[\mskip1.5mu \Varid{a}\mskip1.5mu]\;\Varid{⌝}\;\Varid{a}\;(\Varid{complete}\;\Varid{a}){}\<[E]%
\\
\>[B]{}\hsindent{3}{}\<[3]%
\>[3]{};\Varid{qind}\;{}\<[12]%
\>[12]{}\mathrel{=}\;\Varid{λ}\;\Conid{P}\;\anonymous \;\Varid{f}\;\Varid{→}\;\Varid{λ}\;\Varid{x}\;\Varid{→}\;\Varid{subst}\;\Conid{P}\;(\Varid{stable}\;\Varid{x})\;(\Varid{f}\;\Varid{⌜}\;\Varid{x}\;\Varid{⌝}){}\<[E]%
\\
\>[B]{}\hsindent{3}{}\<[3]%
\>[3]{}\mskip1.5mu\}{}\<[E]%
\ColumnHook
\end{hscode}\resethooks

Or
\begin{hscode}\SaveRestoreHook
\column{B}{@{}>{\hspre}l<{\hspost}@{}}%
\column{10}{@{}>{\hspre}l<{\hspost}@{}}%
\column{E}{@{}>{\hspre}l<{\hspost}@{}}%
\>[B]{}\Conid{QuD→QuH'}\;\mathbin{:}\;\{\mskip1.5mu \Conid{S}\;\mathbin{:}\;\Conid{Setoid}\mskip1.5mu\}\;\Varid{→}\;\{\mskip1.5mu \Conid{PQ}\;\mathbin{:}\;\Conid{PreQu}\;\Conid{S}\mskip1.5mu\}\;{}\<[E]%
\\
\>[B]{}\hsindent{10}{}\<[10]%
\>[10]{}\Varid{→}\;(\Conid{QuD}\;\Conid{PQ})\;\Varid{→}\;(\Conid{QuH}\;\Conid{PQ}){}\<[E]%
\\
\>[B]{}\Conid{QuD→QuH'}\;\{\mskip1.5mu \Conid{S}\mskip1.5mu\}\;\mathrel{=}\;\Conid{Qu→QuH}\;\Varid{∘}\;\Conid{QuD→Qu}{}\<[E]%
\ColumnHook
\end{hscode}\resethooks






\section{Definition of the enumeration $\varphi$ of natural sequences}\label{app:haskell:phi}
We define a family $\set{\varphi_i:\N\to\N}_{i:\N}$ of natural sequences with the property that any finite sequence $[x_0,\dots,x_k]$ of natural numbers is a prefix of some $\varphi_i$, which we need in some of our examples. Furthermore, although not strictly needed here, these sequences are pairwise distinct. One idea to define such a family is to request that the sequences $\varphi_{2i}$ at even indices are those starting with $0$ while the others are in turn split into  those starting with $1$ (the $\varphi_{2i+1+2k}$, i.e. every other sequence of odd index) and the remaining sequences, starting with at least $2$, etc. For each subfamily of sequences starting with the same prefix of length $n$, we define the $(n+1)^{\text{th}}$ term in the same manner.
Table~\ref{tab:phi} shows the prefix of the first 21 sequences.

\noindent
\begin{table}\label{tab:phi}
\[\begin{array}{|c|c|c|c|c|c|c|c|c|c|c|c|c|c|c|c|c|c|c|c|c|}
\hline
\varphi_{0} & \varphi_{1} & \varphi_{2} & \varphi_{3} & \varphi_{4} & \varphi_{5} & \varphi_{6} & \varphi_{7} & \varphi_{8} & \varphi_{9} & \varphi_{10} & \varphi_{11} & \varphi_{12} & \varphi_{13} & \varphi_{14} & \varphi_{15} & \varphi_{16} & \varphi_{17} & \varphi_{18} & \varphi_{19} & \varphi_{20}\\
\hline
0 & 1 & 0 & 2 & 0 & 1 & 0 & 3 & 0 & 1 & 0 & 2 & 0 & 1 & 0 & 4 & 0 & 1 & 0 & 2 & 0\\
0 & 0 & 1 & 0 & 0 & 1 & 2 & 0 & 0 & 0 & 1 & 1 & 0 & 2 & 3 & 0 & 0 & 0 & 1 & 0 & 0\\
0 & 0 & 0 & 0 & 1 & 0 & 0 & 0 & 0 & 1 & 1 & 0 & 2 & 0 & 0 & 0 & 0 & 0 & 0 & 1 & 1\\
0 & 0 & 0 & 0 & 0 & 0 & 0 & 0 & 1 & 0 & 0 & 0 & 0 & 0 & 0 & 0 & 0 & 1 & 1 & 0 & 1\\
0 & 0 & 0 & 0 & 0 & 0 & 0 & 0 & 0 & 0 & 0 & 0 & 0 & 0 & 0 & 0 & 1 & 0 & 0 & 0 & 0\\
0 & 0 & 0 & 0 & 0 & 0 & 0 & 0 & 0 & 0 & 0 & 0 & 0 & 0 & 0 & 0 & 0 & 0 & 0 & 0 & 0\\
\vdots & \vdots & \vdots & \vdots & \vdots & \vdots & \vdots & \vdots & \vdots & \vdots & \vdots & \vdots & \vdots & \vdots & \vdots & \vdots & \vdots & \vdots & \vdots & \vdots & \vdots\\
\hline
\end{array}\]
\caption{Prefixes of $\varphi_0$ to $\varphi_{20}$}
\end{table}

Here are complete definitions written in the programming language Haskell. The first version is perhaps the easiest to understand but is given in terms of lists. The second version is a direct translation of the former in the language of functions directly. Finally, the third version is more direct and perhaps easier to read for some people.


% -- Thomas Anberree -- June 2011
%% ODER: format ==         = "\mathrel{==}"
%% ODER: format /=         = "\neq "
%
%
\makeatletter
\@ifundefined{lhs2tex.lhs2tex.sty.read}%
  {\@namedef{lhs2tex.lhs2tex.sty.read}{}%
   \newcommand\SkipToFmtEnd{}%
   \newcommand\EndFmtInput{}%
   \long\def\SkipToFmtEnd#1\EndFmtInput{}%
  }\SkipToFmtEnd

\newcommand\ReadOnlyOnce[1]{\@ifundefined{#1}{\@namedef{#1}{}}\SkipToFmtEnd}
\usepackage{amstext}
\usepackage{amssymb}
\usepackage{stmaryrd}
\DeclareFontFamily{OT1}{cmtex}{}
\DeclareFontShape{OT1}{cmtex}{m}{n}
  {<5><6><7><8>cmtex8
   <9>cmtex9
   <10><10.95><12><14.4><17.28><20.74><24.88>cmtex10}{}
\DeclareFontShape{OT1}{cmtex}{m}{it}
  {<-> ssub * cmtt/m/it}{}
\newcommand{\texfamily}{\fontfamily{cmtex}\selectfont}
\DeclareFontShape{OT1}{cmtt}{bx}{n}
  {<5><6><7><8>cmtt8
   <9>cmbtt9
   <10><10.95><12><14.4><17.28><20.74><24.88>cmbtt10}{}
\DeclareFontShape{OT1}{cmtex}{bx}{n}
  {<-> ssub * cmtt/bx/n}{}
\newcommand{\tex}[1]{\text{\texfamily#1}}	% NEU

\newcommand{\Sp}{\hskip.33334em\relax}


\newcommand{\Conid}[1]{\mathit{#1}}
\newcommand{\Varid}[1]{\mathit{#1}}
\newcommand{\anonymous}{\kern0.06em \vbox{\hrule\@width.5em}}
\newcommand{\plus}{\mathbin{+\!\!\!+}}
\newcommand{\bind}{\mathbin{>\!\!\!>\mkern-6.7mu=}}
\newcommand{\rbind}{\mathbin{=\mkern-6.7mu<\!\!\!<}}% suggested by Neil Mitchell
\newcommand{\sequ}{\mathbin{>\!\!\!>}}
\renewcommand{\leq}{\leqslant}
\renewcommand{\geq}{\geqslant}
\usepackage{polytable}

%mathindent has to be defined
\@ifundefined{mathindent}%
  {\newdimen\mathindent\mathindent\leftmargini}%
  {}%

\def\resethooks{%
  \global\let\SaveRestoreHook\empty
  \global\let\ColumnHook\empty}
\newcommand*{\savecolumns}[1][default]%
  {\g@addto@macro\SaveRestoreHook{\savecolumns[#1]}}
\newcommand*{\restorecolumns}[1][default]%
  {\g@addto@macro\SaveRestoreHook{\restorecolumns[#1]}}
\newcommand*{\aligncolumn}[2]%
  {\g@addto@macro\ColumnHook{\column{#1}{#2}}}

\resethooks

\newcommand{\onelinecommentchars}{\quad-{}- }
\newcommand{\commentbeginchars}{\enskip\{-}
\newcommand{\commentendchars}{-\}\enskip}

\newcommand{\visiblecomments}{%
  \let\onelinecomment=\onelinecommentchars
  \let\commentbegin=\commentbeginchars
  \let\commentend=\commentendchars}

\newcommand{\invisiblecomments}{%
  \let\onelinecomment=\empty
  \let\commentbegin=\empty
  \let\commentend=\empty}

\visiblecomments

\newlength{\blanklineskip}
\setlength{\blanklineskip}{0.66084ex}

\newcommand{\hsindent}[1]{\quad}% default is fixed indentation
\let\hspre\empty
\let\hspost\empty
\newcommand{\NB}{\textbf{NB}}
\newcommand{\Todo}[1]{$\langle$\textbf{To do:}~#1$\rangle$}

\EndFmtInput
\makeatother
%
%
%
%
%
%
% This package provides two environments suitable to take the place
% of hscode, called "plainhscode" and "arrayhscode". 
%
% The plain environment surrounds each code block by vertical space,
% and it uses \abovedisplayskip and \belowdisplayskip to get spacing
% similar to formulas. Note that if these dimensions are changed,
% the spacing around displayed math formulas changes as well.
% All code is indented using \leftskip.
%
% Changed 19.08.2004 to reflect changes in colorcode. Should work with
% CodeGroup.sty.
%
\ReadOnlyOnce{polycode.fmt}%
\makeatletter

\newcommand{\hsnewpar}[1]%
  {{\parskip=0pt\parindent=0pt\par\vskip #1\noindent}}

% can be used, for instance, to redefine the code size, by setting the
% command to \small or something alike
\newcommand{\hscodestyle}{}

% The command \sethscode can be used to switch the code formatting
% behaviour by mapping the hscode environment in the subst directive
% to a new LaTeX environment.

\newcommand{\sethscode}[1]%
  {\expandafter\let\expandafter\hscode\csname #1\endcsname
   \expandafter\let\expandafter\endhscode\csname end#1\endcsname}

% "compatibility" mode restores the non-polycode.fmt layout.

\newenvironment{compathscode}%
  {\par\noindent
   \advance\leftskip\mathindent
   \hscodestyle
   \let\\=\@normalcr
   \let\hspre\(\let\hspost\)%
   \pboxed}%
  {\endpboxed\)%
   \par\noindent
   \ignorespacesafterend}

\newcommand{\compaths}{\sethscode{compathscode}}

% "plain" mode is the proposed default.
% It should now work with \centering.
% This required some changes. The old version
% is still available for reference as oldplainhscode.

\newenvironment{plainhscode}%
  {\hsnewpar\abovedisplayskip
   \advance\leftskip\mathindent
   \hscodestyle
   \let\hspre\(\let\hspost\)%
   \pboxed}%
  {\endpboxed%
   \hsnewpar\belowdisplayskip
   \ignorespacesafterend}

\newenvironment{oldplainhscode}%
  {\hsnewpar\abovedisplayskip
   \advance\leftskip\mathindent
   \hscodestyle
   \let\\=\@normalcr
   \(\pboxed}%
  {\endpboxed\)%
   \hsnewpar\belowdisplayskip
   \ignorespacesafterend}

% Here, we make plainhscode the default environment.

\newcommand{\plainhs}{\sethscode{plainhscode}}
\newcommand{\oldplainhs}{\sethscode{oldplainhscode}}
\plainhs

% The arrayhscode is like plain, but makes use of polytable's
% parray environment which disallows page breaks in code blocks.

\newenvironment{arrayhscode}%
  {\hsnewpar\abovedisplayskip
   \advance\leftskip\mathindent
   \hscodestyle
   \let\\=\@normalcr
   \(\parray}%
  {\endparray\)%
   \hsnewpar\belowdisplayskip
   \ignorespacesafterend}

\newcommand{\arrayhs}{\sethscode{arrayhscode}}

% The mathhscode environment also makes use of polytable's parray 
% environment. It is supposed to be used only inside math mode 
% (I used it to typeset the type rules in my thesis).

\newenvironment{mathhscode}%
  {\parray}{\endparray}

\newcommand{\mathhs}{\sethscode{mathhscode}}

% texths is similar to mathhs, but works in text mode.

\newenvironment{texthscode}%
  {\(\parray}{\endparray\)}

\newcommand{\texths}{\sethscode{texthscode}}

% The framed environment places code in a framed box.

\def\codeframewidth{\arrayrulewidth}
\RequirePackage{calc}

\newenvironment{framedhscode}%
  {\parskip=\abovedisplayskip\par\noindent
   \hscodestyle
   \arrayrulewidth=\codeframewidth
   \tabular{@{}|p{\linewidth-2\arraycolsep-2\arrayrulewidth-2pt}|@{}}%
   \hline\framedhslinecorrect\\{-1.5ex}%
   \let\endoflinesave=\\
   \let\\=\@normalcr
   \(\pboxed}%
  {\endpboxed\)%
   \framedhslinecorrect\endoflinesave{.5ex}\hline
   \endtabular
   \parskip=\belowdisplayskip\par\noindent
   \ignorespacesafterend}

\newcommand{\framedhslinecorrect}[2]%
  {#1[#2]}

\newcommand{\framedhs}{\sethscode{framedhscode}}

% The inlinehscode environment is an experimental environment
% that can be used to typeset displayed code inline.

\newenvironment{inlinehscode}%
  {\(\def\column##1##2{}%
   \let\>\undefined\let\<\undefined\let\\\undefined
   \newcommand\>[1][]{}\newcommand\<[1][]{}\newcommand\\[1][]{}%
   \def\fromto##1##2##3{##3}%
   \def\nextline{}}{\) }%

\newcommand{\inlinehs}{\sethscode{inlinehscode}}

% The joincode environment is a separate environment that
% can be used to surround and thereby connect multiple code
% blocks.

\newenvironment{joincode}%
  {\let\orighscode=\hscode
   \let\origendhscode=\endhscode
   \def\endhscode{\def\hscode{\endgroup\def\@currenvir{hscode}\\}\begingroup}
   %\let\SaveRestoreHook=\empty
   %\let\ColumnHook=\empty
   %\let\resethooks=\empty
   \orighscode\def\hscode{\endgroup\def\@currenvir{hscode}}}%
  {\origendhscode
   \global\let\hscode=\orighscode
   \global\let\endhscode=\origendhscode}%

\makeatother
\EndFmtInput
%


%import Data.List
%import Data.List


\subsection{Version 1}
\begin{hscode}\SaveRestoreHook
\column{B}{@{}>{\hspre}l<{\hspost}@{}}%
\column{22}{@{}>{\hspre}l<{\hspost}@{}}%
\column{36}{@{}>{\hspre}l<{\hspost}@{}}%
\column{E}{@{}>{\hspre}l<{\hspost}@{}}%
\>[B]{}\Varid{phi}{}\<[22]%
\>[22]{}\mathbin{::}\Conid{Int}\to (\Conid{Int}\to \Conid{Int}){}\<[E]%
\\[\blanklineskip]%
\>[B]{}\Varid{phi}\;\Varid{i}\;\Varid{j}{}\<[22]%
\>[22]{}\mathrel{=}\Varid{sequences}\mathbin{!!}\Varid{i}\mathbin{!!}\Varid{j}{}\<[E]%
\\[\blanklineskip]%
\>[B]{}\Varid{sequences}{}\<[22]%
\>[22]{}\mathrel{=}[\mskip1.5mu \mathrm{0},\mathrm{0}\mathinner{\ldotp\ldotp}\mskip1.5mu]\mathbin{:}(\Varid{tail}\;(\Varid{startWithAtLeast}\;\mathrm{0})){}\<[E]%
\\[\blanklineskip]%
\>[B]{}\Varid{startWithAtLeast}\;\Varid{n}{}\<[22]%
\>[22]{}\mathrel{=}\Varid{interleave}\;{}\<[36]%
\>[36]{}(\Varid{startWith}\;\Varid{n})\;{}\<[E]%
\\[\blanklineskip]%
\>[36]{}(\Varid{startWithAtLeast}\;(\Varid{n}\mathbin{+}\mathrm{1})){}\<[E]%
\\[\blanklineskip]%
\>[B]{}\Varid{startWith}\;\Varid{n}{}\<[22]%
\>[22]{}\mathrel{=}\Varid{map}\;(\Varid{n}\mathbin{:})\;\Varid{sequences}{}\<[E]%
\\[\blanklineskip]%
\>[B]{}\Varid{interleave}\;(\Varid{x}\mathbin{:}\Varid{xs})\;\Varid{ys}\mathrel{=}\Varid{x}\mathbin{:}\Varid{interleave}\;\Varid{ys}\;\Varid{xs}{}\<[E]%
\ColumnHook
\end{hscode}\resethooks

\subsection{Version 2}

\begin{hscode}\SaveRestoreHook
\column{B}{@{}>{\hspre}l<{\hspost}@{}}%
\column{28}{@{}>{\hspre}l<{\hspost}@{}}%
\column{41}{@{}>{\hspre}l<{\hspost}@{}}%
\column{E}{@{}>{\hspre}l<{\hspost}@{}}%
\>[B]{}\Varid{phi}{}\<[28]%
\>[28]{}\mathbin{::}\Conid{Int}\to (\Conid{Int}\to \Conid{Int}){}\<[E]%
\\[\blanklineskip]%
\>[B]{}\Varid{phi}{}\<[28]%
\>[28]{}\mathrel{=}\Varid{startWithAtLeast}\;\mathrm{0}{}\<[E]%
\\[\blanklineskip]%
\>[B]{}\Varid{startWithAtLeast}\;\Varid{n}{}\<[28]%
\>[28]{}\mathrel{=}\Varid{interleave}\;(\Varid{startWith}\;\Varid{n})\;{}\<[E]%
\\[\blanklineskip]%
\>[28]{}\hsindent{13}{}\<[41]%
\>[41]{}(\Varid{startWithAtLeast}\;(\Varid{n}\mathbin{+}\mathrm{1})){}\<[E]%
\ColumnHook
\end{hscode}\resethooks

\begin{hscode}\SaveRestoreHook
\column{B}{@{}>{\hspre}l<{\hspost}@{}}%
\column{18}{@{}>{\hspre}l<{\hspost}@{}}%
\column{28}{@{}>{\hspre}l<{\hspost}@{}}%
\column{E}{@{}>{\hspre}l<{\hspost}@{}}%
\>[B]{}\Varid{startWith}\;{}\<[18]%
\>[18]{}\Varid{n}\;\Varid{i}\;\mathrm{0}{}\<[28]%
\>[28]{}\mathrel{=}\Varid{n}{}\<[E]%
\\[\blanklineskip]%
\>[B]{}\Varid{startWith}\;{}\<[18]%
\>[18]{}\Varid{n}\;\Varid{i}\;(\Varid{j}\mathbin{+}\mathrm{1})\mathrel{=}\Varid{phi}\;\Varid{i}\;\Varid{j}{}\<[E]%
\ColumnHook
\end{hscode}\resethooks

\begin{hscode}\SaveRestoreHook
\column{B}{@{}>{\hspre}l<{\hspost}@{}}%
\column{28}{@{}>{\hspre}l<{\hspost}@{}}%
\column{E}{@{}>{\hspre}l<{\hspost}@{}}%
\>[B]{}\Varid{interleave}\;\Varid{fs}\;\Varid{gs}\;\mathrm{0}{}\<[28]%
\>[28]{}\mathrel{=}\Varid{fs}\;\mathrm{0}{}\<[E]%
\\[\blanklineskip]%
\>[B]{}\Varid{interleave}\;\Varid{fs}\;\Varid{gs}\;(\Varid{i}\mathbin{+}\mathrm{1}){}\<[28]%
\>[28]{}\mathrel{=}\Varid{interleave}\;\Varid{gs}\;(\Varid{fs}\mathbin{\circ}(\mathbin{+}\mathrm{1}))\;\Varid{i}{}\<[E]%
\ColumnHook
\end{hscode}\resethooks

\subsection{Version 3}

\begin{hscode}\SaveRestoreHook
\column{B}{@{}>{\hspre}l<{\hspost}@{}}%
\column{13}{@{}>{\hspre}l<{\hspost}@{}}%
\column{20}{@{}>{\hspre}l<{\hspost}@{}}%
\column{22}{@{}>{\hspre}l<{\hspost}@{}}%
\column{25}{@{}>{\hspre}l<{\hspost}@{}}%
\column{42}{@{}>{\hspre}l<{\hspost}@{}}%
\column{E}{@{}>{\hspre}l<{\hspost}@{}}%
\>[B]{}\Varid{phi}{}\<[22]%
\>[22]{}\mathbin{::}\Conid{Int}\to (\Conid{Int}\to \Conid{Int}){}\<[E]%
\\[\blanklineskip]%
\>[B]{}\Varid{phi}\;\Varid{i}\;\mathrm{0}{}\<[13]%
\>[13]{}\mid \Varid{even}\;\Varid{i}\mathrel{=}{}\<[25]%
\>[25]{}\mathrm{0}{}\<[E]%
\\[\blanklineskip]%
\>[13]{}\mid \Varid{odd}\;{}\<[20]%
\>[20]{}\Varid{i}\mathrel{=}{}\<[25]%
\>[25]{}\Varid{phi}\;(\Varid{i}\mathbin{\Varid{`div`}}\mathrm{2})\;{}\<[42]%
\>[42]{}\mathrm{0}\mathbin{+}\mathrm{1}{}\<[E]%
\\[\blanklineskip]%
\>[B]{}\Varid{phi}\;\Varid{i}\;(\Varid{j}\mathbin{+}\mathrm{1})\mid \Varid{even}\;\Varid{i}\mathrel{=}{}\<[25]%
\>[25]{}\Varid{phi}\;(\Varid{i}\mathbin{\Varid{`div`}}\mathrm{2})\;\Varid{j}{}\<[E]%
\\[\blanklineskip]%
\>[B]{}\hsindent{13}{}\<[13]%
\>[13]{}\mid \Varid{odd}\;{}\<[20]%
\>[20]{}\Varid{i}\mathrel{=}{}\<[25]%
\>[25]{}\Varid{phi}\;(\Varid{i}\mathbin{\Varid{`div`}}\mathrm{2})\;(\Varid{j}\mathbin{+}\mathrm{1}){}\<[E]%
\ColumnHook
\end{hscode}\resethooks

   



\end{document}



